%%
%
\begin{description}
%PATTERN: \option{}[]{}
%OUTPUT:  <dt>#1 (keys: #3; formerly code #2)<dt>
%OUTPUT:  <dd>#x<dd>
%OUTPUT:
\def\option#1[#2]#3{\item[#1:] (key: \code{#3} formerly \code{#2})}

\option{Text}[1]{text, word, wd, w}
 Count as text (i.e. count words).
\option{Header text}[2]{headertext, headerword, hword, hwd, hw}
 Count as header text.
\option{Other text}[3]{otherword, other, oword, owd, ow}
 Count as float/caption text.
\option{Displaymath}[7]{displaymath, dsmath, dmath, ds}
 Count as displayed math formulae.
\option{Inline math}[6]{inlinemath, inline, imath, eq}
 Count as inlined math formulae.
\option{To header}[4]{header, heading, head}
 Count header, then count text as \code{headertext} (transition state).
\option{To float}[5]{float, table, figure}
 Count float, then parse contents as \code{isfloat} (transition state).
\option{Preamble}[-9]{}
 Parse as preamble, i.e. ignore text but look for \code{preambleinclude} macros.
\option{Ignore}[0]{ignore}
 Ignore text, i.e. do not count, but will still parse the code.
\option{Float}[-1]{isfloat}
 Float contents, ignore text but look for \code{floatinclude} macros.
\option{Strong exclude}[-2]{xx}
 Strong ignore which ignore environments, e.g. to use in macro definitions where
 \code{\bs{begin}}--\code{\bs{end}} need not be balanced.
\option{Stronger exclude}[-3]{xxx}
 Stronger ignore, handles macros as isolated tokens without handling their parameters,
 to use with macro definitions like \code{\bs{newcommand}} and \code{\bs{def}}.
\option{Exclude all}[-4]{xall}
 Ignore all, including unbalanced braces (e.g. used by \code{\%TC:ignore} and the \code{verbatim} environment). This rule may be used for environment contents, but not for macro or environment parameters or options since the exclusion causes \{ and [ to be ignored.
\end{description}
