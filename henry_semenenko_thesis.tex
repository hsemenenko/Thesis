% Require the latest version of pdf and compress
%\pdfminorversion		= 7
%\pdfobjcompresslevel	= 9
%
%\title{University of Bristol Thesis Template}
\RequirePackage[l2tabu]{nag}		% Warns for incorrect (obsolete) LaTeX usage
%


% File: henry_semenenko_thesis.tex
% Author: Henry Semenenko
% Description: Contains the thesis template using memoir class,
% which is mainly based on book class but permits better control of 
% chapter styles for example. This template is an adaptation and 
% modification of Oscar's.	
% 
% Memoir is a flexible class for typesetting poetry, fiction, 
% non-fiction and mathematical works as books, reports, articles or
% manuscripts. CTAN repository is found at:
% http://www.ctan.org/tex-archive/macros/latex/contrib/memoir/
%
%
% Formatting guidelines for the thesis can be found here:
% http://www.bristol.ac.uk/academic-quality/pg/pgrcode/annex4/
%
% UoB guidelines:
%
% The hard copies of the dissertation required for the examination 
% process must be printed on A4 white paper. Paper up to A3 may 
% be used for maps, plans, diagrams and illustrative material. 
% Pages (apart from the preliminary pages) should normally be 
% double-sided.
%
% Memoir class loads useful packages by default (see manual).
\documentclass[a4paper,11pt,reqno,oldfontcommands]{memoir} %add 'draft' to turn draft option on (see below)
%

%Defining custom colours
\usepackage{xcolor}

%Bright colours
\definecolor{bristol-red}{HTML}{AB1F2D}
\definecolor{aqua}{HTML}{06827D}  
\definecolor{blue}{HTML}{1C7698}
\definecolor{purple}{HTML}{806AB7}

%Dark colours

\newcommand{\brisred}[1]{\textcolor{bristol-red}{#1}}	
	
%Define command for things that need to be checked in the text
\newcommand{\checkthis}[1]{\textcolor{red}{\textbf{#1}}}	
	
%
%Loading date package and saving date
\usepackage[en-GB,calc]{datetime2}
\DTMsavenow{now}
% When draft option is on. 
\ifdraftdoc 
	\usepackage{draftwatermark}				%Sets watermarks up.
	\SetWatermarkScale{0.25}
	\SetWatermarkText{\textbf{Draft: \today}}
\fi
%
% Declare figure/table as a subfloat.
\newsubfloat{figure}
\newsubfloat{table}
% Better page layout for A4 paper, see memoir manual.
\settrimmedsize{297mm}{210mm}{*}
\setlength{\trimtop}{0pt} 
\setlength{\trimedge}{\stockwidth} 
\addtolength{\trimedge}{-\paperwidth} 
\settypeblocksize{634pt}{448.13pt}{*} 
\setulmargins{4cm}{*}{*} 
\setlrmargins{*}{*}{1.5} 
\setmarginnotes{17pt}{51pt}{\onelineskip} 
\setheadfoot{\onelineskip}{2\onelineskip} 
\setheaderspaces{*}{2\onelineskip}{*} 
\checkandfixthelayout
%
% Change the font
\frenchspacing
\usepackage[T1]{fontenc}  
\usepackage[sfdefault]{roboto}
\usepackage{roboto-mono}
\fontseries{m}
%

%Defining an empty page with key graphic
\def\emptypage{
	\thispagestyle{empty}
	\vspace*{\fill}
	\vspace*{-1.5cm}
	\begin{center}
	\checkoddpage
	\ifoddpage
		\includegraphics[scale=0.4]{./graphics/key_right.pdf}
	\else
		\includegraphics[scale=0.4]{./graphics/key_left.pdf}
	\fi
	\end{center}     
    \vspace{\fill} 
}

% UoB guidelines:
%
% Text should be in double or 1.5 line spacing, and 
% font size should be chosen to ensure clarity and 
% legibility for the main text and for any quotations 
% and footnotes.
%
% Note: This is automatically set by memoir class. Nevertheless \OnehalfSpacing 
% enables double spacing but leaves single spaced for captions for instance. 
\OnehalfSpacing 
%
% Sets numbering division level
\setsecnumdepth{subsection} 
\maxsecnumdepth{subsubsection}
%
% Chapter style
\usepackage{soul}
\makeatletter 
\newlength\dlf@normtxtw 
\setlength\dlf@normtxtw{\textwidth} 
\newsavebox{\feline@chapter} 
\newcommand\feline@chapter@marker[1][4cm]{%
	\sbox\feline@chapter{% 
		\resizebox{!}{#1}{\fboxsep=1pt%
			{\ifdraftdoc\color{gray}\else\color{white}\fi\thechapter}% 
			}
		}%
		\quad%
		\raisebox{\depthof{\usebox{\feline@chapter}}}{\usebox{\feline@chapter}}%
} 
\newcommand\feline@chm[1][4cm]{%
	\sbox\feline@chapter{\feline@chapter@marker[#1]}% 
	\makebox[0pt][c]{% aka \rlap
		\makebox[1cm][r]{\usebox\feline@chapter}%
	}}
\makechapterstyle{daleifmodif}{
	\renewcommand\chapnamefont{\normalfont\Large\scshape\raggedleft\so} 
	\renewcommand\chaptitlefont{\normalfont\huge\bfseries\scshape} 
	\renewcommand\chapternamenum{} \renewcommand\printchaptername{} 
	\renewcommand\printchapternum{\thispagestyle{empty}\ifdraftdoc\else\pagecolor{bristol-red}\fi\null\hfill\feline@chm[2.5cm]\par} 
	\renewcommand\afterchapternum{\par\vspace{25pt}\vskip\midchapskip} 
	\renewcommand\printchaptertitle[1]{
		\if@mainmatter
			\ifdraftdoc
				\color{gray}
			\else
				\color{white}
			\fi
			\chaptitlefont\raggedleft ##1\par
			\begin{figure*}
				\centering
				\includegraphics[scale=0.25]{./graphics/skytale}
			\end{figure*}				
		\else
			\color{gray}\chaptitlefont\raggedleft ##1\par
		\fi
	}
	\renewcommand{\mempostaddchaptertotochook}{\clearpage\nopagecolor\cleartooddpage[\emptypage]\glsresetall\suppressfloats}
} 
\makeatother 
\chapterstyle{daleifmodif}
%
% UoB guidelines:
%
% The pages should be numbered consecutively at the bottom centre of the
% page.
\makepagestyle{myvf} 
\makeoddfoot{myvf}{}{\thepage}{} 
\makeevenfoot{myvf}{}{\thepage}{} 
\makeheadrule{myvf}{\textwidth}{\normalrulethickness} 
\makeevenhead{myvf}{\small\textsc{\leftmark}}{}{} 
\makeoddhead{myvf}{}{}{\small\textsc{\rightmark}}
\pagestyle{myvf}
%

%Command to check if in the appendix
\makeatletter
\newcommand{\inappendix}{TT\fi\expandafter\ifx\@chapapp\appendixname}
\makeatother


% Define command to create clear pages between chapters
% This will clear to next odd page
\def\clearemptydoublepage{%
	\if\inappendix
        \clearpage
        \cleartooddpage[\emptypage]
	\else
        \clearpage%
        \cleartoevenpage[\emptypage]
        \clearpage
        \cleartooddpage[\emptypage]
    \fi
    }%
%
%
% Creates indexes for Table of Contents, List of Figures, List of Tables and Index
\makeindex
% \printglossaries below creates a list of abbreviations. \gls and related
% commands are then used throughout the text, so that latex can automatically
% keep track of which abbreviations have already been defined in the text.
%
% The import command enables each chapter tex file to use relative paths when
% accessing supplementary files. For example, to include
% chapters/brewing/images/figure1.png from chapters/brewing/brewing.tex we can
% use
% \includegraphics{images/figure1}
% instead of
% \includegraphics{chapters/brewing/images/figure1}
\usepackage{import}

% Add other packages needed for chapters here. For example:
\usepackage{lipsum}
\usepackage{nicefrac}
\usepackage{braket}
\usepackage{amsfonts} 					%Calls Amer. Math. Soc. (AMS) fonts
\usepackage[centertags]{mathtools}		%Writes maths centred down
\usepackage{stmaryrd}					%New AMS symbols
\usepackage{amssymb}					%Calls AMS symbols
\usepackage{amsthm}						%Calls AMS theorem environment
\usepackage{graphicx}					%Calls figure environment 
%\usepackage[utf8]{inputenc}				%Needed to encode non-english characters 
%										%directly for mac
%\usepackage{mathrsfs}					%Even more math symbols
\usepackage{float}						%Helps to place figures, tables, etc. 
\usepackage{latexsym}					%Extra symbols
\usepackage{cite}
\usepackage[hyphens]{url}				%Supports url commands
\urlstyle{same}
\usepackage[UKenglish]{babel}				%For languages characters and hyphenation
\usepackage{enumerate}					%For enumeration counter
\usepackage{enumitem}					%This is used to change the font for the enumeration numbers
\setlist[enumerate]{font=\robotoTLF}	%This changes the font to a lining font making the numbers uniformly spaced
\usepackage{footnote}					%For footnotes
\usepackage{microtype}					%Makes pdf look better.
\usepackage{rotfloat}					%For rotating and float environments as tables, 
										%figures, etc. 
\usepackage[version=0.96]{pgf}			%PGF/TikZ is a tandem of languages for producing 
										%vector graphics from a 
\usepackage{tikz}						%geometric/algebraic description.
\usepackage{siunitx}					%package to format SI units
\sisetup{
	detect-all,
	separate-uncertainty,
	multi-part-units=single,
	range-units = single,
	product-units=power
}
%TC:macroword \SI 1
%TC:macroword \SIrange 1

%	detect-all,
%	detect-weight,
%	detect-inline-family = text,
%	detect-inline-weight = math
%}
\DeclareSIUnit[]{\Vpp}{\ensuremath{\si{\V}_\text{pp}}}

\usetikzlibrary{arrows,shapes,decorations,
		       automata,backgrounds,			
		       petri,topaths}				%To use diverse features from tikz	

% Uncomment this to hide page numbers for chapters in toc
%\usepackage{tocloft}
%\cftpagenumbersoff{chapter} 

%%% PGFPLOTS STUFF %%%

\usepackage{pgfplots}
\pgfplotsset{compat=newest}
\usepgfplotslibrary{colorbrewer}

%input the colour definitions
%Cycle colours: https://gka.github.io/palettes/#/9|d|ab1f2d,ffc12e|ffc12e,411f78|1|1
%TC:ignore
\definecolor{RdYlPu-1-1}{HTML}{ab1f2d}

\definecolor{RdYlPu-2-1}{HTML}{ab1f2d}
\definecolor{RdYlPu-2-2}{HTML}{411f78}

\definecolor{RdYlPu-3-1}{HTML}{ab1f2d}
\definecolor{RdYlPu-3-2}{HTML}{ffc12e}
\definecolor{RdYlPu-3-3}{HTML}{411f78}

\definecolor{RdYlPu-8-1}{HTML}{ab1f2d}
\definecolor{RdYlPu-8-2}{HTML}{c15030}
\definecolor{RdYlPu-8-3}{HTML}{d77731}
\definecolor{RdYlPu-8-4}{HTML}{eb9c31}
\definecolor{RdYlPu-8-5}{HTML}{d49550}
\definecolor{RdYlPu-8-6}{HTML}{a96b62}
\definecolor{RdYlPu-8-7}{HTML}{7a446f}
\definecolor{RdYlPu-8-8}{HTML}{411f78}

%New cycle lists
\pgfplotscreateplotcyclelist{RdYlPu-1}{%
	{RdYlPu-1-1, mark = none, thick}
}
\pgfplotscreateplotcyclelist{RdYlPu-2}{%
	{RdYlPu-2-1, mark = none, thick},
	{RdYlPu-2-2, mark = none, thick}
}
\pgfplotscreateplotcyclelist{RdYlPu-3}{%
	{RdYlPu-3-1, mark = none, thick},
	{RdYlPu-3-2, mark = none, thick},
	{RdYlPu-3-3, mark = none, thick}
}
\pgfplotscreateplotcyclelist{RdYlPu-8}{%
	{RdYlPu-8-1, mark = none, thick},
	{RdYlPu-8-2, mark = none, thick},
	{RdYlPu-8-3, mark = none, thick},
	{RdYlPu-8-4, mark = none, thick},
	{RdYlPu-8-5, mark = none, thick},
	{RdYlPu-8-6, mark = none, thick},
	{RdYlPu-8-7, mark = none, thick},
	{RdYlPu-8-8, mark = none, thick}
}
%TC:endignore

\makeatletter \newcommand{\pgfplotsdrawaxis}{\pgfplots@draw@axis} \makeatother

\pgfplotsset{axis line on top/.style={
  axis line style=transparent,
  ticklabel style=transparent,
  tick style=transparent,
  axis on top=false,
  after end axis/.append code={
    \pgfplotsset{axis line style=opaque,
      ticklabel style=opaque,
      tick style=opaque,
      grid=none}
    \pgfplotsdrawaxis}
  }
}

%%% END %%%
		  
%%% hyperref must be loaded before Glossaries-extra or hyperlinks go to the wrong place %%%
	
\usepackage[colorlinks=true,
			pdfpagelayout=TwoColumnRight,			%Makes pdf two page scrolling
			pagebackref,
			allcolors=black]{hyperref}              %Creates hyperlinks in cross references
			
\usepackage{caption}		%Load this after hyperref to make figure references link to the top of the figure

%Define the pdf info
\hypersetup{
	pdfinfo={
		Title={Advances in Chip-Based Quantum Key Distribution},
		Author={Henry Semenenko},
		Keywords={integrated optics; quantum key distribution; quantum optics},	
		CreationDate={D:\DTMfetchyear{now}\DTMfetchmonth{now}\DTMfetchday{now}\DTMfetchhour{now}\DTMfetchminute{now}\DTMfetchsecond{now}},
		Subject={A thesis submitted to the University of Bristol for the degree of Doctor of Philisophy in the Faculty of Science}
	}
}	  	  
	
%%% GLOSSARIES STUFF %%%
\usepackage[acronym,
						toc=false,
						symbols,
						section,
						nogroupskip,
						shortcuts=ac,
						automake]{glossaries-extra}
	%Set acronym style
	\setabbreviationstyle{long-short}
	\setabbreviationstyle[acronym]{long-short}
	%Load entries
	\loadglsentries{glossaryEntries}
	% Something important
	\makeglossaries
	%Add all entries
	\glsaddall
	% This command makes sure the entry name is in bold
	\renewcommand{\glsnamefont}[1]{\textbf{#1}}
	% Make the second column a bit wider
	\setlength{\glsdescwidth}{0.7\textwidth}
	% GLS style modified from that shown on https://tex.stackexchange.com/questions/161443/glossary-style-long-but-without-the-indentation-before-the-notations
	\newglossarystyle{mylong}{%
		\renewenvironment{theglossary}%
			{\begin{longtable}[l]{lp{\glsdescwidth}p{\glspagelistwidth}}}%
			{\end{longtable}}%
		\renewcommand*{\glossaryheader}{}%
		\renewcommand*{\glsgroupheading}[1]{}%
		\renewcommand{\glossentry}[2]{%
			\glsentryitem{##1}\glstarget{##1}{\glossentryname{##1}} &
			%\glossentrysymbol{##1} &
			\Glossentrydesc{##1} &
			##2\tabularnewline
		}%
		\renewcommand{\subglossentry}[3]{%
			&
			\glssubentryitem{##2}%
			%\glossentrysymbol{##2} &
			\glstarget{##2}{\strut}\glossentrydesc{##2} & ##3\tabularnewline
		}%
		\renewcommand*{\glsgroupskip}{%
			\ifglsnogroupskip\else & & \tabularnewline\fi}%
	}
	
%TC:macroword \ac 1
%TC:macroword \acl 1
%TC:macroword \acp 1
%TC:macroword \acs 1
%TC:macroword \aclp 1
%TC:macroword \Ac 1
%TC:macroword \Acl 1
%TC:macroword \Acp 1
%TC:macroword \Acs 1
%TC:macroword \Aclp 1

%%% END OF GLOSSARIES STUFF %%%       	
	
%%% Bibliography Stuff %%%
  
%Changing the style of pagebackref in the bibliography
\renewcommand*{\backref}[1]{
		% default interface
		% #1: backref list
		%
		% We want to use the alternative interface,
		% therefore the definition is empty here.
	}
\renewcommand*{\backrefalt}[4]{%
	% alternative interface
	% #1: number of distinct back references
	% #2: backref list with distinct entries
	% #3: number of back references including duplicates
	% #4: backref list including duplicates
	\par
	\ifnum#3=1 %
		Page %
	\else
		Pages %
	\fi
	#2.\par
}
	  
	       	
%							
%Reduce widows  (the last line of a paragraph at the start of a page) and orphans 
% (the first line of paragraph at the end of a page)
\widowpenalty=1000
\clubpenalty=1000


%
%
% New definition of square root:
% it renames \sqrt as \oldsqrt
\let\oldsqrt\sqrt
% it defines the new \sqrt in terms of the old one
\def\sqrt{\mathpalette\DHLhksqrt}
\def\DHLhksqrt#1#2{%
\setbox0=\hbox{$#1\oldsqrt{#2\,}$}\dimen0=\ht0
\advance\dimen0-0.2\ht0
\setbox2=\hbox{\vrule height\ht0 depth -\dimen0}%
{\box0\lower0.4pt\box2}}

% Theorem styles
%
\theoremstyle{plain}
\newtheorem{theorem}{Theorem}[chapter]

\BeforeBeginEnvironment{theo}{\begin{adjustwidth}{1cm}{1cm}}
\AfterEndEnvironment{theo}{\end{adjustwidth}}

%
% Expand tables
%

\setlength{\tabcolsep}{6pt}
\renewcommand{\arraystretch}{1.2}

%
% Expand equations
%
\addtolength{\jot}{1em}

%
% Hyphenation for some words
%
\hyphenation{res-pec-tively}
\hyphenation{hypo-the-sis}
\hyphenation{para-me-ters}
\hyphenation{sol-va-bi-li-ty}
\hyphenation{quan-tum}
\hyphenation{cryp-to-gra-phy}
\hyphenation{in-for-ma-tion}
\hyphenation{mea-sure-ment}
\hyphenation{in-de-pen-dent}
\hyphenation{metro-pol-it-an}
%
%
%% Change the space between paragraphs
\abnormalparskip{3pt}

\begin{document}

% UoB guidlines:
%
% Preliminary pages
% 
% The five preliminary pages must be the Title Page, Abstract, Dedication
% and Acknowledgements, Author's Declaration and Table of Contents.
% These should be single-sided.
% 
% Table of contents, list of tables and illustrative material
% 
% The table of contents must list, with page numbers, all chapters,
% sections and subsections, the list of references, bibliography, list of
% abbreviations and appendices. The list of tables and illustrations
% should follow the table of contents, listing with page numbers the
% tables, photographs, diagrams, etc., in the order in which they appear
% in the text.
% 
\frontmatter
\pagenumbering{gobble}
%
%
% File: Title.tex
% Author: Henry Semenenko
% Description: Title page
%
% UoB guidelines:
% 
% At the top of the title page, within the margins, the dissertation should give the title and, if 
% necessary, sub-title and volume number. If the dissertation is in a language other than English, the 
% title must be given in that language and in English. The full name of the author should be in the 
% centre of the page. At the bottom centre should be the words ?A dissertation submitted to the 
% University of Bristol in accordance with the requirements for award of the degree of ? in the 
% Faculty of ...?, with the name of the school and month and year of submission. The word count of 
% the dissertation (text only) should be entered at the bottom right-hand side of the page.
%
%
\begin{titlingpage}
\pagecolor{bristol-red}
\begin{SingleSpace}
\calccentering{\unitlength} 
\begin{adjustwidth*}{\unitlength}{-\unitlength}
\vspace*{5mm}
\begin{center}
\vspace{20mm}
%\rule[0.5ex]{\linewidth}{2pt}\vspace*{-\baselineskip}\vspace*{3.2pt}
%\rule[0.5ex]{\linewidth}{1pt}\\[\baselineskip]
{\HUGE \bf \color{white} Advances in Chip-Based\\Quantum Key Distribution }\\[4mm]
%{\Large \textit{Subtitle}}\\
%\rule[0.5ex]{\linewidth}{1pt}\vspace*{-\baselineskip}\vspace{3.2pt}
%\rule[0.5ex]{\linewidth}{2pt}\\
\vspace{30mm}
{\huge\color{white}\textsc{Henry Semenenko}}\\
\vspace*{\fill}
\includegraphics[scale=0.35]{logos/bristollogo_white}\\
\end{center}
\end{adjustwidth*}
\end{SingleSpace}
\newpage\thispagestyle{empty}
\begin{SingleSpace}
\calccentering{\unitlength} 
\begin{adjustwidth*}{\unitlength}{-\unitlength}
\vspace*{0mm}
\begin{center}
\vspace{20mm}
{\color{white}\Large Department of Physics\\
\textsc{University of Bristol}}\\
\vspace{40mm}
\begin{minipage}{8.5cm}
\color{white}
A dissertation submitted to the University of Bristol in accordance with the requirements of the degree of \textsc{Doctor of Philosophy} in the Faculty of Science.
\end{minipage}\\
\vspace*{\fill}
{\color{white}Word count: more than 4}\\
\vspace{20mm}
{\color{white}\large\textsc{\DTMmonthname{\DTMfetchmonth{now}} \DTMfetchyear{now}}}
\end{center}
\end{adjustwidth*}
\end{SingleSpace}
\clearpage
\nopagecolor
\end{titlingpage}
%
%
% File: declaration.tex
% Author: Henry Semenenko
% Description: Contains the declaration page
%
% UoB guidelines:
%
% Author's declaration
%
% I declare that the work in this dissertation was carried out in accordance
% with the requirements of the University's Regulations and Code of Practice
% for Research Degree Programmes and that it has not been submitted for any
% other academic award. Except where indicated by specific reference in the
% text, the work is the candidate's own work. Work done in collaboration with,
% or with the assistance of, others, is indicated as such. Any views expressed
% in the dissertation are those of the author.
%
% SIGNED: .............................................................
% DATE:..........................
%
\chapter*{Author's declaration}
\hspace{-0.75cm}
\begin{SingleSpace}
%\begin{quote}
\noindent I declare that the work in this dissertation was carried out in accordance with the requirements of  the University's Regulations and Code of Practice for Research Degree Programmes and that it  has not been submitted for any other academic award. Except where indicated by specific  reference in the text, the work is the candidate's own work. Work done in collaboration with, or with the assistance of, others, is indicated as such. Any views expressed in the dissertation are those of the author.

\vspace{3cm}
\noindent
\textsc{SIGNED:} \dotfill \textsc{DATE:} \dotfill

\vspace*{\fill}

%\end{quote}
\end{SingleSpace}
\clearpage
\clearemptydoublepage
%
%
% File: abstract.tex
% Author: V?ctor Bre?a-Medina
% Description: Contains the text for thesis abstract
%
% UoB guidelines:
%
% Each copy must include an abstract or summary of the dissertation in not
% more than 300 words, on one side of A4, which should be single-spaced in a
% font size in the range 10 to 12. If the dissertation is in a language other
% than English, an abstract in that language and an abstract in English must
% be included.

\chapter*{Abstract}
\begin{SingleSpace}
Here goes the short summary of the work.....

\vspace{50mm}
\hspace{60mm} ...which is easy....

\vspace{50mm}
\hspace{120mm} ...nothing.
\end{SingleSpace}
\clearpage
\clearemptydoublepage
%
%
% file: dedication.tex
% author: Henry Semenenko
%

\chapter*{Acknowledgements}
I would firstly like to thank Dr Chris Erven for his supervision and encouragement throughout my PhD and I am ever grateful to Dr Philip Sibson for his patience and Wikipedia-like knowledge. Thank you to Dr David Lowndes and Dr Alasdair Price for their (often not misguided) advice and Prof. John Rarity FRS for his endless inspiring tales. Thank you to Dr Djeylan Aktas, Dr Jorge Barreto, Friederike J\"{o}hlinger, Dr Alasdair Price and Lawrence Rosenfeld for taking the time to proofread this thesis.

I am indebted to each and every member of QETLabs of which there are, unfortunately, far to many to name here. 
I am incredibly appreciative of the support from the ops team, past and present, without whom I would have never been able to achieved half of the things I was able to. In particular to Lorraine, Holly, Emma, Helen, Andrea, Lin and Rebecca.

Thank you to the whole of cohort 2, Alex, Dan, Geraint, Jason, Joe, Lawrence, Lucio, Martin and Sam, who have made this journey so memorable; to Dr. Milica Prodik and Neil Simmons for providing the artwork that has been used to decorate this thesis; to Dr Graham Marshall for your help despite my continual harassment; and to Andy Murray for his unrivalled technical knowledge. 

Finally, I want to thank my family for their support, inspiration and for getting me here in one piece. Thank you to my friends for trying their best to make it more than one piece.

\clearpage
\clearemptydoublepage
%
\pagenumbering{roman} 
\renewcommand{\contentsname}{Table of Contents}
\maxtocdepth{subsection}
\tableofcontents*
\addtocontents{toc}{\par\nobreak \mbox{}\hfill{\textbf{Page}}\par\nobreak}
\clearemptydoublepage
%
%
% File: publications.tex
% Author: Henry Semenenko
% Description: Lists the publications from the PhD
{\setlength{\parindent}{0pt} %remove the indentation
\abnormalparskip{18pt} %increase the paragraph skip length
\chapter*{List of Publications}
\addcontentsline{toc}{chapter}{List of Publications}
\section*{Journals}

H. Semenenko, P. Sibson, M. G. Thompson and C. Erven, "Interference between independent photonic integrated devices for quantum key distribution," \textit{Opt. Lett. 44, 275-278}, Jan. 2019 

H. Semenenko, P. Sibson, A. Hart, M. G. Thompson and C. Erven, "Chip-based measurement-device-independent quantum key distribution," \textit{in preparation.} 

\section*{Conferences}

H. Semenenko, P. Sibson, A. Hart, M. G. Thompson and C. Erven, ``Chip-based measurement-device-independent quantum key distribution,'' \textit{oral presentation}, QCrypt, Aug. 2019

H. Semenenko, P. Sibson, M. G. Thompson and C. Erven, ``Integrated devices for measurement-device-independent quantum key distribution,'' \textit{oral presentation}, CLEO, Apr. 2019

P. Sibson, D. Lowndes, S. Frick, A. Price, H. Semenenko, F. Raffaelli, D. Llewellyn, J. Kennard, Y. Ou, F. Ntavou, E. Hugues Salas, A. Hart, R. Collins, A. Laing, C. Erven, R. Nejabati, D. Simeonidou, M. G. Thompson and J. G. Rarity, ``Networked Quantum-Secured Communications with Hand-held and Integrated Devices: Bristol’s Activities in the UK Quantum Communications Hub,'' \textit{oral presentation given by P. Sibson}, QCrypt, Sep. 2017

H. Semenenko, P. Sibson, J. Barreto and C. Erven, ``Towards Accessible Metropolitan Quantum Secure Communication,'' \textit{oral presentation}, Young Quantum Information Scientist, Sep., 2017 

A. Vaquero-Stainer, R. A. Kirkwood, V. Burenkov, C. J. Chunnilall, A. G Sinclair, A. Hart, H. Semenenko, P. Sibson, C. Erven and M. G. Thompson, ``Measurements towards providing security assurance for a chip-scale QKD system,'' \textit{oral presentation given by A. Vaquero-Stainer}, Proc. SPIE 10674, Quantum Technologies, May 2018

H. Semenenko, P. Sibson, M. G. Thompson and C. Erven, ``Integrated Photonic Devices for Measurement-Device-Independent Quantum Key Distribution,'' \textit{poster}, QCrypt, Aug. 2018

D. V. Aktas, P. Sibson, D. Lowndes, S. Frick, A. Price, H. Semenenko, F. Raffaelli, D. Llewellyn, J. Kennard, Y. Ou, F. Ntavou, E. Hugues Salas, A. Hart, R. Collins, A. Laing, C. Erven, R. Nejabati, D. Simeonidou, M. G. Thompson and J. G. Rarity, ``A Metropolitan Quantum Network with Hand-Held and Integrated Devices,'' \textit{poster presentation given by D. V. Aktas}, GDR IQFA, Nov. 2018
}
\clearpage
\clearemptydoublepage
%
\phantomsection
\addcontentsline{toc}{chapter}{List of Tables and Figures}
\listoftables*
\addtocontents{lot}{\par\nobreak\textbf{{\scshape Table} \hfill Page}\par\nobreak}
%\clearemptydoublepage
%
\listoffigures*
\addtocontents{lof}{\par\nobreak\textbf{{\scshape Figure} \hfill Page}\par\nobreak}
\clearemptydoublepage
%
\chapter*{List of Acronyms}
\addcontentsline{toc}{chapter}{List of Acronyms}
\printglossary[type=acronym, nonumberlist, style=mylong, title = {}]
\clearemptydoublepage
%
%
% The bulk of the document is delegated to these chapter files in
% subdirectories.
\mainmatter
%
\import{chapters/chapter01/}{chap01.tex}
\clearemptydoublepage
\import{chapters/chapter02/}{chap02.tex}
\clearemptydoublepage
\import{chapters/chapter03/}{chap03.tex}
\clearemptydoublepage
\import{chapters/chapter04/}{chap04.tex}
\clearemptydoublepage
\import{chapters/chapter06/}{chap06.tex}
\clearemptydoublepage
\import{chapters/chapter07/}{chap07.tex}
\clearemptydoublepage

% And the appendix goes here
\appendix
\import{chapters/appendices/}{app0.tex}
\clearemptydoublepage
\import{chapters/appendices/}{app-tips.tex}
\clearemptydoublepage
%
% The guidelines don't say anything about citations or
% bibliography styles so I guess we can use anything.
\backmatter
\bibliographystyle{ieeetr}
\bibliography{thesisbiblio}
\refstepcounter{chapter}
\ifdraftdoc
	\cleartoevenpage
\else
    \clearpage
    \pagestyle{empty}
    \cleartooddpage[\emptypage]
    \clearpage
    \pagestyle{empty}
    \pagecolor{bristol-red} 
    \cleartoevenpage[\pagestyle{empty}]
	\vspace*{\fill}
	\begin{center}
	\includegraphics[scale=0.25]{./graphics/skytale}
	\end{center}
\fi
%
% Add index
%\printindex
%   
\end{document}