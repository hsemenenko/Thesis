%
% File: chap02.tex
% Author: Henry Semenenko
% Description: Background chapter
%
% Reset all acronyms 
\glsresetall
% Set the graphics path to find figures
\graphicspath{{./chapters/chapter02/fig02/}}

\let\textcircled=\pgftextcircled
\chapter{Background}
\label{chap:background}

%=======
\section{Cryptography}
\label{sec1:crypto}

The idea of obscuring messages from third-party onlookers dates back (as far as we can tell) to ancient Egypt. With the convenience of being able to share thoughts through written methods came with an immediate compromise to security.

\subsection{Symmetric Key Encryption}

\subsection{Public Key Cryptography}

A more practical method of encrypting data is to 

\section{Quantum Theory}

\subsection{Quantum Information}

\subsubsection{Time-bin Encoding}

\subsection{Quantum Photonics}

\section{Quantum Key Distribution}

BB84 \cite{BB84}. Shor \cite{shor1994}.

\subsection{Protocols}

\subsection{Hacking}

While \ac{QKD} is often claimed to be ``perfectly secure'' or ``unhackable'', this is only true under certain assumptions. 

There have been numerous demonstrations that \ac{QKD} system are insecure under real world condition.

\subsubsection*{Device Independence}

To remove potential attacks on \ac{QKD} systems, protocols have been developed to relieve some of the assumptions required for security in a key exchange. 

Creating accurate models for equipment is particularly challenging. By removing assumptions about their operation, full characterisation is no longer required.

\section{Integrated Photonics Devices}

\subsection{Requirements}

\subsection{Indium Phosphide}

\subsubsection*{Lasers}

\subsubsection*{Electro-optic Modulation}

\subsubsection*{Thermo-optic Modulations}

\subsubsection*{Mach-Zehnder Interferometers}

\subsection{Silicon}

\subsection{Integrated Detectors}

\section{Summary}


%=========================================================