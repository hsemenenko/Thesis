%
% File: chap02.tex
% Author: Henry Semenenko
% Description: Background chapter
%
% Set the graphics path to find figures
\graphicspath{{./chapters/chapter02/fig02/}}

\let\textcircled=\pgftextcircled
\chapter{Background}
\label{chap:background}

As with any work of research, it is built upon centuries of previous work. Here we highlight some important advances which will be required for the rest of the thesis.

%=======
\section{Cryptography}
\label{sec1:crypto}

Cryptography is the process of taking a message and translating it into a ciphertext. Importantly, it also encompasses the reverse process of taking a ciphertext and returning the original message. The idea of obscuring messages from third-party onlookers dates back (as far as we can tell) to ancient Egypt and has played an important role in the world for the last 2 millennia. 

The convenience of being able to share thoughts through written methods came with an immediate compromise to security.

\subsection{Symmetric Key Cryptography}

\subsection{Monoalphabetic Ciphers}

\subsection{Block Ciphers}

\subsection{Public Key Cryptography}

A more practical method of encrypting data is to 

\subsection{Cryptanalysis}

\section{Quantum Theory}

\begin{equation}
	\ket{0} + \ket{1}
\end{equation}

\subsection{Quantum Information}

\subsubsection{Time-bin Encoding}

\subsection{Quantum Photonics}

\section{Quantum Key Distribution}

BB84 \cite{BB84}. Shor \cite{shor1994}.

\subsection{Protocols}

\subsection{Hacking}

While \ac{QKD} is often claimed to be ``perfectly secure'' or ``unhackable'', this is only true under certain assumptions. 

There have been numerous demonstrations that \ac{QKD} system are insecure under real world condition.

\subsubsection*{Device Independence}

To remove potential attacks on \ac{QKD} systems, protocols have been developed to relieve some of the assumptions required for security in a key exchange. 

Creating accurate models for equipment is particularly challenging. By removing assumptions about their operation, full characterisation is no longer required.

\section{Integrated Photonics Devices}

\subsection{Requirements}

\subsection{Indium Phosphide}

\subsubsection*{Lasers}

\subsubsection*{Electro-optic Modulation}

\subsubsection*{Thermo-optic Modulations}

\subsubsection*{Mach-Zehnder Interferometers}

\subsection{Silicon}

\begin{equation}
	5+5=10
\end{equation}
	

\subsection{Integrated Detectors}

\section{Summary}


%=========================================================