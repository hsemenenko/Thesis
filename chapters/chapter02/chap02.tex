%
% File: chap02.tex
% Author: Henry Semenenko
% Description: Background chapter
%
% Set the graphics path to find figures
\graphicspath{{./chapters/chapter02/fig02/}}

\let\textcircled=\pgftextcircled
\chapter{Background}
\label{chap:background}

As with any work of research, it is built upon centuries of previous work. Here we highlight some important advances which will be required for the rest of the thesis.

%=======
\section{Cryptography}
\label{sec1:crypto}

Cryptography is the process of taking a message ({\color{bristol-red} plaintext}) and translating it into a {\color{bristol-red} ciphertext}. Importantly, it also encompasses the reverse process of taking a ciphertext and returning the original message. The idea of obscuring messages from third-party onlookers dates back (as far as we can tell) to ancient Egypt and has played an important role in the world for the last 2 millennia. 

The convenience of being able to share thoughts through written methods came with an immediate compromise to security.

\subsection{Symmetric Key Cryptography}

Two parties, typically Alice and Bob, share a secret key that can be used to encrypt messages. The method of encryption is, generally, publicly known. However, it will be hard to decrypt messages without the secret key.

These ciphers have been widely used through history. However, with modern technology, they are not deemed to be secure. We will see in section \ref{sec1:cryptanalysis}

\subsection{Monoalphabetic Ciphers}

The simplest ciphers are those that permute the alphabet and are known as {\color{bristol-red} monoalphabetic ciphers}. The symmetric key in these ciphers is the permutation used. There are two well known cases of monoalphabetic ciphers which we will present here. 

\subsubsection*{Caesar Cipher}

The Caesar cipher one of the oldest that was well documented being used. Seutonius wrote that Julius Caesar used the cipher to send important information to the military. 

By assigning values to each letter of the alphabet (`a' = 0, `b' = 1, etc.), the encryption is done by `adding' the secret key, $k$ to each letter modulo 26. For example, when $k = 5$ we get the encoding

\begin{center}
\begin{tabular}{l l}%{@{}l@{\ }l}
	Plain:  &\quad{\tt a b c \ldots x y z} \\ 
	Cipher: &\quad{\tt F G H \ldots C D E} \\
\end{tabular}
\end{center}

Deciphering the ciphertext is as simple as `subtracting' $k$ from each letter and reading out the plaintext message.

As there are only 25 possibly keys, a brute force attack on this kind of cipher is fairly trivial. It is easy to loop over each choice, especially with modern computers, until the plaintext is revealed.

\subsubsection*{Affine Cipher}

A slightly more complex cipher is the affine cipher. This is still a mapping from the alphabet to itself (i.e. monoalphabetic) but the key now contains two numbers. Firstly

\subsection{Polyalphabetic Ciphers}



\subsubsection*{Vigen\`{e}re Cipher}

This cipher was developed by Giovan Battista Bellaso in 1553 \cite{} but is attributed to Blaise de Vigen\`{e}re \cite{}. The key is a word or phrase which can be `added' to the plaintext to get the cipher text.

\subsubsection*{One-Time Pad}

A polyalphabetic cipher where the key is as long as the message. If the key is truly random, each message is equally likely given a cipher text. Therefore, it is unbreakable provided that the key is completely secure.

The problem then becomes how to distribute the keys.

\subsection{Block Ciphers}

\Ac{aes} is, by far, the most used encryption method used today. It is generally thought that \ac{aes} is secure against attacks by a quantum computer. However, much like security using classical computers, there is no proof.

\subsection{Public Key Cryptography}

A more practical method of encrypting data is to 

\subsection{Cryptanalysis}
\label{sec1:cryptanalysis}

This is the research into how encryption can be broken.

\section{Quantum Theory}

\subsection{Quantum Mechanics}

Quantum mechanics creates a framework which rests on four postulates

\begin{itemize}
	\item[] \textbf{Postulate 1:} The state defined the quantum system
	\item[] \textbf{Postulate 2:} The state evolves in time
	\item[] \textbf{Postulate 3:} Measurements of quantum systems
	\item[] \textbf{Postulate 4:} System can be combined into composite systems
\end{itemize}

\subsubsection{No-Cloning Theorem}

A fundamental theorem in quantum mechanics is the {\color{bristol-red} no-cloning theorem}, which incidentally also underpins the security of \ac{QKD}. The general principle of the theorem states that there is no physical process that copies, or clones, an arbitrary quantum state.

\begin{theo}
	 No-Cloning theorem states that there is no quantum channel, $\mathcal{E}$, from $M(\mathbb{C}^d)$ to $M(\mathbb{C}^d) \otimes M(\mathbb{C}^d)$ such that
	 \begin{equation}
	 	\mathcal{E}(\ket{\psi}\bra{\psi}) = \ket{\psi}\bra{\psi} \otimes \ket{\psi}\bra{\psi}
	 \end{equation}
	 for all states $\ket{\psi} \in \mathbb{C}^d$.
\end{theo}


\subsection{Quantum Photonics}


Coherent states are very useful and are given by

\begin{equation}
	\ket{\alpha} = \mathrm{e}^{-\frac{|\alpha|^2}{2}} \sum_{n=0}^\infty \frac{\alpha^n}{\sqrt{n}} \ket{n} 
\end{equation}


\subsection{Quantum Information}



\subsubsection{Time-bin Encoding}



\subsection{Quantum Computing}



\section{Quantum Key Distribution}

Shor \cite{shor1994}.

There are many different ways that information can be encoded on photons. This had led to quite a few different protocols. Here we will discuss 

The resources available to the two parties, Alice and Bob, are a quantum communication channel that can be used to send quantum states and an authenticated, public classical channel. 

\subsection{Discrete Variable QKD}

The first ways that it was conceived that information could be encoded on photons was using orthogonal states to represent 0 or 1. For example, one could use the vertical and horizontal polarisations of a photon where diagonal and anti-diagonal could be used to encode superpositions. 

A more practical method of encoding information when intending to use fibre optics to transmit the photons is a time-bin encoding. Now the timing and relative phase information between two pulses gives us a complete encoding of all quantum states. Provided that the two time-bins are closely spaced they won't be as affected by fluctuations in the fibre.

\subsubsection*{BB84}

The first \ac{QKD} protocol that was developed was the BB84 protocol, named after its inventors: Bennett and Brassard \cite{BB84}. The protocol stemmed from an earlier idea of quantum money \cite{quantum_money}.

This protocol is typically referred to as {\color{bristol-red}prepared-and-measure}. Alice and Bob play complementary roles. Alice {color{bristol-red}prepares} a quantum state which she sends to Bob who {\color{bristol-red}measures} the state in a predefined way. 

There are many other prepares and measure protocols which have been developed over the years such as \ac{cow} \cite{COW-QKD} and \ac{dps} \cite{DPS-QKD}.

\subsubsection*{E91}

Independently of the development of BB84, another protocol was developed by Ekert called E91 \cite{E91}. The protocol took advantage of the correlations of entangled photons and realised that by distributing two entangled photons between Alice and Bob meant that they could share a secret key. While it could be argues that this could be considered a prepare-and-measure scheme, we distinguish it so as to describe the use of entangled states.

\subsection{Continuous Variable QKD}

More recently there has been an interest in \ac{cvqkd} due to the compatibility with current telecomms equipments. The states are encoded in quadrature space and measured using homodyne receivers, much like how classical information is transmitter.

There are still some questions about the security of \ac{cvqkd} and the practicality of the post-processing requirements. Unlike in \ac{dvqkd} where only successful measurements need to be analyse, each potential event needs to be analysed. This leads to big overheads during the error correction stages of the protocol.

\subsection{Hacking}

While \ac{QKD} is often claimed to be ``perfectly secure'' or ``unhackable'', this is only true under certain assumptions. 

There have been numerous demonstrations that \ac{QKD} system are insecure under real world condition.

\subsection*{Device-Independent QKD}

To remove potential attacks on \ac{QKD} systems, protocols have been developed to relieve some of the assumptions required for security in a key exchange. 

Creating accurate models for equipment is particularly challenging. By removing assumptions about their operation, full characterisation is no longer required.

\section{Integrated Photonic Circuits}

On the route to a scalable technology, we will need to find a platform which allows for many devices to be made without an exponentially increase in resources. 

\subsection{Requirements}

In the search for a platform for \ac{QKD} protocols, we will need to define certain criteria which will need to be met.

\begin{itemize}
	\item High speed modulation
	\item Creation of quantum light
	\item Detectors
\end{itemize}



\subsection{Indium Phosphide}

One of the more recent advances in integrated photonics is from III-V materials which offer certain benefits for \acp{pic}, especially for communication purposes. In this section, we will introduce the \acl{InP} platform and describe some of the components that we will find useful for the remainder of the thesis.

\subsubsection*{Lasers}

As \ac{InP} has a direct bandgap it is possible to monolithically integrate lasers onto the devices. 

Multi quantum well structures in \ac{InP} mean that we can create lasers on chip.

\subsubsection*{Electro-optic Modulation}

A \ac{qcse} gives rise to high speed electro-optic modulation of up to \SI{40}{GHz}.

\subsubsection*{Thermo-optic Modulation}

Very stable phase modulation can be achieved with thermo-optic effects. By heating the material we can change the refractive index and induce a phase.

\subsubsection*{Mach-Zehnder Interferometers}

Combining \acp{eopm}, acp{topm} and \acp{mmi} we can create on-chip \acp{mzi} which can be used for very fast routing. 

\subsection*{Electronics Integration}

\subsection{Silicon}

Silicon is ubiquitous in the world today and powers everything from watches to supercomputers. The advances that have been made in the trillion dollar semiconductor industry have made silicon a hugely successful platform for quantum photonics. With a band gap of $\sim 1100nm$, we can also leverage the telecommunications industry to create low-loss \acl{pic}.

The high index contrast in a \ac{soi} device means that very compact devices with small bend radii are possible.

Coupling can be done with grating couplers meaning that coupling light on and off chip isn't limited to the edges.

\subsubsection*{CMOS Compatibility}

One of the major 

\subsubsection*{Integrated Detectors}

An important part of any communication protocol is the detection of states. Working with very weak pulses introduces 

\subsection{Hybrid platforms}

More recently, there have been effects to hybridise \acp{pic} to create devices which exploit the benefits of multiple platforms. 

For example, hybrids of an \ac{InP} laser and Si ring filters to make a laser device with a linewidth of \SI{290}{Hz} \cite{Fan2017}.

There are also efforts to develop cross-platform electronic integration between III-V materials and Silicon \cite{}. 

\section{Summary}


%=========================================================