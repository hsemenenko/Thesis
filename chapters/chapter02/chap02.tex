%
% File: chap02.tex
% Author: Henry Semenenko
% Description: Background chapter
%
% Set the graphics path to find figures
\graphicspath{{./chapters/chapter02/fig02/}}

\let\textcircled=\pgftextcircled
\chapter{Background}
\label{chap:background}

%=======
\section{Cast of Characters}

Before we begin, we will need to introduce the cast of characters that will each have parts to play throughout this thesis. While the exact reasons behind each persons' role is unclear, we can be sure that they will follow their roles without question. 

\brisred{Alice}, for reasons that we will not discuss here, would like to send a message to Bob. The particular message is unknown publicly and she would like it to remain as such. While she has never met Bob, she wants to make sure that the message is actually being sent to Bob. Often she will require to know that the message has reached Bob but will rarely seek any further response.

\brisred{Bob} wants to receive and also read the message from Alice, whom he has never met, whilst also being sure that the message is, in fact, from Alice. He also wants to ensure that the message he received is a faithful representation of the one Alice sent and has not been modified along the way. Bob is not very talkative and doesn't often respond to messages from Alice.

\brisred{Charlie} is a mutual acquaintance of Alice and Bob (despite having never met either of them) who volunteers to act as a mediator for their communication. His specific role will depend on exactly how Alice and Bob are communicating and he is often not required beyond moral support. While he is not actively trying to sabotage Alice and Bob's communication, neither is he actively protecting them.

\brisred{Eve} is a very nosey individual who listens to all public communication between Alice and Bob. She is intent on reading the message sent by Alice but while very persistent, she is not very inventive so will not put very much effort into deciphering the message. Eve is often mistaken as Mallory, as they have similar intents.

\brisred{Mallory} would also like to know what Alice has sent to Bob, but will go to more extreme measures to discover their secrets. She is particularly fortunate in having unlimited resources available (confined only by the laws of physics) and will use all possible methods within her power to expose the hidden message. Her motivations are unknown.

The reader is encouraged to picture their own cast of characters in their mind's eye to bring this thesis to life. Any likeness of characters to any persons, real or imaginary, is purely coincidental.

\section{Cryptography}
\label{sec1:crypto}

Elements of cryptography date back almost 4000 years to Ancient Egyptian times where unusual hieroglyphs were found to have been added to potential shroud mystery over the meaning\cite{singh1999code}. While generally not considered a serious attempt at hiding any secret message (and therefore considered steganography) such concepts of concealment are fundamental in modern cryptography.

It has been argued that cryptography was known to the Ancient Greeks. To incite a revolt against the Persians around 440 BC, Histi{\ae}us was said to shave the head of his most trusted slaves so that a tattooed message would be hidden by his regrown hair \cite{Herodotus}. References were also made centuries later to a long stick called a scytale around the 3rd century BC \cite{Deipnosophistae}. Parchment was wound around the scytale and, in effect, created a transposition of the message which was then used in the Spartan military. While these were revolutionary for their time, they are not considered cryptographic as the message is merely hidden rather than encrypted.

%The history of cryptography is long with elements dating back as long ago as Egyptian and Ancient Greek times \cite{Deipnosophistae, singh1999code}.

%Cryptography is the process of taking a message (\brisred{plaintext}) and translating it into a something that is indecipherable (\brisred{ciphertext}). Importantly, it also encompasses the reverse process of taking a ciphertext and returning the original message. 

%The idea of obscuring messages from third-party onlookers dates back (as far as we can tell) to ancient Egypt and has played an important role in the world for the last 2 millennia. 

%The convenience of being able to share thoughts through written methods came with an immediate compromise to security.

To better define whether a system can be classified a cryptography, and truly distinguishable from steganography, we must fast-forward to the 19th century. Systems designed at hiding messages, but not necessarily encrypting them, were commonplace. Auguste Kerckhoffs presented a set of six principles that any cryptographic system should satisfy \cite{KerckhoffsPrincple}. While technological advances have superseded some of the list, Kerckhoffs' principle remains. It is succinctly summarised as

\begin{quote}
\centering
\textit{``The enemy knows the system''} - Claude Shannon \cite{shannon1949communication}
\end{quote}

Any cryptographic system should remain secure without requiring secrecy of the underlying protocol. Alice and Bob should assume that Eve and Mallory have a copy of their system when considering its security. Only the \textit{key} will remain secret to Alice and Bob. Of course, we will see that the definition of ``secure'' will change depending on the context to mean \textit{practically} or \textit{mathematically} indecipherable. 

This section will introduce some of the important advances in cryptographic history. We will discuss how the modest origins of classical cryptography will become a crucial part of society and how cryptography may have even altered the course of history.

\subsection{Symmetric-Key Cryptography}

The first algorithms used for cryptography were so called \brisred{symmetric-key} algorithms. Alice and Bob initially decide on a shared key. Historically, these keys would have been shared in person, although technological advances have allowed these keys to be shared from a distance. The shared key can be used by Alice and Bob to both encrypt and decrypt the message, but the exact transformation for encryption or decryption may differ.

%Symmetric key algorithms are, broadly speaking, separated into three categories.

%through some other means which they can then used to encrypt and decrypt messages. 

%Two parties, typically Alice and Bob, share a secret key that can be used to encrypt messages. The method of encryption is, generally, publicly known. However, it will be hard to decrypt messages without the secret key.

%These ciphers have been widely used through history. However, with modern technology, they are not deemed to be secure. We will see in section \ref{sec1:cryptanalysis}

\subsubsection{Monoalphabetic Ciphers}

The most basic ciphers are those that are simple permutations of the alphabet. These are known as \brisred{monoalphabetic ciphers}. The symmetric key in these ciphers is the permutation used, that can be as simple as a single number. Here, we will introduce a couple of simple examples of monoalphabetic ciphers. 

\subsubsection*{Caesar Cipher}

One well-known example of cryptography in history was used by the Romans and is now referred to as the \brisred{Caesar cipher}. According to Seutonius, Julius Caesar used this simple cipher to communicate important messages to his generals \cite{Suetonius}. 

Each letter of the alphabet is assigned a numeric value i.e. `a' = 0, `b' = 1, etc. and Alice and Bob agree a key $k\in\mathbb{Z}_{26}$ which they keep secret. When Alice wants to send a message, she converts her message into a number and systematically `adds' the secret key, $k$, to each of the letters, modulo 26. Caesar was reported to use $k=3$, which would give the encoding 

%The algorithm assigns numeric values to each letter of the alphabet () and a key  is agreed between Alice and Bob.  the encryption is done by `adding' the secret key, $k$ to each letter modulo 26. For example, when $k = 5$ we get the encoding

%The Caesar cipher one of the oldest that was well documented being used. Seutonius wrote that Julius Caesar used the cipher to send important information to the military. 

%By assigning values to each letter of the alphabet (`a' = 0, `b' = 1, etc.), the encryption is done by `adding' the secret key, $k$ to each letter modulo 26. For example, when $k = 5$ we get the encoding

\begin{center}
\begin{tabular}{l l}%{@{}l@{\ }l}
	Plain:  &\quad\texttt{a b c d e f g h i j k l m n o p q r s t u v w x y z} \\ 
	Cipher: &\quad\texttt{D E F G H I J K L M N O P Q R S T U V W X Y Z A B C} \\
\end{tabular}
\end{center}
where the Alice finds the letter of her plain text in the top row and substitutes it with the cipher text in the bottom row. Eve and Mallory would not be able to make out the works without the secret key. 

Once Bob receives the message from Alice, he can decipher the message by `subtracting' $k$ from the message i.e. moving from the bottom row of the table above to the top row. This allows him to read the message.

In an era where most people would have been illiterate, the cipher was likely to be secure. However, with only 25 possible combinations to try, guess each one is rather trivial. Once a single word has been uncovered the entire message would be forfeit. 

\subsubsection*{Affine Cipher}

A slightly more complex monoalphabetic permutation comes from the affine cipher, although it seems to be confined more for academic interest than real world use. The cipher is still a simple permutation of the alphabet but Alice and Bob now share two numbers. They agree on two integers $k,\lambda\in \mathbb{Z}_{26}$ such that $\lambda$ is coprime to $k$ i.e. they have no common factors greater than $1$. For each letter, $x$, we can encrypt using the mapping

\begin{equation}
	x \rightarrow \lambda \cdot x + k\quad\text{(mod 26)}
\end{equation}
where the decryption requires the inverse of $\lambda$. For example, using $k = 3$ and $\lambda = 7$, we get the mapping

\begin{center}
\begin{tabular}{l l}
	Plain:  &\quad\texttt{a b c d e f g h i j k l m n o p q r s t u v w x y z} \\ 
	Cipher: &\quad\texttt{D K R  Y F M T A H O V C J Q X E L S Z G N U B I P W} \\
\end{tabular}
\end{center}
which seems more `random' than the Caesar cipher. However, the cipher still only relies on two numbers to encrypt and decrypt. Eve or Mallory may just guess two of the mappings which allows them to decipher the entire message. Alternatively, as the cipher contains only $12 \cdot 26$ possible combinations, Mallory could easily perform a brute force attack.

\subsubsection*{Frequency Analysis}

\begin{figure}[t]
	\centering
	\def\svgwidth{0.6\textwidth}
   	\import{chapters/chapter02/fig02/}{letter_freq.pdf_tex}
   	\caption[Relative letter frequency in the English language]{A histogram of the relative letter frequency in the English language which can be used to break monoalphabetic ciphers.}
   	\label{fig:letter_freq}
\end{figure}

With ciphers being commonly used, interest turned to analysing cipher text. In the case of a monoalphabetic cipher, Mallory can consider the relative frequency of the letters. Figure \ref{fig:letter_freq} shows a histogram of the letters used in the English language. One finds that the letter `e`  is the most common, while letter such as `j', `q', `x' and `z` are rarely used.

If Mallory knows that the cipher being used is a monoalphabetic one (which we will assume from Kerckhoffs' principle) then they can calculate the letter frequency of the message. By matching the most used letters in the cipher text to the letter frequency, they may be able to decode the message. 

In more complex analysis, one can even consider the frequency of letters at the beginning or ends of words, and also the frequency of two or three letter fragments. Once parts of the cipher text have been decrypted, it will usually be sufficient to recover the rest.

\subsubsection{Polyalphabetic Ciphers}

Since the invention of the Caesar cipher in Roman times, there were not many advances in encryption until the 15th century. Up until this point, encryption methods generally used simple substitutions or permutations. The next evolution was to develop polyalphabetic ciphers where each letter was encrypted (and decrypted) using a different permutation of the alphabet. Essentially, they were a collection of monoalphabetic ciphers used in a particular order.

\subsubsection*{Vigen\`{e}re Cipher}

This cipher was originally developed by Giovan Battista Bellaso in 1553 \cite{belaso1553cifra} but is attributed to Blaise de Vigen\`{e}re \cite{vigenere1586traite}. The cipher was believed for a long time to be unbreakable as it was not initially clear how frequency analysis techniques could be adjusted for the multiple permutations. 

The cipher introduces a word or phrase shared by Alice and Bob which is used as the key. The keyword is repeated to match the length of the message and each letter represents its own Caesar cipher. For example, Alice and Bob may choose to share the keyword `\texttt{photon}' in which case they would encrypt as

\begin{center}
\begin{tabular}{l l}
	Plain:  &\texttt{wewillbuildaquantumcomputerbytheendofthisdecade} \\ 
	Keyword: &\texttt{photonphotonphotonphotonphotonphotonphotonphoto}\\
	Cipher: &\texttt{LLKBZYQBWERNFBOGHHBJCFDHILFUMGWLSGRBUAVBGQTJOWS} \\
\end{tabular}
\end{center}
where decryption is the reverse and each letter is `subtracted' from the message.

If the message is long enough, then the problem can be reduced to analysing $k$ individual Caesar ciphers, where $k$ is the length of the keyword. However, this requires Eve or Mallory to know the length of the keyword. Therefore, they need a method of calculating $k$.

A method introduced by William F. Friedman considered the comparing the cipher text with a shifted version of itself. To understand how this might work, consider compare two passages of text with one shifted relative to the other. How often would we expect the letters between the two to match? For example, 
\begin{center}
\begin{tabular}{l}
	\brisred{\texttt{{ }{ }{ }{ }{ }{ }{ }{ }{ }l{ }{ }{ }{ }{ }{ }{ }u{ }{ }{ }{ }{ }{ }{ }{ }{ }{ }{ }{ }{ }{ }{ }{ }{ }{ }{ }{ }{ }{ }{ }{ }{ }{ }{ }de{ }{ }{ }{ }{ }}} \\
	\texttt{wewillbuildaquantumcomputerbytheendofthisdecadeusingtodaystools} \\
	\texttt{{ }{ }{ }{ }wewillbuildaquantumcomputerbytheendofthisdecadeusingtodaystools}
\end{tabular}
\end{center}
where the second message has been shifted by four letters. The letters that match between the two texts are highlighted above in red. In this example, we found 4 out of 63 letters match (\SI{6.3}{\percent}). How often should we expect this to happen?

Considering the frequencies of each letters, as shown in figure \ref{fig:letter_freq}, we can calculate the probability as

\begin{equation}
	P(\mathrm{id}) = p_\text{`a'}^2 + p_\text{`b'}^2 + \dots +p_\text{`z'}^2 \approx 0.0655
\end{equation}
where the `id' refers to the fact that the messages were both written with the same alphabet.

Suppose we were to instead compare two messages which had a different encoding, for example the plain text message with a monoalphabetic cipher text. Let us denote a monoalphabetic encoding of the alphabet as $\pi(i)$ which is acts on the $i$th letter. We then like to calculate

\begin{equation}
	P(\pi) = \sum_\text{`a'}^\text{`z'} p_i \cdot p_{\pi(i)}
\end{equation}

In general, the probability of coincidence will \emph{always} be less when comparing two different encodings. For any non-trivial permutation $\pi$,

\begin{align}
	P(\mathrm{id}) - P(\pi) &= \sum p_i^2 - \sum p_i \cdot p_{\pi(i)}\\
	&= \frac{1}{2}\left(\sum p_i^2 + \sum p_{\pi(i)}^2\right) - \sum p_i \cdot p_{\pi(i)}\\
	&= \frac{1}{2} \sum (p_i - p_{\pi(i)})^2 > 0
\end{align}

On average, we would expect a match probability of around \nicefrac{1}{26} \SI{\approx 3.85}{\percent}. To apply this to the Vigen\`{e}re cipher, one simply needs to compare the cipher text with a shifted version of itself. The probability of coincidence will be maximised when the text is shifted by a multiple of the keyword length. Once the length of the keyword has been discovered, the problem reduces to analysing multiple Caesar ciphers. Frequency analysis can be applied to break each individually and recover the plain text from the cipher text. 

%\subsubsection*{The Enigma Machine}
%
%Perhaps the most famous polyalphabetic cipher is the enigma machine that was widely used by the Axis powers during WWII.

\subsubsection*{One-Time Pad}

%A polyalphabetic cipher where the key is as long as the message. If the key is truly random, each message is equally likely given a cipher text. Therefore, it is unbreakable provided that the key is completely secure.

The \ac{otp} is a polyalphabetic cipher where the key is made as long as the message and every message is encrypted with a unique and random key. It was initial described in 1882 by Frank Miller \cite{miller1882telegraphic} but it wasn't widely recognised until 1919 when a Gilbert Vernam patented a method to combine the message and key on punched-tape \cite{vernam1919}. In the patent, the tape that stored the key was reused after a full cycle. It was later realised that if the key was completely random and not reused then ``the messages are rendered entirely secret'' \cite{vernam1926cipher}. 

Typically, the \ac{otp} is concerned with bit level operations where encrypting and decrypting are both performed with addition modulo 2. Consider that the message, $m$, can be represented as a bit string $m_1m_2\ldots m_k$, and similarly for the key and cipher text, then encrypting the message is simply

\begin{equation}
	c_i = m_i + k_i\quad\text{(mod 2)} 
\end{equation}
where $k_i$ are the bits that represent the key which is of the same length as the message and $c_i$ is the cipher text bit. Equivalently, decryption is

\begin{equation}
	m_i = c_i + k_i\quad\text{(mod 2)}
\end{equation}

Since each letter is encrypted with a random monoalphabetic cipher, there is no way to perform frequency analysis. Likewise, the general cryptanalysis used for the Vigen\`{e}re cipher will provide no information as there is no repetition in the key.

Providing that the key is chosen randomly and not reused, the \ac{otp} can be proven to be \brisred{information-theoretic} secure \cite{shannon1949communication}. That is to say that even Mallory, with her unlimited power, would not be able to comprise the message without the secret key. Intuitively, as the key is completely random, each decryption is equally likely so there is no way to reveal the true message.

%We can consider the cipher to be polyalphabetic where they length of the key is as long as the message itself. 

While the protocol itself provides a `perfect' method of encryption, this doesn't exclude physical vulnerabilities in the system. Information may be leaked through side-channels in specific system implementations. Alice and Bob also have to overcome the problem of securely distributing the keys. 

\subsubsection{Block Ciphers}

Despite the inherent security provided by the \ac{otp}, it is rarely used in practice. Challenges with key exchange render it impractical as Alice and Bob must have keys that are as long as the message itself. Methods to exchange these keys remain slow meaning that, typically, the theoretical security of a system is relaxed in favour of practicality. 

The most used encryption methods in modern communication are \brisred{block ciphers} that were introduced by Horst Feistel \cite{feistel1970cryptographic}. Instead of considering the message as individual letters (or binary bits), a block cipher will consider the message in blocks of a certain length. As the data is encoded in vectors encryption and decryption can be performed by matrix multiplication which is computationally efficient. The exact length of the block will depend on the cipher being used. The block is split into two equal parts $v$ and $w$ which are encoded in the first round using 

\begin{equation}
	v' = w \quad \text{and} \quad w' = v + F(w, k)
\end{equation}
where $F$ is a vector-valued function and $k$ is the key. The key is then modified to generate a subkey in which the process above is repeated with $v$, $w$ and $k$ replaced by $v'$, $w'$ and $k'$. The function $F$ and the method of generating the subkeys is depending on the block cipher being used. This process is easily reverse by

\begin{equation}
	w = v' \quad \text{and} \quad v = w' + F(v', k)
\end{equation}

The first widely used block cipher was \ac{des} introduced in 1977 \cite{DES1977} and was based mostly on Feistel's idea. The algorithm uses a key length of 56 bits and separates the data into 64 bit blocks which are divided into two vectors, $v$ and $w$, each of length 32 bits. The vector-valued function for \ac{des} is described in algorithm \ref{alg:des} which is repeated for 16 rounds.

\begin{algorithm}[Data Encryption Standard]
\label{alg:des}
\begin{enumerate}
	\item $w$ is expanded from 32 bits to 48 bits by repeating certain bits.
	\item The key, $k$, is divided into two separate 28-bit keys. 
	\item In each successive round, the two keys are shifted by one or two bits depending on the round.
	\item 24 bits from each key are selected to create a 48-bit subkey which is added to the expanded vector, $w$. 
	\item Each block is separated into eight six-bit sections which are substituted with four-bit outputs from a lookup table. This transformation is non-linear.
	\item Finally, the 32 outputs are permuted by a fixed mapping.
\end{enumerate}
\end{algorithm}

Immediately following the release of \ac{des}, there was scrutiny concerning the involvement of the \ac{nsa} and the short length of the key. It was suggested that a machine could be built at the cost of only \$20 million which could break \ac{des} in a day \cite{Diffie1977}. Despite these concerns, \ac{des} remained a standard until 2001. At which point, a number of brute-force attacks had been demonstrated, notably by the Electronic Frontier Foundation. 

\Ac{des} was replaced by \ac{aes} in 2001 \cite{aes2001} which was based on the same block cipher structure. The message is now split into 128 bit messages and secret keys of length 128, 192, or 256 bits available to prevent brute-force attacks. Block ciphers are not information-theoretic secure, as each bit of key is expanded and used to encrypted more than one bit of the message. It is believed that \ac{aes} remains \brisred{computationally secure}. That is to say that it is not feasible to build a computer that can break an \ac{aes} cipher in a reasonable amount of time. 

%The new standard increased the possible key size to a minimum of 128 bits with keys up to 256 bits also possible. 

%The first block encryption was the \ac{des} cipher which was developed in the early 70's \cite{}. Upon its introduction, there was criticism that the NSA deliberately weakened the security of the cipher \cite{}. It remained a standard in cryptography until the late 1990's when it was demonstrated that it was insecure against modern computers \cite{}.

%The \ac{aes} cipher is by far the most used encryption method used today which superseded \ac{des} in the early 2000's \cite{}. It is generally thought that \ac{aes} is secure against attacks by a quantum computer. However, much like security using classical computers, there is no proof.

%Typically, 256 bit keys are exchanged using public key cryptography which are then used in a block cipher. 

\subsection{Public-Key Cryptography}

We have alluded to the challenges of distributing keys between Alice and Bob for use in a symmetric key algorithm. The obvious solution is for Alice and Bob to meet in person prior to their secret communication. However, it might not always be possible for them to meet and it certainly is not practical. 

In order to exchange secret messages over insecure communication channels, \brisred{public-key} cryptography algorithms were developed. Alice and Bob act asymmetrically basing security from number theory problems are assumed hard. There has yet to be an algorithm that can claim with certainty to be even computationally secure.

Such problems employ \brisred{trapdoors}: functions that are easy to compute one way, but challenging to invert without secret information. Alice will use the function to encrypt, but only Bob has the secret information to decrypt the message. As the name suggests, Bob will publicly announce certain information which Alice (or anyone else for that matter) can use to encrypt a message. However, he keeps a private key which will allow him to decrypt the messages.

This section will introduce two public-key algorithms that are widely used. While the algorithms can be used as a method to send messages, the computational intensity of encryption and decryption means that they are typically used to distribute keys. These keys are then used in a symmetric key algorithm such as \ac{aes}.

%A more practical method of encrypting data is to employ an asymmetric scheme which allows many people to send encrypted messages to a single person. For example, customers of a bank may want an easy way to send encrypted messages about their finances.

%These ciphers include what is know as a . These functions are such that they are easy to perform one way, but are difficult to inverse without a private key. 

%Generally, public key ciphers are computationally intensive. Therefore, they are generally used to exchange keys which can then be used in a block cipher, such as \ac{aes}.

\subsubsection*{Diffie-Hellman Algorithm}

The first public-key cryptography algorithm was proposed by Whitfield Diffie and Martin Hellman in 1976 \cite{diffie1976new}. It was later revealed that researchers at \ac{gchq} had previously demonstrated that such a protocol was possible \cite{Ellis1970, williamson1974}.

The security of the protocol is based on the hardness of the discrete logarithm problem. The problem is to find $k = \log_a(b)$ given a prime number, $p$, and integers $1<a<p$ and $b$, such that

\begin{equation}
	a^k = b\quad\text{(mod } p\text{)}
\end{equation}

There is no publicly known classical algorithm that can efficiently compute the discrete logarithm. However, an algorithm was presented by Peter Shor in 1994 that uses a quantum algorithm to solve the discrete logarithm in polynomial time \cite{shor1994}.

The Diffie-Hellman algorithm is not used to send a message between Alice and Bob, but instead is used to establish a key between them. This key can then be used in a symmetric key algorithm. The protocol is presented in algorithm \ref{alg:dh}.

\begin{algorithm}[Diffie-Hellman Key Exchange]
\label{alg:dh}
\begin{enumerate}
	\item Alice and Bob agree on a large prime number, $p$, and a primitive root, $a$. These numbers can be publicly available.
	\item They both also secretly choose integers $d_A$ and $d_B$.
	\item Alice sends her public key, $q_A = a^{d_A}$, to Bob, and Bob send his public key, $q_B = a^{d_B}$ to Alice.
	\item On receiving a public key, Alice and Bob compute $k = q_B^{d_A} = a^{d_A\times d_B}$ and $k = q_A^{d_B} = a^{d_A\times d_B}$, respectively.
	\item Alice and Bob now share the secret key $k$.
\end{enumerate}
\end{algorithm}

The information that is available to Eve or Mallory are the numbers $p, a, q_{A}$ and $q_{B}$. However, without being able to solve the discrete logarithm, they are unable to compute $d_A$ or $d_B$. Hence, they cannot efficiently compute the secret key $k$.  

Now that Alice and Bob share a secret key $k$, they can choose a symmetric key algorithm to send encrypted messages.

%Typically, the Diffie-Hellman algorithm is used as a key exchange method to 

%The Diffie-Hellman cipher isn't used to share messages, but instead distributes keys between Alice and Bob in a secure way that can then be used in a symmetric key cipher. It is attributed to Whitfield Diffie and Martin Hellman\footnote{Malcolm J. Williamson had already discovered the Diffie-Hellman key exchange whilst working at GCHQ but it remained classified.} after they discovered it in 1976 \cite{}.

\subsubsection*{RSA Algorithm}

Two years later, the \ac{rsa} algorithm would be released publicly, named after its inventors \cite{rivest1978method}. As with the Diffie-Hellman algorithm, it was later discovered that the \ac{rsa} method had been known by Clifford Cocks during his time at \ac{gchq} \cite{Cocks1973}. 

Unlike the Diffie-Hellman algorithm, the \ac{rsa} algorithm allows Alice to send a predetermined message to Bob. Its security relies on the hardness of factoring semi-primes: a number that is the product of two prime numbers. The \ac{rsa} algorithm is given below.

%Suppose Alice would like to send a message to Bob. The algorithm goes something like:

\begin{algorithm}[Rivest-Shamir-Adleman Key Exchange]
\begin{enumerate}
	\item Bob needs to first choose a secret key. He selects two large prime numbers, $p$ and $q$, and calculates $n = p \times q$.
	\item He then chooses some $d < (p-1)(q-1)$ which is coprime to $(p-1)(q-1)$.
	\item Finally, Bob calculates the inverse, $e$, of $d$ modulo $(p-1)(q-1)$ and publishes the public key $N$ and $e$ whilst keeping $p,q$ and $d$ private.
	\item Alice breaks her message into integers $m < N$ and encrypts as $c = m^d\text{ }(\text{mod }N)$. The ciphertext $c$ can now be sent to Bob.
	\item To decrypt Alice's message, Bob computes $c^d$ and the communication is complete. 
\end{enumerate}   
\end{algorithm}

To prove that the decoded message is equivalent to the original message, we note that

\begin{equation}
	b \equiv c^d \equiv m ^{d\cdot e}\quad(\text{mod } N)
\end{equation} 
and as $e$ is the inverse of $d$ modulo $(p - 1)(q - 1)$, we have

\begin{equation}
	d \cdot e = t \cdot (p - 1)(q - 1) + 1
\end{equation}
for $t\in\mathbb{Z}$. If $m$ and $p$ are coprime, then Fermat's little theorem states that 

\begin{equation}
	m^{p-1} \equiv 1\quad(\text{mod }p)
\end{equation}
from which, we find
\begin{equation}
	b \equiv m^{t \cdot (p - 1)(q - 1) + 1} \equiv m \cdot (m^{p-1})^{t\cdot(q-1)} \equiv m \quad(\text{mod } p)
\end{equation}

If $p$ is not coprime to $m$, then $p$ divides $m$ and also $b$. Therefore, $b\equiv m \text{ (mod } p$) which can be equivalently reasoned for $q$. Since $b$ and $m$ were taken modulo $N$, we must have that $b\equiv m \text{ (mod } N$).

Generating keys for the \ac{rsa} algorithm uses computationally efficient algorithm to check for primality and fast modular exponentiation. However, there is no method at the present to break the cipher without factoring $N$. 

Interestingly, while this outlines the underlying concepts, the algorithm as it stands is not secure. As the algorithm is deterministic, Mallory can produce the cipher text of any plain text message. They can compare these to the cipher text of the message being sent from Alice to Bob. To protect against this attack, Alice introduces some random bits into the message, $m$, through a method called padding \cite{goldwasser1982probabilistic}. 

%Although this algorithm outlines the general understanding of RSA encryption, it is not actually secure as it stands. The problem is that it is essentially a monoalphabetic cipher i.e. each $m$ modulo $N$ will always encode to the same ciphertext $c$. Implementations used introduce some randomness to make it secure.


\subsection{Authentication}

We have described how Alice can send a message to Bob without having ever met to share a secret key by using public-key cryptography. How can she be sure that the public key that she is using to encrypt her message was actually \emph{Bob's} public key? Without being sure, they can fall foul to a \ac{mitm} attack.

As Mallory is aware of the protocol that Alice and Bob are using (again referring to Kerckhoffs' principle) she can appear to Bob to Alice and, likewise, appear as Alice to Bob. By providing a public key to Alice and using Bob's public key to establish a secure communication, Mallory can take full control of the protocol. 

Alice will encrypt a message, that is intended for Bob, with Mallory's public key who is able to decrypt the message into plain text. Mallory can then re-encrypt the message using Bob's public key and send it to him. Without authentication neither Alice or Bob will know that their message was compromised. 

For Bob to verify that the message came from Alice, they will use an authentication protocol. This is typically achieved using a pre-shared key between Alice and Bob from which then can create an authentication tag to send alongside their message. Specific authentication protocols are beyond the scope of this thesis. However, we will note that their importance in modern communication is paramount and remains a prevalent discussion for the future of cryptography. 

%The reader is encouraged to learn more that a variant of \ac{aes} is often used for authentication and the digital signature algorithm is used for public-key cryptography. 

%In modern day cryptography, protocols rely on trusted third parties who verify the authenticity of users. However, this system is far from infallible. Nevertheless, it is hugely important.

%Until this point, we have discussed how Alice and Bob and create ciphertext that would deceive Eve. However, we also need to be aware that Mallory is more cunning than Eve. 

%As we have mentioned before, Alice and Bob have never met. However, they would like to be sure that they are talking to each other. If they don't ensure they are talking to each other, Mallory can perform a \ac{mitm} attack. Mallory is able to intercept messages sent from Alice to Bob. To Alice, she will imitate Bob, and to Bob, she will imitate Alice. By exchanging keys with both, she can fool Alice and Bob into sending their secret messages to her without ever knowing.

%As we will see later, authentication will play a crucial role in a \ac{QKD} system.

\subsection{Post-Quantum Cryptography}

In light of recent advances in quantum computing architectures \cite{arute2019quantum}, it is becoming more realistic that current public-key cryptography will become vulnerable to Shor's algorithm \cite{shor1994} in the near future. To counter these advances, \brisred{post-quantum} algorithms have been suggested as alternatives that will remain secure even against quantum computers. These algorithms use the same classical computers and networks, but are based on different problems. These algorithms are part of \brisred{quantum-safe} cryptography.

Since November 2017, \ac{nist} has been developing a new public standard for post-quantum algorithms. The process invited protocols to be submitted for assessment. In total 82 candidates were submitted, of which 69 were accepted as the first round candidates \cite{alagic2019status}. Finalists are expected to be selected in 2020.

Proving the security of an algorithm remains challenging, particularly against quantum computers. A recent algorithm, \textit{Soliloquy}, was a lattice-based algorithm created by researchers at \ac{gchq}. After a brief development between 2010 and 2013, a quantum algorithm was discovered that would efficiently break the cipher. The story was published as a cautionary tale for future algorithms. A paper from Peter Shor that claimed to efficiently break lattice-based cryptography was quickly withdrawn following a mistake\cite{eldar2016efficient}. Such developments demonstrate the rapidly changing landscape of cryptography.

% for security against quantum computers and processors. A status report has recently been 

%Post-quantum cryptography falls under the umbrella term of \brisred{quantum-safe} cryptography which encompasses key exchange protocols which are thought (or proven) to be secure against attacks that use a quantum computer. Instead of relying on problems such as discrete logarithm or factoring, which are known to be efficiently solved with Shor's algorithm, the protocols utilise problems that are thought to be hard even on a quantum processor. However, much like a proof of \ac{rsa}'s security in classical computing has eluded mathematicians, proof of security for post-quantum algorithms in both quantum and classical regimes has not yet been established.

%There are a number of examples in which quantum algorithms that were thought to be hard against quantum computers turned out to be broken using classical computers \checkthis{Can't remember the algorithm}. 

%Some interest proposals were made with RSA that would be easy for a classical computer but hard for a quantum computer. The protocol aimed to exploit the difficulty in loading data into a quantum machine which would bottleneck the operation. However, the scheme used many small RSA number making it easy to break on a classical computer \checkthis{Find this protocol and analysis}.

\subsection{Cryptanalysis}

\brisred{Cryptanalysis} aims to test the integrity of a cryptographic system and find weaknesses. This may be to analyse the mathematical problems which form the basis of the security of an algorithm or it may be testing the physical implementation. Successful cryptanalysis has been performed throughout history, mainly motivated through war, and may have even changed the course of history.

A notable example of cryptanalysis was during WWI in the form of the Zimmermann telegram \cite{zimmermann1917}. The message, intended to form an alliance between Germany and Mexico, was intercepted and deciphered by British intelligence. The United States, who were neutral at the time, joined the war three months later.

The most well known example of cryptanalysis was in Bletchley Park during the second world war. Hut 8 was specifically designated to analysis the enigma machine that was being used by the Axis powers \cite{hodges2012alan}. The team used \emph{cribs}, plain text that was suspected to be in a cipher text, with great success to break enigma codes. Such messages included ``nothing to report'' and physical intervention was even used to force the German Navy to send cribs \cite{hodges2012alan}. This work eventually led to Turning developing the first computer from the Polish Bombes.

%\label{sec1:cryptanalysis}
%
%For some of the system above, we have discuss their security. \brisred{Cryptanalysis} is the process of assessing the security of a system. 
%
%Cryptography has been crucial to society and at some points critical to historical development. Had the Zimmermann telegram not been decipher, the United States may have never joined WWI.
%
%This is the research into how encryption can be broken. Simple ciphers can easily be broken by comparing the ciphertext to the percentage of each letter used. As \texttt{`e'} is the most common letter in the English language, it follows that if a monoalphabetic cipher is used, then the most common letter in a passage is likely to be decoded to \texttt{`e'}.
%
%Parts of the enigma code were broken in WWII when the German's would send the same message ever morning \cite{hodges2012alan}. The British would even deliberately fly planes to force messages that they could decipher.

%\section{Quantum Theory}
%
%At the turn of the 20th, physics was built upon Newtonian mechanics, Maxwell's equations and statistical mechanics. 
%
%\subsection{Linear Algebra}
%
%To be able to introduce quantum theory, we first need to introduce some linear algebra which provides much of the language of quantum mechanics.

\section{Quantum Information}

From a thorough understanding of quantum mechanics came the new field of \brisred{quantum information}. Using fundamental particles, it was suggested that information could be encoded in a completely new way \cite{manin1980computable, feynman1982simulating}. This section will give a brief introduction to quantum mechanics theory and how it can be used to encode information. Some fundamental aspects will be discussed including a brief digression into quantum computing.

%Exploiting quantum mechanical processes gives rise to a new information theory called . Using quantum phenomena, we introduce quantum bits (qubits) as a generalisation of the bits that are typically used in classical computing.

%The success of quantum mechanics to describe the physical world has yet to be beaten. 

%The quantum computer concept had been suggested in the 1980s by Yuri Manin \cite{} and Richard Feynman \cite{}.

%Divincenzo's criteria \cite{divincenzo2000physical}.

%Despite the success of describe the physical world, quantum mechanics remains a mathematical framework from which physical laws can be derived. 
%
%Quantum mechanics can be seen as a generalised probability theory and was born out of mathematics. While quantum mechanics 

%The ideas create a framework to describe physical systems which rests on four postulates:

\subsection{Quantum Mechanics}

The postulates of quantum mechanics are not quite axiomatic, but rather guidelines for developing a more comprehensive framework for a particular system. The postulates are what connects the mathematical framework of quantum mechanics to physical systems.

\subsubsection*{Postulate 1: The State Space}

The first postulate concerns the state of the system. Any closed quantum system exists in a Hilbert space, $\mathcal{H}$, which is known as the state space of the system. The system is described by a vector of unit length within $\mathcal{H}$. Any vectors that differ by a global factor represent the same state.

Of course, the postulate doesn't tell us which Hilbert space to choose for our particular quantum system. Nor does it give the vector associated with our quantum state. To answer such questions, one must consult theories such as quantum electrodynamics. The simplest spaces and states are those found in quantum information which is restricted to two-level systems. The states are called \brisred{qubits} and take the general form

\begin{equation}
	\ket{\psi} = \alpha\ket{0} + \beta\ket{1}
\end{equation} 
where $\ket{0}$ and $\ket{1}$ form an orthonormal basis with the Hilbert space and are called the computational basis. $\alpha$ and $\beta$ are complex numbers which satisfy $|\alpha|^2 + |\beta|^2 = 1$ so that the state has unit length.

\subsubsection*{Postulate 2: Quantum State Evolution}

Evolution of a closed quantum system is described by the second postulate. It states that quantum states evolve through unitary transformations. The state at some time $t_2$ is related to the state at an earlier time $t_1$ by a unitary operator i.e.

\begin{equation}
	\ket{\psi_{t_2}} = \hat{U}\ket{\psi_{t_1}}
\end{equation}

When one is concerned with the time evolution of a quantum system, it is described by the Shr\"{o}dinger equation,

\begin{equation}
	i \hbar \frac{\mathrm{d}\ket{\psi}}{\mathrm{d}t} = \hat{H}\ket{\psi}
\end{equation}
where $\hbar$ is Plank's reduced constant and $\hat{H}$ is the Hamiltonian of the system. If the Hamiltonian of a system is known, then we can completely describe its dynamics. The Hamiltonian is a Hermitian operator meaning that the unitary transformation is given by

\begin{equation}
	\hat{U} = \mathrm{e}^{-\nicefrac{i(t_2 - t_1)\hat{H}}{\hbar}}
\end{equation}

\subsubsection*{Postulate 3: Measurement of the State}

Contrary to classical systems, where the state can be measured with arbitrary accuracy, quantum mechanics places inherent restrictions on the accuracy of measurements. Before describing the measurement process, we will introduce \brisred{observables}. Physical quantities in a quantum system are given by the observables which are a self-adjoint operators on $\mathcal{H}$. The spectral theorem gives a decomposition of a $d$-dimensional observable $\hat{A}$ as 

\begin{equation}
	\hat{A} = \sum_{i=1}^d \lambda_i \ket{e_i}\bra{e_i} = \sum_{i=1}^d \lambda_i \hat{P}_{\lambda_i}
\end{equation}
where the \brisred{eigenvectors} are the set $\{\ket{e_1},\ldots ,\ket{e_d}\}$ associated with the \brisred{eigenvalues} $\{\lambda_1,\ldots ,\lambda_d\}$. $\hat{P}_{\lambda_i}$ is the \brisred{eigenprojector} associated to the eigenvalue $\lambda_i$.

To describe measurements of a quantum state, we will introduce a \ac{pvm} over a set $\{\lambda_1, \ldots, \lambda_n\}$. The \ac{pvm} is given by a collection of projections $\{\hat{P}_{\lambda_1},\ldots,\hat{P}_{\lambda_n}\}$ over the subsets of the Hilbert space $\mathcal{H}$ such that

\begin{enumerate}
	\item The projections are orthogonal i.e. $\hat{P}_{\lambda_i}\hat{P}_{\lambda_j} = \delta_{ij}\hat{P}_{\lambda_i}$
	\item The projections are complete i.e. $\sum_{i=1}^n \hat{P}_{\lambda_i} = \hat{\mathbb{I}}$
\end{enumerate}

Measurements on a quantum system, $\ket{\psi}$ have outcomes $\{\lambda_1, \ldots, \lambda_N\}$ which are described by a \ac{pvm} on the Hilbert space $\mathcal{H}$. The outcome of the projection is random with a probability distribution

\begin{equation}
	P(\lambda_k) = |\hat{P}_{\lambda_k}\ket{\psi}|^2 = \bra{\psi}\hat{P}_{\lambda_k}\ket{\psi}
\end{equation}

Given that the outcome of the measurement was $\lambda_k$, the resulting state after normalisation will be

\begin{equation}
	\ket{\psi'} = \frac{\hat{P}_{\lambda_k}\ket{\psi}}{|\hat{P}_{\lambda_k}\ket{\psi}|}
\end{equation}

This postulate describes the inherent randomness in quantum mechanics. The amount of information that we can gain by measuring a quantum state or system is limited regardless of the precision of the measurement device. It also raises the question of how the state is disturbed during measurement. The measurement process gives rise to different interpretations of quantum mechanics which are beyond the scope of this thesis.

%This postulate is generally known as the Born rule \cite{Born1926}.

\subsubsection*{Postulate 4: Composite Systems}

The final postulate allows us to describe systems that are made from more than one state. Any composite quantum states exist in a Hilbert space that is the tensor product of the Hilbert spaces of the individual states. If each of $N$ states is prepared individually, the combined state is simply

\begin{equation}
	\ket{\psi} = \ket{\psi_1}\otimes\cdots\otimes\ket{\psi_N}
\end{equation}

This postulate also applies to states that are not simply the product of individual states i.e. not \emph{product} states. Such systems are called \brisred{entangled} states.

\subsection{Quantum Bits and Entanglement}

\begin{figure}
	\centering
	\def\svgwidth{0.5\textwidth}
   	\import{chapters/chapter02/fig02/}{Bloch_Sphere.pdf_tex}
   	\caption[Bloch ball representation of a qubit]{Bloch ball representation of the qubit state $\ket{\psi}$. We introduce the real variables $0\le\theta\le\pi$ and $0\le\varphi<2\pi$ where an unphysical global phase has been omitted.}
   	\label{fig:bloch}
\end{figure}

Similar to how information is represented as bits in classical computers, we can introduce the concept of a \brisred{quantum bit}, which we will call \brisred{qubits}. As mentioned above, qubits are two-level systems that exist in $\mathcal{H}_2$ where we introduce the orthonormal basis $\ket{0}$ and $\ket{1}$, called the computational basis. It is more usual to consider qubits in a Bloch representation,

\begin{equation}
	\ket{\psi} = \cos\left(\nicefrac{\theta}{2}\right)\ket{0} + \mathrm{e}^{i\varphi}\sin\left(\nicefrac{\theta}{2}\right)\ket{0}
\end{equation}
where $0\le\theta\le\pi$ and $0\le\varphi<2\pi$ as shown in figure \ref{fig:bloch}. To account for mixed states, this representation is a ball, rather than a sphere. Pure states exist on the surface on the ball, while mixed states exist within.

We introduce the Pauli matrices

\begin{equation}
	\hat{x} = \left(\begin{matrix}
		0 & 1 \\
		1 & 0
	\end{matrix}\right),
	\quad
	\hat{y} = \left(\begin{matrix}
		0 & -i \\
		i & 0
	\end{matrix}\right),
	\quad
	\hat{z} = \left(\begin{matrix}
		1 & 0 \\
		0 & -1
	\end{matrix}\right)
\end{equation}
which form the observables of qubits and are given in the computational basis. 

As per postulate 4, we can form composite systems through the tensor product of individual qubits for form more complex systems. Given two qubits,

\begin{equation}
	\ket{\psi} = \alpha\ket{0} + \beta\ket{1} \quad \text{and} \quad \ket{\phi} = \delta\ket{0} + \gamma\ket{1}
\end{equation} 
that were prepared individually, the composite system is

\begin{equation}
	\ket{\psi}\ket{\phi} = \alpha\delta\ket{00} + \alpha\gamma\ket{01} + \beta\delta\ket{10} + \beta\gamma\ket{11}
\end{equation}
which exists in the Hilbert space $\mathcal{H}_4$. Equally, we could consider a state that was prepared in $\mathcal{H}_4$ that is not a product state. Then general form of such a state is

\begin{equation}
	\ket{\Psi} = a\ket{00} + b\ket{01} + c\ket{10} + d\ket{11}
\end{equation}
where we are only constrained by normalisation conditions that $|a|^2 + |b|^2 + |c|^2 + |d|^2 = 1$. This allows us to introduce the states

\begin{align}
	\label{eq:BellStateStart}
	\ket{\Phi^+} &= \frac{\ket{00} + \ket{11}}{\sqrt{2}}\\
	\ket{\Phi^-} &= \frac{\ket{00} - \ket{11}}{\sqrt{2}}\\
	\ket{\Psi^+} &= \frac{\ket{01} + \ket{10}}{\sqrt{2}}\\
	\ket{\Psi^-} &= \frac{\ket{01} - \ket{10}}{\sqrt{2}}
	\label{eq:BellStateEnd}
\end{align}
which are \brisred{entangled} and form the \brisred{Bell basis}. If we consider measuring the Bell states, perhaps in the computational basis, we will find the outcomes will be random but always exactly correlated. In fact, we would find that the outcomes would be correlated for any basis choice.

These states were first introduced in the seminal paper by Einstein, Podolsky and Rosen in 1935 \cite{einstein1935can}, but it was Schr\"{o}dinger who coined the term

\begin{quote}
\textit{``I would not call [entanglement] one but rather the characteristic trait of quantum mechanics, the one that enforces its entire departure from classical lines of thought.''}\sourceatright{ - Erwin Schrödinger \cite{schrodinger1935discussion}}
\end{quote}

It wasn't until 1964 that a method to test whether entanglement was non-classical was proposed by John Bell who placed a bound on correlations between systems \cite{Bell1964Einstein}. From the theory proposed by Bell, Clauser, Horne, Shimony and Holt (CHSH) proposed an experiment in 1969 that would verify the theorem \cite{CHSH}. The CHSH inequality is given by 

\begin{equation}
	|S| \le 2
\end{equation}
for
\begin{equation}
	S = E(a,b) - E(a, b') + E(a',b) + E(a',b')
\end{equation}
where $E$ is the expectation value for $\{a,a'\}$ measurements on the first state and $\{b,b'\}$ on the second. This bound has been violated numerous times experimentally, most notably in 2015 when three separate experiments closed all loopholes \cite{Giustina2015, shalm2015, hensen2015}.

\subsubsection{Encoding}

How $\ket{0}$ and $\ket{1}$ are encoded will depend on the physical system being used. Some notable examples of encoding are electron spins; nuclear spins; photon path, polarisation or timing; and superconducting flux. Each has their own benefits and drawbacks and will be suited for different tasks in quantum information processing. 

Of course, for quantum communication we will be interested in sending quantum information between distant parties, Alice and Bob. Light is already fundamental in modern communications networks. By considering single-photons, we can encode qubits that are compatible with much of the existing network architecture. These encodings will be discussed further in section \ref{sec:photon_encoding}.

% for which photons will are well suited. The encodings for photons will be discussed further below.

%However, interactions between photons remains challenging meaning other encodings are more widely explored for quantum computing applications.

%For communication protocols, we will find it useful to encode information using time and phase of quantum states as it is generally less susceptible to fluctuations in real-world fibres.

\subsection{No-Cloning Theorem}

A fundamental theorem in quantum mechanics is the {\color{bristol-red} no-cloning theorem} which will also underpin the security of \ac{QKD}. The general principle of the theorem states that there is no physical process that copies, or clones, an arbitrary quantum state. We can state this more formally:

%\begin{theorem}[No-Cloning Theorem]
%	 The no-cloning theorem states that there is no quantum channel, $\mathcal{E}$, from $M(\mathbb{C}^d)$ to $M(\mathbb{C}^d) \otimes M(\mathbb{C}^d)$ such that
%	 \begin{equation*}
%	 	\mathcal{E}(\ket{\psi}\bra{\psi}) = \ket{\psi}\bra{\psi} \otimes \ket{\psi}\bra{\psi}
%	 \end{equation*}
%	 for all states $\ket{\psi} \in \mathbb{C}^d$.
%\end{theorem}

\begin{theorem}[No-Cloning Theorem]
	 The no-cloning theorem states that there is no unitary transformation such that
	 \begin{equation}
	 	\hat{U}\ket{\psi}\ket{a} = \ket{\psi}\ket{\psi}
	 \end{equation}
	 for an arbitrary quantum state $\ket{\psi}$ and an ancillary quantum state $\ket{a}$.
\end{theorem}

Consider two arbitrary quantum states $\ket{\psi}$ and $\ket{\phi}$. Let us assume that there is a unitary that allows us to copy a quantum state

\begin{align}
	\hat{U}\ket{\psi}\ket{a} &= \ket{\psi}\ket{\psi}\\
	\hat{U}\ket{\phi}\ket{a} &= \ket{\phi}\ket{\phi}
\end{align}

If we compare the overlap of the states on the left and right of the equation we find

\begin{equation}
	\braket{\psi|\phi} = \left(\braket{\psi|\phi}\right)^2
\end{equation}
which has two solutions. Either $\braket{\psi|\phi} = 0$ and the states are orthogonal or $\braket{\psi|\phi} = 1$ and the states are identical. Therefore, we cannot clone an \emph{arbitrary} quantum state.

\subsection{Quantum Computing}

To motivate the introduction of quantum-safe communication, we will introduce some quantum computing algorithms. While the progression in quantum computing has played a large part in the motivation of quantum-safe cryptography, it is not the only factor. However, an algorithm that will break modern cryptography provides a very tangible reason to replace public-key algorithms.

%While quantum computing has been a big motivator for moving to new encryption, we still have no assurances that the trap-door functions we rely on for security are secure even against classical computers.

Shor's algorithm efficiently factors composite numbers into their prime factors \cite{shor1994} meaning that both \ac{rsa} and Diffie-Hellman are vulnerable to attacks from a quantum computer. The estimated number of physical qubits needed to factor \ac{rsa}-2048 has reduced drastically in recent years from one billion \cite{mosca2018, fowler2012} to twenty million \cite{gidney2019}. A state-of-the-art quantum processor was recently demonstrated with fifty qubits \cite{arute2019quantum}. While very far away from the estimates, it is a necessary first step towards ubiquitous quantum computers.

Grover's algorithm could also be used to search through keys in a block cipher protocol \cite{grover1996fast}. However, the speed-up is only $\sqrt{n}$ which has been proven as the lower bound for an unordered search \cite{Bennent1997}. Therefore, the same security in a post-quantum world can be achieved by doubling the key length i.e. moving from 128 bit to 256 bit \ac{aes}.

\section{Quantum Key Distribution}

\Acf{QKD} is a fundamentally new method of key exchange by exploiting quantum mechanics to ensure security that is guaranteed by the laws of physics. It falls under the umbrella of \brisred{quantum safe} cryptography and, together with post-quantum cryptography, is likely to form a crucial part of future global networks.

Unlike the public-key protocols introduced in section \ref{sec1:crypto}, \ac{QKD} does not rely on assumed computationally hard trapdoor functions. Instead, the security is based on the no-cloning theorem and, with minimal assumptions, can provide information-theoretic secure key exchange.

The concept for \ac{QKD} was derived idea developed in the 1960's by Stephen Wiesner. However, this wouldn't be published until 1983 \cite{quantum_money}. Through \emph{conjugate coding} Wiesner developed two schemes. The first allowed Alice to send two messages to Bob, but Bob could only choose to read one message depending on his basis. Upon measurement, the other message would be destroyed.

The second idea allow uncounterfeitable money to be created based simply on the superposition principle. By introducing two sets of orthogonal states, \{$a$, $b$\} and \{$\alpha$, $\beta$\} where

\begin{equation}
	\alpha = \frac{a+b}{\sqrt{2}} \quad \text{and} \quad \beta = \frac{a-b}{\sqrt{2}}
\end{equation}
banks could print a serial number on each note made from the four states. The basis choice associated with each qubit would be stored in the bank records. Upon return of a note, the bank would be able to measure each qubit in the correct basis to verify that it was the original. Any attempt to counterfeit a note would result in errors during the verification. 

%There are many different ways that information can be encoded on photons. This had led to quite a few different protocols. Here we will discuss 

In a \ac{QKD} protocol, we will introduce a new resource that is available to Alice and Bob:  a quantum channel. This channel allows Alice to send quantum states to Bob. We will, of course, assume that the channel is prone to errors but we need not assume that the channel is private. Both Eve and Mallory will have full access. Uniquely, \ac{QKD} offers the ability to detect an eavesdropper as measurements of quantum states causes them to collapse.

As with public-key cryptography, we will require an \emph{authenticated} classical channel between Alice and Bob to avoid \ac{mitm} attacks. This classical channel, as with the quantum channel, can be completely public. The authentication of the classical channel is something that is not solved with \ac{QKD} and will need to utilise post-quantum algorithms. As Alice and Bob will require some initial bit of shared secret to authenticate their messages, \ac{QKD} is sometimes referred to as quantum key \emph{expansion}. 

As the security of \ac{QKD} systems stems from the laws of physics it has often been claimed to be \emph{perfectly} secure. However, we will see later that side-channels or imperfections in different physical implementation will decrease the security. It is perhaps more accurate to say that, provided that the key is securely stored following a protocol, the security of that key does not decrease with advances in computing. 

%Of course, we will need to be aware that Eve and Mallory are listening and trying to gain valuable secret information.

%In this section, we will introduce the main concepts behind \ac{QKD} and some of the original protocols. 

\subsection{Protocols}

\Ac{QKD} protocols are generally separated into three distinct categories: \acf{dvqkd}, \acf{cvqkd} and \acf{dprqkd}. Here, we will introduce each with discussion of how qubits are encoded and the practicalities of each.

\subsubsection{Discrete-Variable}

\begin{figure}
	\centering
	\def\svgwidth{0.9\textwidth}
   	\import{chapters/chapter02/fig02/}{qkd_encoding.pdf_tex}
   	\caption[\acs{dvqkd} and \acs{cvqkd} encoding]{States used to encode information for \acs{dvqkd} and \acs{cvqkd}.}
\end{figure}

The first ways that it was conceived that information could be encoded on photons was using orthogonal states to represent 0 or 1. For example, one could use the vertical and horizontal polarisations of a photon where diagonal and anti-diagonal could be used to encode superpositions. 

A more practical method of encoding information when intending to use fibre optics to transmit the photons is a time-bin encoding. Now the timing and relative phase information between two pulses gives us a complete encoding of all quantum states. Provided that the two time-bins are closely spaced they won't be as affected by fluctuations in the fibre.

\subsubsection*{BB84}

The first \ac{QKD} protocol to be developed \ac{bb84}, named after its inventors \cite{BB84} which was demonstrated in a proof-of-principle experiment a few years later \cite{bennett1992experimental}. Since then, there have been numerous demonstrations of \ac{bb84} either demonstrating new platforms or realising backbone quantum networks \cite{pirandola2019advances}. \Ac{bb84} remains a favourite choice for systems despite alternative protocols being available \cite{sarg2004, DPS-QKD, COW-QKD, B92}.

We will introduce the four states

\begin{equation}
	\ket{0}, \quad \ket{1}, \quad \ket{+} = \frac{\ket{0} + \ket{1}}{\sqrt{2}}, \quad \text{and} \quad \ket{-} = \frac{\ket{0} - \ket{1}}{\sqrt{2}}
\end{equation}
that we will refer to as the \ac{bb84} states.

\Ac{bb84} is a {\color{bristol-red}prepared-and-measure} protocol where Alice and Bob play complementary roles. Alice {\color{bristol-red}prepares} a quantum state which she sends to Bob who {\color{bristol-red}measures} the state in a predefined way. 

\begin{algorithm}[Bennett-Brassard 1984]
\begin{enumerate}
	\item Alice generates two uniform and random bit strings, $b_a$ and $n$. The first bit string will determines the basis, $Z$ = \{$\ket{0}$,$\ket{1}$\} if 0 or $X$ =  \{$\ket{+}$,$\ket{-}$\} if 1. The second bit string $n$ determines whether the first or second state in each basis is chosen. 
	\item Each state is sent sequentially to Bob via the quantum channel.
	\item Similarly, Bob generates a bit string, $b_b$ uniformly and random. This determines either the $X$ or $Z$ basis in which to measure. These are equivalent to the Pauli $\hat{x}$ and $\hat{z}$ observables introduced earlier.
	\item Bob measures each of the bits in turn using the randomly selected bases and records the outcomes.
	\item Alice and Bob announce their basis bit strings, $b_{\{a,b\}}$, over the authenticated classical channel. They discard all events where the bases didn't match.
	\item Using some of the remaining events, they compare their values to check for eavesdropping. If an unacceptable amount do not match, they abort the protocol.
	\item Finally, privacy amplification can be applied to satisfy security requirements.
\end{enumerate}
\end{algorithm}

Upon successfully finishing a \ac{bb84} protocol, Alice and Bob share a symmetric key. If they wished, they could use a \ac{otp} cipher to ensure information-theoretic secure communication. However, as one bit of key is needed to encrypt each bit of message, more pragmatic systems will implement a block cipher. 

\subsubsection*{BB84 with Decoy States}

As single-photon sources remain technically challenging to engineer, \ac{QKD} system have preferred using weakly attenuated laser pulses to generate \acp{wcs}. However, as \ac{wcs} have a distribution of photon numbers, it opens the system up to a \ac{pns} attack. 

In such an attack, we assume that Mallory has access to a device that can make a non-demolition of the photon number of the state. She will block each state which contains zero or one photons, and for each multi-photon state will keep one photon in a memory until Alice and Bob announce the chosen bases. 

One can consider the security of using \ac{wcs} in the \ac{bb84} protocol describe above. However, this will offer a far reduced secret key rate due to the information leakage \cite{Norbert2000Security, Brassard2000Limitations}. Instead, decoy state protocols have been developed allowing Alice and Bob to bound the knowledge that could have been accessed by Eve \cite{Lo2005}. 

During the first step of the protocol, Alice will randomly modulate the intensity of her \ac{wcs}. Typically, Alice and Bob will agree on two decoy states that will be chosen to optimise the key rate depending on the error rate. As Mallory has no way to tell which decoy state Alice has sent, she is more likely to block the decoy states than the signal states. During the basis discussion stage, Alice and Bob can bound the number of true single-photon events and use that to inform how much privacy amplification is required, or to abandon the protocol altogether.

Ideally, Alice and Bob would like to have perfect, unbiased estimators of the single-photon events in their key exchange. However, in a realistic exchange, they will necessarily have a finite number of events in which to estimate the parameters. Therefore, they must instead consider the finite key effects and calculate bounds on the single-photon statistics \cite{gottesman2004security}. 

\subsubsection*{E91}

Independently of the development of \ac{bb84}, another protocol was developed by Artur Ekert in 1991 \cite{E91}. The \ac{e91} protocol takes advantage of the correlations of entangled states and realised that by distributing two entangled photons between Alice and Bob meant that their measurements would be correlated. Further, using Bell's test and CHSH measurements they could verify that an eavesdropper hadn't interfered with the states \cite{Bell1964Einstein, CHSH}. 

\begin{algorithm}[Ekert 1991]
\begin{enumerate}
	\item Alice generates an entangled state, for example one of the Bell states in equations \ref{eq:BellStateStart} to \ref{eq:BellStateEnd}.
	\item She sends one qubit from the state to Bob and keeps one for herself.
	\item Alice and Bob independently and randomly choose one of the CHSH angles \cite{CHSH} in which to measure their qubit.
	\item After a sufficient number of states, they each announce their basis choices while keeping the results secret.
	\item They use a random set of the measurement results to calculate the $S$ and verify that the state was entangled.
	\item If they can successfully violate a Bell inequality, the remainder of the events should be correlated.
	\item Finally, they apply error reconciliation and privacy amplification as required.
\end{enumerate}
\end{algorithm}

As Alice and Bob are able to verify that the state is entangled, we don't need to assume that one of them is generating the entangled state. The protocol could be equally secure if Charlie, Eve or Mallory were generating the state. Any tampering with the state would only result in a reduction of key rate.

Entanglement-based protocols are often used as a method to prove the security of other prepare-and-measure protocols. For example, the chosen randomness used in a \ac{bb84} protocol is equivalent to a postponed measurement on an entangled state.

\subsubsection{Continuous-Variable}

More recently there has been an interest in \ac{cvqkd} due to the compatibility with current telecommunication equipment \cite{laudenbach2018continuous}. Instead of \acp{spd}, Bob can use photodiodes to perform measurements on the states through homodyne (or heterodyne) detection. 

\Ac{cvqkd} uses states that are described in Hilbert spaces of infinite dimension. Protocols have been proposed that encode information in Gaussian states \cite{Ralph1999} and squeezed states \cite{Hillery2000Quantum}. Both are considered prepare-and-measure schemes and the outline of the protocols remains the same.

\begin{algorithm}[Continuous-Variable Quantum Key Distribution]
\begin{enumerate}
	\item Alice encodes a random variable in a quantum state. In the case of Gaussian modulation, Alice encodes the information from a set of overlapping Gaussian states. In the case of squeezed state encoding, she randomly chooses between squeezing in the $\hat{x}$ and $\hat{p}$ quadratures.
	\item The states are sent to Bob through the quantum channel which is typically assumed to be a thermal-loss channel.
	\item Bob performs a homodyne (or heterodyne) measurement, switching between the $\hat{x}$ and $\hat{p}$ quadratures.
	\item Alice and Bob use the classical channel to compare the quadratures they chose.
	\item Using some of the matching bases, they perform parameter estimations to bound the knowledge that could have been gained by Eve or Mallory.
	\item Finally, they perform privacy amplification as required.
\end{enumerate}
\end{algorithm}

%Protocols date back to 1999  Instead of \acp{spd}, measurements can be achieved through homodyne measurements which is already used in modern networks. 

%\Ac{cvqkd} can also be performed with squeezed states \cite{Hillery2000Quantum}.

%\Ac{cvqkd} using states that are described in Hilbert spaces with infinite dimension. It is still a prepare-and-measure scheme. Bob requires a local oscillator which acts as a phase reference. The local oscillator needs to be distributed between Alice and Bob.

%The states are encoded in quadrature space and measured using homodyne receivers, much like how classical information is transmitter.

For homodyne detection to be meaningful, Alice and Bob need to make sure they share a local oscillator. This provides a phase reference so that the prepared states can be reconciled with the measured states. Often, the local oscillator is multiplexed with the encoded state to avoid any mismatch in the phase fluctuations from the quantum channel \cite{jouguet2013experimental}.

For a long time, there were questions about the security of the \ac{cvqkd} systems. There has since been a composable proof against general attacks \cite{Leverrier2015Composable} which was later extended to better consider finite key effects \cite{Leverrier2017Secrity}. The post-processing requirements for \ac{cvqkd} are also computationally challenging. Unlike in \ac{dvqkd}, where only photons that reach Bob require analysis, \ac{cvqkd} requires each potentially event to be analysed which leads to large overheads.

%There are still some questions about the security of \ac{cvqkd} \cite{Scarani2009security} and the practicality of the post-processing requirements. Unlike in \ac{dvqkd} where only successful measurements need to be analyse, each potential event needs to be analysed. This leads to big overheads during the error correction stages of the protocol.

\subsubsection{Distributed-Phase-Reference}

\Ac{dprqkd} protocols were developed in order to solve practical issues with the security of \ac{QKD} systems when using realistic equipment \cite{diamanti2016practical}. As single-photon sources are not widely available, system typically use coherent states that are prone to \ac{pns} attacks. In \ac{dprqkd} protocols, the qubits are encoded in timing or relative phases of signals where a joint measurement of subsequent signals is required. Protocols include \ac{cow} \cite{COW-QKD} and \ac{dps} \cite{DPS-QKD}.

\subsection{Security and Hacking}

Generally, security proofs for \ac{QKD} protocols fall into three categories. \brisred{Individual} attacks allow an adversary to interact with each state individually as it is sent. A \brisred{collective} attack allows them to process the states and store state indefinitely in a quantum memory. Finally, a \brisred{coherent} attack allows Mallory to generate arbitrary ancilla states which can interact with the states and be measured jointly. Coherent attacks only limit Mallory to the laws of physics and nothing more.

A proof of a protocol gives a security guarantee against an adversary but only under certain assumptions. It is often claimed that \ac{QKD} is information-theoretic secure when used with a \ac{otp}. However, verifying the assumptions required for security is challenging in a practical setting and has been scrutinised by the quantum hacking community. 

As the systems exist in the real world, the security of a key exchange is only as good as the model used to describe it. Any part of the system that is not fully characterised may leak information through \brisred{side-channels} allowing Eve to gain knowledge of the secret key. Such channels may include polarisation, after-pulses or wavelength. The concept of side-channels is not something unique to \ac{QKD} and has been previously exploited to break \ac{rsa}-4096 keys using a smartphone microphone \cite{Genkin2014RSA}. In fact, side-channels were present even in the first demonstrations by Bennett and Brassard:

%While it is true that under certain assumptions, a \ac{QKD} protocol can be proven to be information-theoretic secure, \emph{verifying} the assumptions in a practical setting becomes more challenging. 

% this is only true under certain assumptions.  Mallory, however, is quite determined to gain any knowledge of the secret key.  the security model. However, it is challenging to completely model every part of a system. This leaves side-channels, where sensitive information is leaked through uncharacterised channels such as sound or temperature.

%The first demonstration of \ac{QKD} by Bennett and Brassard was also the first to demonstrate side-channel attacks in a \ac{QKD} system: 

\begin{quote}
\textit{``…power supplies make noise, and not the same noise for the different voltages needed for different polarizations... Thus, our prototype was unconditionally secure against any eavesdropper who happened to be deaf!''} - Gilles Brassard \cite{Brassard2005}
\end{quote}

\begin{table}[t]
\centering
\begin{tabular}{lll}
	\textbf{Attack}\hspace{4cm} & \textbf{Target}\hspace{4cm}  & \textbf{References}\\ 
	Photon-number splitting    & Photon source          & 	\cite{Norbert2000Security, Brassard2000Limitations}  \\
	Inter-symbol interference      & State modulation      & 	\cite{yoshino2018quantum}   \\
	Trojan horse & Phase modulation      &   \cite{Gisin2006, jain2014trojan, sajeed2017invisible} \\
	Time-shift 		&	Single-photon detectors	& \cite{Qi2007Time, Zhao2008Quantum} \\
	Detector Blinding	&	Single-photon detectors 	&	\cite{Lydersen2010a, Gerhardt2011a}	\\
	Laser damage	&	Any	& \cite{Bugge2014Laser, Makarov2016Creation}
\end{tabular}
\caption[Attacks demonstrated against QKD systems]{Attacks demonstrated against \acs{QKD} systems. Further details about the attacks, including countermeasures and bounds, can be found in refs. \cite{pirandola2019advances, xu2019quantum}.}
\label{tab:hacks}
\end{table}

Since then, there have been a number of attacks proposed on \ac{QKD} systems exploiting uncharacterised channels \cite{xu2019quantum}. Many have been demonstrated against both research and commercial systems. A brief list is compiled in table \ref{tab:hacks}, although this list is far from exhaustive. 

%numerous demonstrations that \ac{QKD} system are insecure under real world condition. These have demonstrated vulnerabilities in all parts of both the transmitters and receivers: transmitter modulation \cite{}. 

One of the most vulnerable parts of a \ac{QKD} system seems to be the \acp{spd}. Due to their complexity, they have been exposed to many attacks which have allowed Mallory to be in complete control of a key exchange \cite{Makarov2006, Gerhardt2011a, Lydersen2010a, Lydersen2010b, Lydersen2011, Sauge2011, Makarov2009, Wiechers2011}.

%It has even been demonstrated that Bell violations can be experimentally faked \cite{Gerhardt2011b}.

Counter measures have been proposed to alleviate specific attacks against certain systems \cite{Lydersen2010c, Yuan2010}. There have also been bounds set for the amount of information that can be gained against particular attacks such as the trojan horse attack \cite{Lucamarini2015Practical}. However, continually testing and patching systems with physical changes will be impractical for any ubiquitous network. A new vulnerability could render an entire system insecure requiring the hardware to be modified or replaced.

%It is possible to develop methods to counter certain attacks. However, verifying the security and continually patch new exploits and vulnerabilities will be impractical for a quantum secured network.

\subsection{Device-Independence}

Realising that the arms race between cryptographers and hackers would lead to an unending cycle, interest turned to reducing the assumptions required for security of a system \cite{Mayers1998, Acin2007, Barrett2005}. In particular, how the assumptions about specific equipment can be removed. \Ac{diqkd} provides a method for Alice and Bob to verify the operation of the equipment \emph{during} a protocol through a Bell test. Alice and Bob then only need to ensure that there are no communication channels out of their lab and that the laws of physics are correct. 

Implementing a \ac{diqkd} scheme remains incredibly challenging with modern technology as it requires near unity single-photon detection \cite{pironio2009device}. It also requires a loophole-free Bell test to be performed which, while such experiments have recently been realised \cite{Giustina2015, shalm2015, hensen2015}, had rates that were far slower than needed for communication protocols. 

In order to create more practical systems with improved security, protocols were developed which relaxed the device-independence. \Ac{MDI} removes the need to characterise the detection system, which remains challenging due to their complexity \cite{mdi-qkd}. It does, however, still required characterisation of the transmitters. \Ac{MDI} will be discussed further in chapter \ref{chap:mdiqkd}.

%To remove potential attacks on \ac{QKD} systems, protocols have been developed to relieve some of the assumptions required for security in a key exchange. \Ac{diqkd}  \cite{}. 

%Creating accurate models for equipment is particularly challenging. By removing assumptions about their operation, full characterisation is no longer required.

%Fully \ac{diqkd} is practically challenging due to the requirements of the equipment. \ac{MDI} can relieve assumptions about the detector device and is possible with current technology \cite{mdi-qkd}, as we will see in chapters \ref{chap:mdiqkd}. 

\section{Integrated Photonic Circuits}

On the route to an accessible technology, we will need to find a platform which allows for many devices to be made without an exponential increase in resources. Photonic integration is by no means a new idea having been considered for more than half a century \cite{miller1969}. However, the techniques have recently been adopted by the quantum photonic community for information processing.

In this section, we will cover the basic concepts in quantum photonics before discussing integrated photonics circuits. Some fundamental components that are used to create circuits will be presented and, finally, some of the more common platforms for photonic integration will be introduced.

%\Acp{pic} are still widely being research and although there are many commercial applications there are also many challenges before they become mass producible. Packing, the process of electrically and optically connecting a \ac{pic}, is still a highly manual process and typically each \ac{pic} package will need custom built. We will see in chapter \ref{chap:future} how packaging problems can limit \ac{pic} operation.

%In the search for a platform for \ac{QKD} protocols, we will need to define certain criteria which will need to be met.
%
%\begin{itemize}
%	\item Creation of quantum light
%	\item High-speed modulation
%	\item Detection of single-photons
%\end{itemize}

\subsection{Quantum Photonics}

We will begin with the description of a quantised electromagnetic field as a harmonic oscillator with associated annihilation and creation operators, $\hat{x}$ and $\hat{p}$. For a more complete description or derivation of these operators, we refer the reader to any number of quantum optics textbooks \cite{gerry2005introductory, fox2006quantum, loudon2000quantum}. The energy in the field corresponds to discretised packets of energy known as photons. The number of excitations in the field is described in the Fock basis and written as $\ket{n}$ for $n\in \mathbb{Z}_{\ge 0}$. The annihilation and creation operators change the photon number accordingly,

\begin{alignat}{2}
	&\hat{a}\ket{n} &&= \sqrt{n}\ket{n-1}\\
	&\hat{a}^\dagger \ket{n} &&= \sqrt{n+1}\ket{n+1}\\
	&\hat{a}\ket{0} && = 0
\end{alignat}
where the final equation imposes that the energy of the field must be positive. The number operator, $\hat{n} = \hat{a}^\dagger\hat{a}$, is an eigenvector of Fock states such that

\begin{equation}
	\hat{n}\ket{n} = n\ket{n}
\end{equation}

%We will take as our starting point the quadrature variables  which define the position and momentum operators, respectively, in our 

%A photon number, or Fock, state with $n$ photons is represented by $\ket{n}$. Annihilation  and creation operations, $\hat{a}$ and $\hat{a}^\dagger$ give the relations

An interesting state in the quantised electromagnetic field is the coherent state, which exist at the boundary of quantum and classical. Coherent states satisfy 

\begin{equation}
	\hat{a}\ket{\alpha} = \alpha\ket{\alpha}
\end{equation}
for $\alpha \in \mathbb{C}$. As the Fock states form a complete set, we must be able to write

\begin{equation}
	\ket{\alpha} = \sum_{n=0}^\infty c_n \ket{n}
\end{equation}
which of course must then satisfy

\begin{equation}
	\hat{a}\ket{\alpha} = \sum_{n=1}^\infty c_n \sqrt{n} \ket{n-1} = \alpha  \sum_{n=0}^\infty c_n \ket{n}
\end{equation}

Each $c_n$ is uniquely determined in terms of $\alpha$ and $c_0$. By imposing normalisation conditions on the state, we arrive at

\begin{equation}
	\ket{\alpha} = \mathrm{e}^{-\frac{|\alpha|^2}{2}} \sum_{n=0}^\infty \frac{\alpha^n}{\sqrt{n}} \ket{n} 
\end{equation}

The generation of coherent states from vacuum is described through the displacement operator, $\hat{\mathcal{D}}(\alpha)$, and defined as

\begin{equation}
	\hat{\mathcal{D}}(\alpha) = \mathrm{e}^{\alpha\hat{a}^\dagger - \alpha^\ast\hat{a}}
\end{equation}
which acts on the vacuum state, $\ket{0}$, to generate a coherent state with average photon number $|\alpha|^2$:

\begin{equation}
	\ket{\alpha} = \hat{\mathcal{D}}(\alpha)\ket{0}
\end{equation}

The study of coherent states stemmed from an interest in the boundary of quantum and classical mechanics and the concept is almost as old as quantum mechanics itself \cite{Schrodinger1926}. More practically speaking, coherent states are readily produced from lasers which makes them far more practical for \ac{QKD} than probabilistic single-photon sources.

\subsection{Photon Encoding}
\label{sec:photon_encoding}

\begin{figure}
	\centering
	\def\svgwidth{\textwidth}
   	\import{chapters/chapter02/fig02/Integrated_optics/}{encodings.pdf_tex}
   	\caption[Quantum information encoding with photons]{Information can be encoding in photons in different ways where superposition of each basis give access to the full Bloch sphere. Subscripts refer to the mode than the Fock state is in. \textbf{a} Orthogonal polarisations of light can be used as a basis, typically horizontal, $H$, and vertical, $V$ are chosen. \textbf{b} The position, or path encoding can be used as a spatial encoding. Basis states are given by a photon either in the top path, $t$, or the bottom path, $b$. \textbf{c} The early or late arrival time of a photon, $e$ and $l$, can be used to encode the computational basis.}
   	\label{fig:encodings}
\end{figure}

We have discussed how single-mode photons can be represented but have yet to mention exactly what mode is used to encode information. This section will discuss the degrees of freedom available in photonics in which to encoding information, which are shown in figure \ref{fig:encodings}.

\subsubsection*{Polarisation}

The orientation of the electric field can be used to encode information in the polarisation.  We can describe the computational basis by encoding $\ket{0}$ as a horizontal photon, $\ket{1}_H$, and $\ket{1}$ as vertical, $\ket{1}_V$. Relative phases and amplitudes between these states allow the entire Bloch sphere to be access. The manipulation of the states is performed through sequences of half and quarter wave plates. Polarisation is typically used in free-space quantum optics where there is minimal rotation of the fields. More recently, there have been developments in integrated optics that have allows waveguide polarisation rotation \cite{smit2014} meaning it could be a useful encoding scheme in the future.

\subsubsection*{Path}

The spatial mode of photons can be use to encode qubits in which path they are following. The phase stability of integrated circuits means that paths are the typically encoding of choice. Two separate waveguides provide different paths which we will call the top, $t$, and bottom, $b$, paths. These can, with relative phases and intensities between the paths, encode arbitrary qubits in a computational basis. Beam splitters and phase modulators are used to manipulate the states and linear optic components can convert from path encoding to polarisation easily. 

\subsubsection*{Time-bin}

Finally, the time of arrival can be used to form a basis in distinct bins. Early and late photons, $e$ and $l$ modes, can be used to represent $\ket{0}$ and $\ket{1}$. Again, the relative phases of the time-bins can be controlled using phase modulation. Delay lines can be used to allow the early and late photons to interact. Time-binning is used in fibre optics where a polarisation or path encoding would be unstable from environmental drifts.

\subsection{Components}

Integrated photonics utilises fundamental components that will be used to guide, control, manipulate and detect photons. Here we will describe some fundamental components that will facilitate quantum information processing through linear optics.

\subsubsection*{Waveguides}

\begin{figure}[t]
	\centering
	\includegraphics[width=0.9\textwidth]{./Integrated_optics/waveguides.pdf}
	\caption[Main types of waveguide structures]{There are three main types of waveguide structures:  \textbf{a} Slab; \textbf{b} Rib; \textbf{c} Strip. The core is shown in purple while the cladding is shown in red.}
	\label{fig:waveguides}
\end{figure}

To manipulate the path of lights, waveguide structures are created in a \ac{pic} which guide light through total internal reflection. It is well established that the theory of light is is governed by Maxwell's equations which provide a set of relationships between the electric field, $\mathcal{E}$, and the magnetic field, $\mathcal{H}$ \cite{lifante2003integrated}. For light propagating in an insulating material, where there are no free electric charges or currents, the set of equations become

\begin{align}
	\nabla\cdot\mathcal{E} &= 0\\
	\nabla\cdot\mathcal{H} &= 0\\
	\nabla\times\mathcal{E} &= -\mu_0\frac{\partial\mathcal{H}}{\partial t}\\
	\nabla\times\mathcal{H} &= \epsilon\frac{\partial\mathcal{E}}{\partial t}
\end{align}
where $\mu_0$ is the magnetic permeability of free space and $\epsilon$ is the dielectric permittivity of the material. These equations describe the electromagnetic field in a semiconductor material where the energy of the photon is less than the band gap. By combining the above equations, we can derive the wave equations

\begin{align}
	\nabla^2\mathcal{E} + \nabla\left(\frac{1}{n^2}\nabla n^2\mathcal{E}\right) - \epsilon_0\mu_0 n^2 \frac{\partial^2\mathcal{E}}{\partial t^2} = 0\\
	\nabla^2\mathcal{H} + \frac{1}{n^2}\nabla n^2\times\left(\nabla\times\mathcal{H}\right) - \epsilon_0\mu_0 n^2 \frac{\partial^2\mathcal{H}}{\partial t^2} = 0
\end{align}
where the refractive index, $n(\mathbf{r})$, is dependent on the spatial coordinates. These equations yield a plane wave solution which are of the form

\begin{align}
	\mathcal{E}(\mathbf{r}, t) &= E(x,y)e^{i(\omega t - \beta z)}\\
	\mathcal{H}(\mathbf{r}, t) &= H(x,y)e^{i(\omega t - \beta z)}
\end{align}
introducing the complex amplitudes, $E$ and $H$, of the electric and magnetic fields. The angular frequency  of the wave is $\omega$ and $\beta = \nicefrac{\omega N}{c}$ is the propagation constant for some effective refractive index $N$.

%\begin{align}
%	&\nabla^2\mathcal{E} = \mu_0\epsilon\frac{\partial^2\mathcal{E}}{\partial t^2}\\ 
%	&\nabla^2\mathcal{H} = \mu_0\epsilon\frac{\partial^2\mathcal{H}}{\partial t^2}
%\end{align}





%where the charge density, $\rho$, is zero. This is valid in a semiconductor where the photon energy is below the band gap \cite{SilverstoneThesis}.
%
%$\mu_0$ is the magnetic permeability and $\epsilon_0$ is the dielectric permittivity.
%
%We assume that there is no current in the medium typically written as $\sigma = 0$ where sigma is the conductivity.
%
%The relationship between the magnetic field and flux is $\mathcal{B} = \mu_o \mathcal{H}$ if the material is not magnetic.
%
%From these four equations, we can derive
%
%given that there is no conductivity ($\sigma=0$).
%
%For a waveguide, we find the equations \cite{lifante2003integrated}.

\subsubsection*{Multi-Mode Interferometer}

\begin{figure}[t]
	\centering
	\includegraphics[width = 0.6\textwidth]{mmi.pdf}
	\caption[Multi-mode interferometer operating principle]{Illustration of \acs{mmi} operation. The input light excites a superposition of modes with different propagation velocities. Depending on the length, the modes will either constructively or destructively interfere. This figure shows the simplest case for a \si{2 x 2} \acs{mmi} but in principle can be designed to be \si{n x m}.}
	\label{fig:mmi}
\end{figure}

Different spatial modes in photonics experiments are typically interfered using a beam splitter. The traditional half-silvered mirror isn't something that is easily fabricated as a linear optical components, so another method is required.

Directional couplers allow waveguides to interfere as they are brought in close proximity. Evanescent coupling allows the mode to shift from one to the other. The splitting ratio a directional coupler is dependent on the length of interaction. This makes fabricating accurate couplers challenging as small variations in length can cause a large change in splitting ratio.

Alternatively, waveguide modes can be interfere with \acp{mmi} that are based on the self-imaging principle \cite{soldano1995optical}. A schematic is shown in figure \ref{fig:mmi}. Light in the input mode excites a superposition of modes within the \ac{mmi}. As these modes propagate, they constructively and destructively interfere. The length is chosen to allow the light to be split equally when there is constructive interference into two output modes. The same structure can equally be designed to create \si{n x m} splitters. In general, the transfer matrix is

\begin{equation}
	\hat{U}_\mathrm{MMI} = \left(
	\begin{matrix}
	r & t\\
	t & r
	\end{matrix}
	\right)
\end{equation}
where $r$ and $t$ are complex numbers that correspond to the reflectivity and transmission, respectively, and will be constrained such that the matrix is unitary. A balanced \ac{mmi} will have the transformation matrix

\begin{equation}
	\hat{U}_\mathrm{MMI} = \frac{1}{\sqrt{2}}\left(
	\begin{matrix}
	1 & i\\
	i & 1
	\end{matrix}
	\right)
\end{equation}

\subsubsection*{Phase modulation}

There are a few different methods that we can use to control the phase of photons. Of course, it only makes sense to talk about a relative phase, such as when a photon is in superposition. The transfer matrix of a phase modulator over two modes is given by

\begin{equation}
	\hat{U}_\mathrm{PM} = \left(
	\begin{matrix}
	1 & 0\\
	0 & e^{i\theta}
	\end{matrix}
	\right)
\end{equation}
where a phase modulation is applied to the second mode, relative to the first which is left untouched.

Thermo-optic effects can be used to change the phase of a waveguide by changing the refractive index and is ubiquitous in integrated photonics. The change of phase, $\Delta\theta$, is given by

\begin{equation}
	\Delta\theta = \frac{2\pi\cdot L\cdot (\Delta T)}{\lambda}\frac{\mathrm{d}n}{\mathrm{d}T}
\end{equation}
where $L$ is the length of the modulator, $\Delta T$ is the change in temperature, $\lambda$ is the wavelength of light and $\nicefrac{\mathrm{d}n}{\mathrm{d}T}$ is the thermo-optic coefficient of the material. \Acp{topm} can achieve very good precision and stability with very low loss but have a slow response time. 

To achieve faster modulation, which will be required for communication protocols, electro-optic effects have been explored. The effects exploit non-linear properties of a material where a change of phase is given by

\begin{equation}
	\Delta\theta = \frac{2\pi\cdot L\cdot  \chi^{(2)}\cdot E}{\lambda}
\end{equation}
where $E$ is the applied electric field and $\chi^{(2)}$ is the second-order non-linear optical susceptibility. Therefore, only materials that exhibit a second-order component can exhibit this type of electro-optic modulation. Due to the centro-symmetry in silicon, other methods are required. However, lithium niobate modulators are commercially available with bandwidths up to \SI{40}{GHz} while \SI{100}{GHz} has been demonstrated in laboratories \cite{Louay2001Advances, Atsushi2010NRZ}.

Carrier effects can be used in semiconductor materials to change the absorption in p-i-n or p-n junctions which in turn generates a phase relationship through a Kramer-Kronig relation. These modulations are used where high-bandwidths are required in materials that don't have intrinsic electro-optic effects, such as silicon. These modulators have been shown to operate at \SI{10}{GHz} bandwidths \cite{Sibson2017Si}.

A final effect that we will mention is the \ac{qcse}. The effect presents itself in materials where the absorption can be varied through an electric field applied over a multi-quantum well structures. Again, the change in absorption provides a phase relationship through the Kramers-Kronig relation and can provide modulation bandwidths above \SI{10}{GHz} \cite{Sibson2017InP, semenenko2019integrated, semenenko2019mdi, semenenko2019, smit2014}. The \ac{qcse} will be discussed in more detail in chapter \ref{chap:hom}.

\subsubsection*{Mach-Zehnder Interferometer}

\begin{figure}[t]
	\centering
	\includegraphics[width = 0.6\textwidth]{mzi.pdf}
	\caption[Mach-Zehnder interferometer schematic]{Illustration of \acs{mzi} made from two 50:50 beam splitters and a phase modulator.}
	\label{fig:mzi}
\end{figure}

Combining phase modulators and \acp{mmi}, we can create on-chip \acp{mzi} which can be used for very fast routing, intensity modulation and phase encoding. A schematic is shown in figure \ref{fig:mzi} where a phase modulation is applied to one arm of the \ac{mzi}.  We can calculate the transfer matrix by simply combing the matrices for phase modulation and {50:50} \ac{mmi} to get

\begin{equation}
	\frac{1}{\sqrt{2}}\left(
	\begin{matrix}
	1 & i\\
	i & 1
	\end{matrix}
	\right)
	\left(
	\begin{matrix}
	1 & 0\\
	0 & e^{i\theta}
	\end{matrix}
	\right)
	\frac{1}{\sqrt{2}}\left(
	\begin{matrix}
	1 & i\\
	i & 1
	\end{matrix}
	\right)=
	\frac{1}{2}\left(
	\begin{matrix}
	1 - e^{i\theta} & i(1 +  e^{i\theta})\\
	i(1 +  e^{i\theta}) & -1 +  e^{i\theta}
	\end{matrix}
	\right)
\end{equation}

When the phase modulator in the circuit has a high bandwidth, \acp{mzi} can be used for fast intensity modulation of the states, as we will see in chapter \ref{chap:hom}, but also more stable phase modulation, which is discussed further in chapter \ref{chap:mdiqkd}. In the ideal case, the phases accumulated in each arm of the \ac{mzi} will be equivalent. However, due to fabrication tolerances this is challenging to achieve. Therefore, \acp{mzi} combine slow phase modulation to correct for an offset with a fast phase modulator for high speed operation. This combination maximises the performance.

\subsubsection*{Single-photon Detection}

An important part of any quantum photonic information processing is detection of single-photon states. A number of different technologies are available for single-photon detection \cite{hadfield2009single}. Some metrics used to compare \acp{spd} are

\begin{itemize}
	\item \textbf{Efficiency:} the probability that the detector will fire given a photon was present.
	\item \textbf{Timing Jitter:} the uncertainty in the time of arrival of a photon.
	\item \textbf{Dead Time:} the time in which a detector cannot detect photons after firing.
	\item \textbf{Dark Counts:} the number of events when no photons are present.
\end{itemize}

\Acp{apd} have been widely used for quantum optics experiments with efficiency in the visible spectrum exceeding \SI{60}{\percent} \cite{hadfield2009single}. Dark counts for \acp{apd} can be relatively high with \si{kHz} rates which can be reduced with cooling. Jitter is typically O(\SI{100}{ps}). The detection efficiency for longer wavelengths, such as light used in telecommunications, is typically lower. More recently, there has been a demonstrated of a \SI{1}{GHz} gated \ac{apd} for \ac{QKD} with an efficiency of \SI{55}{\percent} \cite{comandar2015gigahertz}. There have been demonstrations of integrated Ge-on-Si \acp{apd} with efficiency up to \SI{35}{\percent} at \SI{125}{\kelvin} \cite{Martinez2017Single, vines2019high}. While the efficiency here is lower, there are obvious advantages in terms of scalability and reduced coupling losses.

% Within the C-band, InGaAs \acp{apd} have been demonstrated with \SI{55}{\percent} efficiency in a \SI{1}{GHz} gated mode \cite{comandar2015gigahertz}. On-chip \acp{apd} have been demonstrated \cite{}.

Advances in cryogenic cooling technology have allowed \acp{snspd} to become more widely available. Their uptake in quantum photonics experiments has been rapid due to their unmatched efficiency which can easily exceed \SI{90}{\percent} with timing jitter of O(\SI{10}{ps}) \cite{hadfield2009single}. Short recovery times can allow for high count rates and dark counts lower than 1 per hour \cite{wollman2017}. Waveguide integrated detectors have been demonstrated \cite{sprengers2011} which benefit from enhanced interaction between the confined light and the detector. Photonic cavities have been explored to further increase the detection efficiency and reduce the jitter \cite{vetter2016, yun2019, tyler2016modelling}.

%As \ac{QKD} will typically be concerned with single photon states (or close to) the efficiency of detection will be crucial. The losses associated with coupling on and off of integrated devices are a big problem. By fabricating detectors onto the waveguides we gain a few benefits.

%Moving the detectors on-chip removes the coupling losses and the high confinement in waveguides means a strong interaction between photons and the detectors. This also means that superconducting nanowires can be made shorter which reduces the jitter and dead times.

\subsection{Platforms}

Different materials will exhibit different properties that provide benefits and weaknesses when making \acp{pic}. Some different platforms will be introduced here with some of their favourable properties discussed.

\subsubsection*{Silicon-on-Insulator}

One of the more mature \ac{pic} platform is \ac{soi} which has leveraged fabrication from the ubiquitous silicon microelectronics industry. \Ac{cmos} compatibility has enabled large, complex devices to be fabricated \cite{wang2019integrated}. A high index contrast between the waveguide and the substrate allows tight confinement of the light meaning small bend radii are possible. 

As silicon is centro-symmetric, it has no natural $\chi^{(2)}$ which means that there is no electro-optic effect to use for phase modulation. Instead, modulation is achieved through thermo-optic, which are slow, or carrier injection/depletion modulators, which suffer from phase dependent losses. There have been attempts to break this centro-symmetry to achieve high bandwidth modulation \cite{cazzanelli2016second, castellan2019origin}.

A strong $\chi^{(3)}$ non-linearity allows single-photon generation through \ac{sfwm} \cite{SilverstoneThesis}. However, as silicon doesn't have a direct band gap lasers there is no immediate route to integrating lasers. Instead, hybrid platforms have emerge to utilise III-V lasers \cite{Fan2017, Agnesi2019}. Doped silicon has allowed fast photodiodes to be waveguide integrated with high bandwidths \cite{raffaelli2018generation}.

Light can be edge-coupled but grating structures are more common as they are not restricted to the edge of the device. Periodic structures in the silicon projects the waveguide mode verically into single-mode fibres. Losses have been shown to be as low as \SI{0.36}{dB} \cite{Notaros2016}, although the loss is typically higher. Silicon suffers from high non-linear losses which are suppressed at longer wavelengths \cite{rosenfeld2019mid}.

%Silicon is ubiquitous in the world today and powers everything from watches to supercomputers. The advances that have been made in the trillion dollar semiconductor industry have made silicon a hugely successful platform for quantum photonics. With a band gap of \SI{\sim 1100}{nm}, we can also leverage the telecommunications industry to create low-loss \acl{pic}. However, as it has an indirect band gap, it has been challenging to develop efficient light sources on-chip \cite{}. While efforts have been made to create integrated lasers and amplifiers, their performances don't match those of other integrated approaches, such as III-V materials.

%The high index contrast in a \ac{soi} device means that very compact devices with small bend radii are possible.

%Coupling can be done with grating couplers meaning that coupling light on and off chip isn't limited to the edges.

%It is not possible to realise \ac{eopm} in silicon as there is no $\chi^{(2)}$ non-linearity because of the centrosymmetry of the structure. High-speed modulation can be achieved through current injection but the phase dependent losses limit their performance \cite{} or the symmetry can be broken by introducing strain such as depositing a silicon nitride thin film.

%One of the major advantages of the silicon platform is the potential for integration for control electronics on the same device as the photonics. This could facilitate creating transmitter and receiver devices which have all of the photonic and electrical components for a key exchange. 

\subsubsection*{Silicon Nitride}

Silicon nitride (Si\textsubscript{3}N\textsubscript{4}) offers a wider band gap than silicon meaning that it has a much wider transparency. This has added benefits at mid-infra-red wavelengths as the non-linear losses, such as two-photon absorption \cite{tan2018nonlinear}, are reduced which is crucial for quantum photonic technologies \cite{lu2019chip}. Reconfigurable circuits can be achieved through themo-optic phase modulation and \acp{mmi}. Like silicon, there is no natural electro-optic effect so any fast modulation is achieved through carrier effects.

% which reduces non-linear losses such as two-photon absorption in the mid-IR spectrum. Phase modulation is achieved through thermo-optic effects which allows circuit reconfigurability.

\subsubsection*{Indium Phosphide}

\Ac{InP} offers benefits over other platforms, especially with its direct band gap which allows easily integrable lasers. \Acp{SOA} and \acp{DBR} structures allow Fabry-P\'{e}rot lasers to be created and directly coupled to waveguides. Alternatively, high-bandwidth \ac{dfb} components are also available \cite{smit2014, JeppixRoadmap}. Multi-quantum well structures allow fast phase modulation up to \SI{40}{GHz} through the \ac{qcse} with similar bandwidth photodiodes also available \cite{smit2014}. Thermo-optic effects are also available for stable modulation and tuning.

As there is no insulator, like in \ac{soi}, there is a very low index contrast to create grating couplers as the light from similar structures would launch into the substrate. Therefore, edge couplers are required to access the optical circuits. \Acp{ssc} are tapered waveguides that convert the waveguide to single-mode fibre. This low contrast also means that the bend radius is larger meaning that the component density is lower than silicon.

More recently, there has been work to replicate \ac{soi} with \ac{imos} \cite{IMOS, van2011photonic}. This increases the index contrast in the waveguides meaning that bend radii can be smaller \cite{Kumar2019}. In this first demonstration of \ac{sfwm} in \ac{InP} it was noted that the non-linearity is actually stronger than silicon. However, the non-linear losses through two-photon absorption is far larger. This could be solved by either increasing the band gap of the material, much like silicon nitride compared with silicon. Alternatively, longer wavelength photons would be unable to cause two-photon absorption \cite{rosenfeld2019mid}. 


%A less mature platform for quantum photonics comes from III-V materials which offer certain benefits for \acp{pic}, especially for communication purposes. Unlike silicon, \Ac{InP} has a direct band gap which means that lasers can be easily integrated into waveguides \cite{smit2014}.


%In this section, we will introduce the \acl{InP} platform and describe some of the components that we will find useful for the remainder of the thesis.

%\Ac{InP} has a higher non-linearity than silicon \cite{Kumar2019}. Index contrast is low, high bend radius. Improved with InP membrane.

%As \ac{InP} has a direct band gap it is possible to monolithically integrate lasers onto the devices. 

%Multi quantum well structures in \ac{InP} mean that we can create lasers on chip.

%We can make either \ac{DBR} or \ac{dfb} lasers in \ac{InP} with very competitive linewidths and mW power into fibre \cite{smit2014, jeppix}. An example spectrum is show in 

\subsubsection*{Lithium Niobate}

Lithium niobate is a favourite choice for phase and intensity modulation offering very high bandwidth modulation \cite{Louay2001Advances, Atsushi2010NRZ}. Single-photon generation is also available through \ac{spdc} in periodically poled structures \cite{tanzilli2002ppln}. Reconfigurable circuits have been demonstrated making it a potential candidate for future networks \cite{jin2014chip}. 

%\subsubsection{Hybrid Integration}

%More recently, there have been effects to hybridise \acp{pic} to create devices which exploit the benefits of multiple platforms. 

%For example, hybrids of an \ac{InP} laser and Si ring filters to make a laser device with a linewidth of \SI{290}{Hz} \cite{Fan2017}. 

%\ac{InP} lasers have also been directly coupled to silicon devices for \ac{QKD} transmission \cite{Agnesi2019}.

%While \ac{InP} offers many appealing optical properties, control through integrated electronics is desirable to create a scalable \ac{pic} platform. There are two ways that could create hybrid \ac{InP}-Si devices. 

%Heterogeneous integrated can be a achieved by wafer bonding an \ac{InP} \ac{pic} to either a silicon photonics chip or a \ac{cmos} control chip \cite{jeppix}. The bonded devices can then be further processed on a wafer scale.

%\Ac{imos} aims to introduce an \ac{InP} membrane onto a silicon wafer to provide efficient optic-electronic integration \cite{jeppix, IMOS}. The \ac{imos} platform has been demonstrated with some basis building blocks.

%There are also efforts to develop cross-platform electronic integration between III-V materials and Silicon \cite{}. 

\section{Summary}

In this background chapter, we have introduced important cryptographic techniques, quantum theory and integrated photonics that will form the foundation for the remainder of this thesis. The humble beginning of cryptography have been incredibly influential throughout history and remains a vital endeavour. With a better understand of cryptoanalysis techniques the field will need to keep evolving.

Advances in the understanding of quantum mechanics has allowed a more complete view of the world. It has also facilitated a new range of quantum technologies to complement their classical counterparts. Through quantum key distribution, we can develop entirely new ways to securely communicate. At the same time, thorough analysis will be crucial to their claims of security.

As with any technology, the mass-manufacturability of quantum photonic circuits will be key to its success. Developments in integrated platforms have allowed quantum photonics to progress from modest proof of principle experiments into the commercial world. Quantum communication systems are set to utilise these techniques to become part of future quantum networks. The rest of this thesis concerns developing the integrated platform for quantum key distribution.

%=========================================================