%
% File: chap04.tex
% Author: Henry Semenenko
% Description: Chip based MDI-QKD
%
% Set the graphics path to find figures
\graphicspath{{./chapters/chapter04/fig04/}}

\let\textcircled=\pgftextcircled
\chapter[Chip-Based Measurement-Device-Independent QKD]{Chip-Based Measurement-Device-Independent Quantum Key Distribution}
\label{chap:mdiqkd}

\section*{Introduction}

Advances in \ac{QKD} have seen many protocols developed for different purposes and security levels. The emerging quantum hacking community has exposed discrepancies between security theory and practical implementations. 

Technological advances require scalability. All currently available commercial \ac{QKD} systems are big, bulky and fibre-based. To encourage widespread adoption beyond commercial and military applications, an integrated platform it required to require manufacturing costs, as well as the size, weight and power requirements.

In this section, we will be experimentally demonstrating \ac{MDI} \cite{mdi-qkd} using integrated \ac{InP} transmitter devices. We will first outline the \ac{MDI} protocol before describing the experiment and showing our results. We will then discuss potential improvements to the experiment and further work.

%=======
\section{Measurement-Device-Independent Quantum Key Distribution}
\label{sec:mdi-qkd}

Unlike a traditional point-to-point key exchange, \ac{MDI} introduces third party (Charlie) who mediates the exchange by announcing correlations between pulses sent by Alice and Bob. Importantly, it improves security by removing all, current and future, side-channels on the detectors \cite{}. This is an important step as the detectors are often vulnerable to attacks due to their complexity \cite{}.

Secondly, it removes the asymmetry between Alice and Bob which allows for simpler metropolitan networks without reducing security with trusted nodes. This will be discussed further in chapter \ref{chap:node}.

\subsection{Protocol}

\begin{figure}[tbp]
	\centering
	\includegraphics[scale=0.25]{MDI-QKD_Protocol.png}
	\caption[MDI-QKD protocol]{Measurement-device-independent quantum key distribution protocol.}
	\label{fig:mdi_protocol}
\end{figure}

An outline of the protocol is shown in figure \ref{fig:mdi_protocol}. We will use a four decoy state protocol \cite{zhou2016} which treats the $Z$ and $X$ bases separately. The $Z$ basis is used entirely to generate keys, while the $X$ basis is used to bound knowledge gained by an eavesdropper. This is optimal for the symmetric case. The asymmetric case needs to be dealt with slightly differently \cite{wang2018}.

\begin{enumerate}
	\item \textbf{Preparation:} Alice and Bob independently and randomly choose \acp{wcs} from the BB84 states. 
	\item \textbf{Measurement:} Charlie performs joint Bell state projections on states from Alice and Bob
	\item \textbf{Announcement:}
	\item \textbf{Parameter Estimation:}
\end{enumerate}

\subsection{Security and Key Rates}

As we are not using a true single photon source, we need to be able to bound any knowledge gained by Eve through a photon number splitting attack. This is done through decoy-state \cite{}. 

Here we used a four intensity decoy state protocol where we consider four sources: $o, x, y, z$. The $o$ source is vacuum, $x$ and $y$ are decoy pulses in the $X$ basis with average photon numbers $\mu_x$ and $\mu_x$, respectively. Finally, the $z$ source emits pulses with intensity $\mu_z$ in the $Z$ basis. 

The estimate key rate of \ac{MDI} per pulse in the infinite case is given as

\begin{equation}
	R = (p_z)^2 p_{11}^Z Y_{11}^Z \left(1 - H_2(E_{11}^X)\right) - Q_{S,S}^Z f_e E_{S,S}^Z H_2(E_{\mu_s\mu_s}^Z)
\end{equation}
where $Y_{11}^Z$ and $e_{11}^X$ are the yield in the $Z$ basis and error in the $X$ basis, respectively, when Alice and Bob both sent single photons. These quantities are not directly measurable so we will need to bound them using a decoy state protocol. $Q_{S,S}^Z$ and $E_{S,S}^Z$ are the gain and error in the $Z$ basis and both can be measured directly. Finally, we introduce the error correction efficiency, $f>1$, and the binary entropy function defined as
\begin{equation}
	H_2(x) = -x\mathrm{log}_2(x) - (1-x)\mathrm{log}_2(1-x).
\end{equation}

Therefore, we define the upper bound of the secret key rate as

\begin{equation}
	\underline{R} \geq R
\end{equation}
for the lower and upper bounds of $Y_{11}^Z$ and $E_{11}^X$, respectively. These bounds can be found through the decoy state technique \cite{}. In essence, decoys states are measuring the loss of the channel during a key exchange to bound the knowledge that could be gain by an eavesdropper. 

By measuring different intensities in a decoy state protocol, we can measure

\begin{equation}
	Q_{\mu_a \mu_b}^Z = \sum_{n,m=0} e^{-(\mu_a + \mu_b)}\frac{\mu_a^n}{n!}\frac{\mu_b^m}{m!} Y_{n,m}^Z
\end{equation}

\begin{equation}
	Q_{\mu_{a} \mu_{b}}^X E^{X}_{\mu_{a} \mu_{b}}=\sum_{n, m=0} e^{-\left(\mu_{\mathrm{a}}+\mu_{\mathrm{b}}\right)} \frac{q_{\mathrm{a}}^{n}}{n !} \frac{q_{\mathrm{b}}^{m}}{m !} Y^{X}_{n, m} E^{X}_{n, m}
\end{equation}

where $\mu_a$ ($\mu_b$) are the intensities of Alice's (Bob's) pulses.

\begin{equation}
	\underline{Y_{11}^Z} = \frac{1}{\left(\mu_{\mathrm{a}}-\omega_{\mathrm{a}}\right)\left(\mu_{\mathrm{b}}-\omega_{\mathrm{b}}\right)\left(v_{\mathrm{a}}-\omega_{\mathrm{a}}\right)\left(\nu_{\mathrm{b}}-\omega_{\mathrm{b}}\right)\left(\mu_{\mathrm{a}}-v_{\mathrm{a}}\right)}
\end{equation}

\begin{equation}
	\overline{E_{11}^Z} = \frac{1}{\left(v_{\mathrm{a}}-\omega_{\mathrm{a}}\right)\left(v_{\mathrm{b}}-\omega_{\mathrm{b}}\right) \underline{Y^{X}_{1,1}}}
\end{equation}

We can use the decoy sates in $X$ to bound the single photon events in $Z$ \cite{zhou2016}.

\subsection{Shared Resources}

\section{Integrated Transmitters}

\subsection{Pulse Carving}

\subsection{Phase Encoding}

\subsection{Phase Randomisation}

\subsection{Decoy State Preparation}

\section{Receiver}

\subsection{Fibre components}

\subsection{Superconducting Detectors}

\section{Results}



\subsection{Calibration}

As mentioned before, the \ac{MDI} protocol is very dependent on high fidelity \ac{HOM} interference. Therefore, we must make sure that the two different transmitters have a good overlap in all degrees of freedom. In this section we will discuss method used to create indistinguishable pulses.

\begin{figure}[tbp]
	\centering
	\includegraphics[scale=0.25]{BSM_Curr_Sweep/Fit.pdf}
	\caption[Laser current-error sweep]{Sweeping the current on one laser changes the wavelength relative to the other. This changes the relative phases between the two pulses in the +/- states. Using this allows good calibration of wavelength.}
	\label{fig:wavelength_cal}
\end{figure}


\subsection{Phase Randomised Transmitters}

\subsection{2GHz Clock Rate}

\section{Outlook}

\subsection{Finite Key Effects}

As this demonstration was focussed towards the technology, we have not included analysis of finite key effects. The optimal method of including finite key effects is given in Ref. \cite{zhou2016}.

Statistical fluctuations in the measurements mean that the values used to estimate the single photon cases are not infinitely accurate. 

\subsection{Improved Transmitters}

There are many ways of improving the QKD transmitters. For example, encoding in time-bin using a delay line to separate the pulses or generating different states using independent lasers.

\begin{sidewaysfigure}
	\centering
	\includegraphics[width=0.8\textwidth]{HHI_Transmitter}
	\caption[Latest generation InP QKD Transmitter]{Latest generation HHI indium phosphide transmitter. The \SI[product-units=power]{6x4}{mm} chip contains a few ways to create BB84 states for QKD. Firstly, we have designs to compare \ac{dfb} and \ac{DBR} lasers. Secondly, we can use a delay line to separate the time bins. Finally, we have multiplexed lasers to pulse independently lasers for each state.}
\end{sidewaysfigure}

\subsection{Security of Transmitters}

The security of the transmitters is still important \cite{makarov2019}.

\subsection{Multiplexing}

\subsection{Miniaturised Electronics}


%=========================================================