%
% File: chap06.tex
% Author: Henry Semenenko
% Description: Laser Seeding
%
% Set the graphics path to find figures
\graphicspath{{./chapters/chapter06/fig06/}}

\let\textcircled=\pgftextcircled
\chapter{Next Generation Integrated Transmitters}
\label{chap:future}
\glsresetall

\section*{Statement of Work}

Philip Sibson and I designed the photonic chips and compiled the chip mask that was fabricated by Fraunhofer HHI. I design the PCB for electrical operation and optical connection. I packaged the chip with support from Alasdair Price and Graham Marshall. I performed charaterisation of the initial parts of the circuits which included laser driving under both DC and RF operation and a preliminary test of laser seeding capability.

%=======
\section{Introduction}

The transmitters that were used to demonstrate \ac{MDI} could be improved. Here, we will outline the next generation devices that have been designed and fabricated. 

\section{Improved Transmitters}

There are many ways of improving the \ac{QKD} transmitters. For example, encoding in time-bin using a delay line to separate the pulses or generating different states using independent lasers.

\begin{sidewaysfigure}
	\centering
	\includegraphics[width=0.8\textwidth]{QKD_transmitter}
	\caption[Latest generation InP QKD Transmitter]{Latest generation HHI indium phosphide transmitter. The \SI[product-units=power]{6x4}{mm} chip contains a few ways to create BB84 states for QKD. Firstly, we have designs to compare \ac{dfb} and \ac{DBR} lasers. Secondly, we can use a delay line to separate the time bins. Finally, we have multiplexed lasers to pulse independently lasers for each state.}
\end{sidewaysfigure}

\subsubsection*{Multiple Laser Sources}

We introduce multiple laser sources into the circuit which can each be independently gain-switched to generate the four BB84 states. 

\subsubsection*{Delay Line Time-Bin Encoding}

While the previous generation of devices included delay lines to encode timing information, they we too lossy to be used. We have included a delay line in the design of this chip which should have lower losses. This should reduce the complexity of the control electronics and make the system compatible with a gain-switch DFB laser.

\section{Laser Seeding}

Laser seeding provides a method of generating \ac{QKD} states by overcoming some of the issues with gain switching lasers while also reducing certain electrical requirements.. This has been performed using fibre lasers \cite{}. In this chapter, we will demonstrate the first integrated laser seeding transmitter for time-bin encoding \ac{QKD}. 

\subsection{Integrated Laser Seeded Transmitter}

There are no circulators or isolators currently available on-chip, so other methods of laser seeding need to be considered.

Figure \ref{fig:hhi_laser_seeding} shows the GDS for the integrated device. The complexity of integrated devices is 

\begin{sidewaysfigure}
	\centering
	\includegraphics[width=0.9\textwidth]{Laser_Seeding_gds.png}
	\caption[InP laser seeding transmitter with QRNG]{This shows the layout of the laser seeded transmitter device fabricated by HHI. The chip measures \SI[product-units=power]{6x4}{mm} and contains two laser seeding prototype circuits, a homodyne \ac{qrng} and test structure to measure laser and waveguide performances. This demonstrates the complexity and compactness possible in the integrated platform.}
	\label{fig:hhi_laser_seeding}
\end{sidewaysfigure}

\subsection{Optical and Electrical Packaging}

\subsubsection*{PCB Design}

\begin{figure}[tbp]
	\centering
	\includegraphics[width=\textwidth]{HHI_PCB/HHI_PCB_render_2.png}
	\caption[PCB breakout for an InP integrated circuit]{PCB designed to connect an \ac{InP} integrated device to control and test equipment. Packaging integrated devices remains challenging due to the small size of the chips. This is especially true when high speed connections are required.}
	\label{fig:hhi_pcb}
\end{figure}

The electrical connections for integrated devices is challenging, especially in test devices where real estate is valuable. In figure \ref{fig:hhi_pcb} we show a PCB which is required to connect most (but not all) components of the chip shown in figure \ref{fig:hhi_laser_seeding}. 

The chip was angled at \SI{7}{\degree} so that the spot-size converters were perpendicular to the side of the PCB. The \ac{vga} could then be polished at a \SI{7}{\degree} in the opposite direction to compensate and maximise coupling.

A section of the PCB was partial milled out the thickness of the chip so that the top of the chip and top of the PCB would be flush. This makes wirebonding the chip easier and allows wirebond length to be minimised. Thermal vias were drilled through the chip and plated for thermal management and stabilisation. 

Again, the PCB substrate was Rogers 6006 and tracks designed for \SI{50}{\ohm} impedance. RF lines were shield with vias to ground to increase the performance for high-speed operation. Due to the number of connections required for the chip, DC connections near the chip were routed through plated and capped vias. These were then routed to an FFC connector. In total, 33 wirebonds were required for this test.

The PIN photodiode was given a separate ground from the rest of the chip to electrically isolate it for testing.

A section of the PCB was cut out to allow for a \ac{vga} to be held during the glue process described later. A section towards the edge of the chip was also milled out \SI{1}{mm} in depth so that the fibres from the \ac{vga} could be glued to the PCB for strain relief.

Pads for a thermistor were placed as close to the chip as possible with via routing to the bottom of the chip to provide good thermal conductivity.

\subsubsection*{Chip Mounting}

There is no need to electrically connect the bottom of the chip to ground, as was with the case with devices fabricated from Oclaro. Silver epoxy was still used for its good thermal conductivity. The plated vias provided a good thermal connection to the bottom of the PCB so that a peltier could be used for thermal stabilisation using feedback from the thermistor.

\subsubsection*{Optical Connection}

The \ac{vga} was held using a vacuum chuck and moved using a 6-axis micro-positioner (NanoMax). The on-chip lasers provided a light source for alignment. 

UV cured index matched glue was used to attach the \ac{vga} to the chip. The make and model was Dymax OP-4-20632.

The \ac{vga} was aligned using the on-chip laser. Then a small amount of glue could be applied along the connection. The \ac{vga} was then pressed against the chip. A UV LED lamp illuminated the area to allow for drying. This was left for 12 hours to cure. Silicon was applied to the fibres 

A photograph of the packaged chip under the steriliser is shown in figure \ref{fig:hhi_photo}. The fibre strain relief is shown in the left with silicone to secure the fibre to the PCB. 

\begin{figure}
	\centering
	\includegraphics[width = 0.8\textwidth]{hhi_packaged}
	\caption[Photograph of packaged HHI transmitter]{The \ac{InP} device is glued to the centre of the PCB with the \ac{vga} glued to the edge couplers. Fibre strain relief is shown to the left were the bare fibres from the \ac{vga} are attached to the PCB.}
	\label{fig:hhi_photo}
\end{figure}

\subsection{Laser Operation}

A current sweep of the laser \ac{SOA} was performed to see threshold and power characteristics. Two different lasers were tested on chip with \num{300} and \SI{500}{\um} \ac{SOA} lengths to see the different in performance. The rear \ac{DBR} and phase modulator were \SI{300}{\um} and \SI{50}{\um}, respectively, for both lasers. However, the front \ac{DBR} was only \SI{50}{\um} for the \SI{500}{\um} laser, while the other was \SI{150}{\um}.

The sweeps are show in figure \ref{fig:hhi_soa_sweep}. Both lasers show a lasing threshold of around \SI{30}{\mA}. Fluctuations in power at higher driving current is likely attributed to mode competition. At very high currents, the \SI{300}{\um} \ac{SOA} laser starts to decrease in power. This will be due to the efficiency of the laser diode at higher temperatures.

\begin{figure}[tp]
	\centering
	\small	
	\def\svgwidth{0.9\textwidth} 
	\import{chapters/chapter06/fig06/I_sweep/}{current_sweep.pdf_tex}
	\caption[Current sweep of HHI laser SOA]{Sweeping laser current. Threshold of \SI{30}{\mA} for both. Fluctuations in power likely due to the mode competition in the cavity. Reduction in power for \SI{300}{mA} attributed to heating of the diode reducing efficiency.}
	\label{fig:hhi_soa_sweep}
\end{figure}

Laser spectrum shows multi-modal operation. Sidebands can be suppressed through current injection of the tunable \acp{DBR}. However, they are still only around \SIrange{30}{40}{dB} below the peak. 

\begin{figure}[tp]
	\centering
	\small	
	\def\svgwidth{0.9\textwidth} 
	\import{chapters/chapter06/fig06/Spectra/}{spectra.pdf_tex}
	\caption[Wavelength sweep through current injection of tunable DBRs]{Current injection of the tunable \acp{DBR} causing a change of peak reflected wavelength. The lasers show multi-modal operation and a sideband suppression of only \SI{35}{dB}.}
	\label{fig:hhi_dbr_sweep}
\end{figure}

The wavelength can be tuned but around \SI{3}{nm} by current injection of the tunable \acp{DBR}. Typical values are less than \SI{2}{V}. High driving current can cause heating of the cavity causing the wavelength to increase. 

The transmitter (shown in figure \ref{fig:hhi_laser_seeding}) contains two methods possible to achieve laser seeding.

\subsection{Gain Switching}

We can characterise the on-chip laser ability to gain switch, which is important for phase randomisation in \ac{QKD}. 

\begin{figure}[tp]
	\centering
	\small	
	\def\svgwidth{\textwidth} 
	\import{chapters/chapter06/fig06/laser_gain_switch/}{gain_switch_test.pdf_tex}
	\caption[Gain switching test of the HHI on-chip lasers]{\SI{20}{mA} (left), \SI{30}{\mA} (middle), \SI{40}{\mA} (right). \SI{1}{GHz} \SI{2}{\Vpp} square wave from \ac{ppg}. Laser threshold around \SI{30}{mA}. Pulses are around \SI{120}{ps} \ac{FWHM}.}
	\label{fig:gain_switch_test}
\end{figure}

\begin{figure}[tp]
	\centering
	\small	
	\def\svgwidth{0.9\textwidth} 
	\import{chapters/chapter06/fig06/Spectra/}{spectrum_gain_switch.pdf_tex}
	\caption[Spectrum of the gain switched laser]{Spectral broadening due to the laser gain switching. \SI{30}{mA} current, \SI{2}{\Vpp} \SI{1}{\GHz} square wave.}
	\label{fig:gain_switch_spectrum}
\end{figure}

\subsection{Laser Seeding}


\begin{figure}[tp]
	\centering	
	\def\svgwidth{0.9\textwidth} 
	\import{chapters/chapter06/fig06/schematics/}{laser_seeding.pdf_tex}
	\caption[Schematic of on-chip laser seeding]{Schematic of on-chip laser seeding. As there is no non-linearity to exploit for a circulator, the usual method of laser seeding cannot be achieved in integrated photonics. These are the simplest conceivable circuits to attempted it in an integrated platform}
	\label{fig:las_seed_schem}
\end{figure}


\begin{figure}[tp]
	\centering	
	\def\svgwidth{0.8\textwidth} 
	\import{chapters/chapter06/fig06/}{laser_seeding_test.pdf_tex}
	\caption[Integrated laser seeding test]{By applying \SI{1}{GHz} square waves to both lasers, we can get pulses that should be phase randomised. However,  this will need testing. Pulses show \SI{120}{ps} \ac{FWHM}.}
	\label{fig:las_seed_test}
\end{figure}

\subsection{Quantum Random Number Generation}

Random numbers are required for many parts of \ac{QKD}. One way of generating this randomness is with \acp{qrng} which has been demonstrated on \ac{InP} \cite{FrancescoThesis,Abellan2016}. However, there has not yet been a single device which fulfils both the requirements for quantum random number generation \textit{and} quantum key generation.

The chip shown in figure \ref{fig:hhi_laser_seeding} contains both a \ac{qrng} and a laser seeded \ac{QKD} transmitter. Together with electronics to directly read the randomness and generate \ac{QKD} states this could been a fully packaged system.

\section{Outlook}

This chapter has extended the use of integrated devices for \acl{QKD}.

%=========================================================