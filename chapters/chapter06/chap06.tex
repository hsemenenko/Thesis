% File: chap06.tex
% Author: Henry Semenenko
% Description: Laser Seeding
%
% Set the graphics path to find figures
\graphicspath{{./chapters/chapter06/fig06/}}

\let\textcircled=\pgftextcircled
\chapter{Next Generation Integrated Transmitters}
\label{chap:future}

\section*{Statement of Work}

Philip Sibson and I designed the photonic chips and compiled the chip mask that was fabricated by Fraunhofer HHI. I design the PCB for electrical operation and optical connection. I packaged the chip with support from Alasdair Price and Graham Marshall. I performed characterisation of the initial components of the circuits which included laser driving under both DC and RF operation and a preliminary test of laser seeding capability.

%=======
\section{Introduction}

Integrated photonics has facilitated a drastic increase in the complexity possible for quantum information experiments \cite{wang2019integrated}. The inherent phase stability allows a level of control on a scale simply not possible with fibre or bulk alternatives. However, the technology is still under heavy development to improve the performances of devices at both component and circuit levels. As such, further optimisation is required before commercial deployment of integrated devices is viable. 

While the demonstrations in previous chapters showed state-of-the-art performances from \ac{InP} devices, there were several parts which are in need of revision. For example, tuning of \acp{mzi} requires \ac{topm} and delay line losses were too high to be used. 

In this chapter, we will introduce the next generation designs for \ac{InP} \ac{QKD} transmitters. By learning from the lessons from the first generation devices, we can introduce more components to the circuit for more fine tuning. Developments in processing have reduced the losses from components and waveguides meaning that optical delay lines become more achievable. Waveguides losses can be as low as \SI{1}{dB\per\cm}

%The transmitters that were used to demonstrate \ac{MDI} could be improved. Here, we will outline the next generation devices that have been designed and fabricated. 

\begin{figure}
	\centering
	\includegraphics[width=0.9\textwidth]{./schematics/components.pdf}
	\caption[Components used for chip schematics]{List of components used for the chip schematics: \acf{dfb} laser, \acf{mmi}, \acf{DBR}, \acf{cipm}, \acf{SOA}, \acf{pd}, \acf{topm}.}
	\label{fig:components}
\end{figure}

\section{Pulsed Laser Seeding}

\Ac{pls} is an optical technique that is based on optical-injection locking of lasers which can be used to create narrow pulses with low timing jitter \cite{Seo1996, Gunning1996}. An initial (master) laser runs in a \ac{CW} mode which is injected into a second (slave) laser. The slave is electrically pulsed which creates a narrow pulse from the injected photons and inherits the phase of the master. However, this would still require phase randomisation for \ac{QKD} as the phase from the initial laser would only slow drift. 

Laser seeding provides a method of generating \ac{QKD} states by overcoming some of the issues with gain switching lasers while also reducing certain electrical requirements.. This has been performed using fibre lasers \cite{Comandar2016PLS, Comandar2016}. In this section, we will present two designs for integrated laser seeding transmitter for time-bin encoding \ac{QKD}. Initial characterisation of the devices was performed but further work is required to 

Since this chip had been designed, similar work was published demonstrating an integrated laser-seeding transmitter based on the \ac{InP} platform \cite{paraiso2019}. 

Laser seeing using two pulsed lasers: a master and a slave. Both are driving with electrical square waves. The master laser is kept just below threshold and gain switching causing a laser pulse from a spontaneous emission. The DC offset is such that the timing jitter is minimised (although still large) while maintaining a randomised phase. This initial pulse is sent through a circulator to the slave laser The slave has a DC offset below lasing threshold such that without the master laser pulse, no lasing can occur even with the RF signal.

\subsection{Integrated Laser Seeded Transmitter}

There are no circulators or isolators currently available on-chip, so other methods of laser seeding need to be considered.

Figure \ref{fig:hhi_laser_seeding} shows the GDS for the integrated device. The complexity of integrated devices is obvious. Shared grounded needed to be implemented to reduce the number of electrical wires on the chip.

\begin{sidewaysfigure}
	\centering
	\includegraphics[width=0.9\textwidth]{Laser_Seeding_gds.png}
	\caption[InP laser seeding transmitter with QRNG]{This shows the layout of the laser seeded transmitter device fabricated by HHI. The chip measures \SI[product-units=power]{6x4}{mm} and contains two laser seeding prototype circuits, a homodyne \ac{qrng} and test structure to measure laser and waveguide performances. This demonstrates the complexity and compactness possible in the integrated platform.}
	\label{fig:hhi_laser_seeding}
\end{sidewaysfigure}

The schematic for the on-chip laser seeding test is shown in figure \ref{fig:las_seed_schem} where master and slave lasers can interact.

Back reflections are a potential issue for this design which could introduce challenges achieving phase randomisation.

\begin{figure}[tp]
	\centering
	\includegraphics[width=0.9\textwidth]{./schematics/laser_seeding.pdf}
	\caption[Schematic of on-chip laser seeding]{Schematic of on-chip laser seeding. As there is no non-linearity to exploit for a circulator, the usual method of laser seeding cannot be achieved in integrated photonics. These are the simplest conceivable circuits to attempted it in an integrated platform}
	\label{fig:las_seed_schem}
\end{figure}

\subsection{Optical and Electrical Packaging}

The complexity of the device required careful consideration to ensure optical and electrical packaging for high-speed operation. This section will describe the design choices to perform the initial tests.

\subsubsection*{PCB Design}

The electrical connections for integrated devices is challenging, especially in test devices where on-chip real estate is valuable. In figure \ref{fig:hhi_pcb} we show the PCB designed to connect most (but not all) components of the chip shown in figure \ref{fig:hhi_laser_seeding}. For this initial test, the \ac{qrng} section of the chip was not connected. In total, 33 electrical lines were included in the design for the initial testing.

The PCB was again made from Rogers 6006 for its high dielectric constant and contained four layers. The top layer of the PCB is electroless nickel immersion gold plated for the wirebond connections. The first two layers were dedicated to the RF tracks. The top layer provided signal lines through \ac{cpw} while the second layer provided isolation from the DC tracks below. The \ac{cpw} dimensions can be calculated using equations presented in chapter \ref{chap:hom}. The high dielectric constant allowed narrow \acp{cpw} while maintaining a \SI{50}{\ohm} impedance. 

To improve the performance over previous designs, a via fence was included along the RF tracks to create a \ac{gcpw} which should increase the bandwidth \cite{haydl2002use}. This via fence reflects the signal and stops a surface wave being created along the track. The spacing between the vias is important to maximise bandwidth, as is the spacing between the track and via. Generally speaking, a microwaves will reflect off of gaps which are less than \nicefrac{1}{4} of their wavelength \cite{Sain2016}. Therefore, the spacing between each via can be calculated as 

\begin{equation}
	S_\text{via} = \frac{\lambda}{4} = \frac{c}{4 f \sqrt{E_r}}
\end{equation}
where $f$ is the maximum frequency of operation and $E_r$ is the dielectric constant of the substrate. For this PCB design, the vias were placed \SI{500}{\um} from the track and spaced \SI{1}{\mm} apart meaning the \ac{gcpw} should have a bandwidth up to \SI{25}{GHz}. SMA edge-launch connectors rated to \SI{26.5}{GHz} were soldered to the tracks and grounds for high speed signals.

\begin{figure}[tbp]
	\centering
	\includegraphics[width=\textwidth]{HHI_PCB/HHI_PCB_render_2.png}
	\caption[PCB breakout for an InP integrated circuit]{PCB designed to connect the HHI device to control and measurement equipment. The chip sits in the middle of the PCB so that RF track length can be minimised. RF lines are shielded to increase bandwith and are connected to edge-launch SMA connectors. DC lines are routed through a lower layer of the PCB and connected to an FFC receptacle. A section of the PCB is milled out to allow a \ac{vga} to be mounted for optical connection. Screw holes allow the package to be mounted to a heat sink for thermal management.}
	\label{fig:hhi_pcb}
\end{figure}

The DC tracks were routed through the third and fourth layers of the PCB. To save space on the top layer of the PCB, the bond pads for the DC connections were capped vias that had been previously plated and plugged. This allow the connections to be immediately routed to a lower layer. These were then routed to an FFC connector on the side of the PCB.

The chip contains a PIN \ac{pd} for test purposes. To isolate this from the rest of the circuit, a separate ground was created so that it could be tested separately from the remained of the circuit. A \ac{gcpw} was used to connect this where both ground pads on the chip could be connected to the grounds of the G-S-G line.

%Due to the number of connections required for the chip, DC connections near the chip were routed through plated and capped vias. These were then routed to an FFC connector. 

The chip was placed in the center of the PCB and the side was curved to minimise the RF track lengths required. This section of the PCB was partial milled out the width, length and thickness of the chip so that the top of the chip would be flush with the top of the PCB. This makes wirebonding the chip easier and allows wirebond length to be minimised. This section was plated for thermal conductivity and plated vias added to the bottom of the PCB for thermal management and temperature stabilisation. 

As in the previous chips, the \acp{ssc} were angled at \SI{7}{\degree} from the edge of the chip. This is to reduce the back reflections into the chip. To increase the coupling from the chip to the \ac{vga}, the chip was rotated by \SI{7}{\degree} in the PCB design. The \ac{vga} could then be polished at a \SI{7}{\degree} in the opposite direction and be aligned parallel to the PCB. Unfortunately, the geometry of this design was not fully considered meaning that once rotated, the \ac{vga} was no longer \SI{500}{\um} \textit{vertically} spaced. This is shown in figure \ref{fig:geometry_lesson}. When the first fibre is aligned to the chip, the subsequent fibres no longer line up with the chip facets. Future designs should consider this in the compilation of the chip mask. By adjusting the spacing between \acp{ssc} on the chip, a standard polished \ac{vga} can be used to connect all the optical ports. Alternatively, one can sacrifice coupling efficiency by leaving the chip parallel with the chip and using a standard flat-faced \ac{vga}.

%The chip was angled at \SI{7}{\degree} so that the \acp{ssc} were perpendicular to the side of the PCB. Unfortunately, as shown in figure \ref{fig:geometry_lesson}, this meant that the \ac{vga} no longer lined up with the output ports of the chip. 

One should also note that \SI{7}{\degree} is not the optimal angle for coupling from the chip to the \ac{vga}. The refractive index of the \ac{InP} waveguides is 3.26 whereas standard optical fibre has a refractive index of around 1.44. Therefore, through Snell's law, the optimal angle for coupling from an \ac{ssc} to fibre would be \SI{16}{\degree}. 

A section at the edge of the chip was milled out to provide a space for the \ac{vga} to be held and glued onto the chip. The cutout was designed to allow the chip to hang over the side by about \SI{1}{\mm} to ease gluing. A section at the edge of the PCB was also partially milled out to a depth of around \SI{0.5}{\mm}. This was to allow the bare fibres to remain horizontal and not change the pitch of the \ac{vga} when aligned to the chip. It also provides a place for the fibres to be fixed to the PCB for strain relief.

Solder pads were placed as close to the chip as possible for a thermistor to be used for thermal stabilisation. Tracks were routed to the bottom of the PCB to provide good thermal connectivity to the chip and peltier. Screw holes were provided to allow the device to be connected to a heat sink and were surrounded by vias to shield the PCB from RF radiation.

\begin{figure}[t]
	\centering
	\includegraphics[width=\textwidth]{geometry_lesson.pdf}
	\caption[A quick lesson in geometry for edge coupling]{The chip facets are designed as \SI{500}{\um} spaced so as to be compatible with standard \acp{vga}. As previously mentioned, the facets are at a \SI{7}{\degree} angle to the edge of the chip to reduce back reflections into the chip. To maximise coupling, the \ac{vga} chosen was polished to match this angle. However, once rotated, the vertical spacing of the fibres in the \ac{vga} are no longer \SI{500}{\um} spaced. this meant only one fibre could be connected to the device.}
	\label{fig:geometry_lesson}
\end{figure}

\subsubsection*{Chip Mounting}

Unlike the device from Oclaro, where the bottom of the device was the ground plane, there is no need to electrically connect the bottom of the devices from HHI. However, silver epoxy was still used for its strong adhesive properties and good thermal conductivity to aid with temperature stabilising the devices. A section of the PCB was partially milled out meaning that placing the chip was more challenging as the sides of the chip would be below the top of the PCB.

To accurately place the chip onto the PCB, vacuum tweezers were used. A silicone cup was used to pick up the chip from the top which could be released when the chip was in position. Silver epoxy was spread onto the milled out section where the chip was then placed. The chip was allowed to overhang the PCB by \SI{0.5}{\mm} to ease \ac{vga} connection. The PCB was then heated to \SI{100}{\celsius} for \SI{30}{\minute} to allow the silver epoxy to cure.

%A section for the chip was milled out so that the top of the device would sit flush with the top of the PCB. The was to ease wirebonding and minimise the distance between chip and PCB bond bads. This milled out section was plated and contained four plated vias to the bottom layer of the PCB. This was to bypass the thermal isolation of the substrate and provide a good thermal contact between the chip and the peltier that will be used for thermal management.

\subsubsection*{Wirebonding}

\begin{figure}[t]
	\centering
	\includegraphics[width = 0.8\textwidth]{hhi_packaged}
	\caption[Photograph of packaged HHI transmitter]{Photograph of the packaged HHI device. The \ac{InP} chip is glued to the centre of the PCB with the \ac{vga} glued to the edge couplers. Fibre strain relief is shown to the left were the bare fibres from the \ac{vga} are attached to the PCB. The electrical pads on the chip have been wirebonded to the PCB for electrical control.}
	\label{fig:hhi_photo}
\end{figure}

\SI{25}{\um} gold wire is used to connect the chip bond pads to the PCB which heated to \SI{100}{\celsius}. A high-voltage electrode provides a spark to the end of the wire creating a \SI{50}{\um} ball at the end of a capillary. The capillary is positioned above the chip bond pad connected to the chip bond pad through thermosonic bonding, which using a combination of heat, pressure and ultrasonic energy to form a bond. The wire is then fed through the capillary to form a loop so the wire can be connected to the PCB pad with a wedge bond, again using thermosonic bonding. This method can be used to create connections to devices with bandwidths above \SI{60}{GHz} \cite{chen2015bandwidth}, while coax wirebonds have been developed to support bandwidths above \SI{100}{GHz} \cite{cahill2006development}.

\subsubsection*{Optical Connection}

The device contains eight \acp{ssc} on the side of the chip which are designed to convert from the the \SI{2.5}{\um} waveguide mode to a \SI{10}{\um} mode that is more circular. This means that we don't need to use lensed fibres and can connect the \acp{ssc} directly to SMF fibre. The \acp{ssc} are \SI{500}{\um} spaced so as to be compatible with the standard \acs{vga} which was provided by Oz Optics for this device. 

A brass vacuum chuck was designed from the dimensions of the \ac{vga} so that it could be securely held during alignment and gluing. The chuck was placed on a 6-axis micro-positioner (NanoMax) so that it could be well aligned with the orientation of the chip. 

The integrated lasers were used to align the \ac{vga} to the chip. A stable current source provided each laser with \SI{75}{\mA}.The device had been designed with lasers at both the top and bottom of the chip to allow the roll of the \ac{vga} to match that of the chip. As mentioned before, and show in figure \ref{fig:geometry_lesson}, it was not possible to align all of the fibres in the \ac{vga} with the \acp{ssc}. Therefore, the \ac{ssc} with the most test structures was chosen to be the only fibre aligned. 

%was held using a vacuum chuck and moved using a . The on-chip lasers provided a light source for alignment. 

Once the fibre was sufficiently aligned, the \ac{vga} was moved back from the chip to allow glue to be applied between the \ac{vga} and the chip. Index matched, UV cured glue (Dymax OP-4-20632) was used to attach the fibres and the chip. Only a small gap between the \ac{vga} and chip was required due to the low viscosity of the glue allowing it to permeate between them. 

Once the glue had been applied, the \ac{vga} moved back to be in contact with the chip. A UV LED lamp illuminated the area to cure the glue which was left for 12 hours. Silicone glue was then applied to the bar fibres at the edge of the PCB for strain relief and left to dry. Once the silicone had dried, the vacuum was turned off and the brass chuck could be lowered away from the device. To ensure that the UV cured glue was completely cured, the entire device was placed into a UV steriliser and left for a further 12 hours. 

A photograph of the packaged chip under the steriliser is shown in figure \ref{fig:hhi_photo}. The HHI chip sits in the middle of the device and is glued and wirebonded to the PCB. The \ac{vga} is attached with UV cured glue and the bare fibres are attached to the PCB with silicone glue to provide strain relief. Electrical connections can be made through the SMA and FFC connections on the edge of the PCB.

The packaged device was screwed to a heat sink with nylon screws to avoid thermal loops. A peltier placed directly under the chip. An Arroyo 6305 was connected to both the thermistor and peliter and a PID loop was calibrated to stabilise the temperature. A case was placed over the PCB to reduce air flow over the device which can cause thermal instability. The temperature was set to \SI{25}{\celsius} and the PID loop maintained a temperature instability of less than \SI{0.01}{\celsius}. The device could then be electrically and optically tested to characterise the performance.

\subsection{Laser Operation}

\begin{figure}[t]
	\centering
	\small	
	\def\svgwidth{\textwidth} 
	\import{chapters/chapter06/fig06/I_sweep/}{current_sweep.pdf_tex}
	\caption[Current sweep of the HHI Fabry-P\'{e}rot test lasers]{Current sweep of the on-chip, Fabry-P\'{e}rot test lasers. The two designs have different \ac{SOA} and front \ac{DBR} lengths. Both show a lasing threshold of around \SI{30}{\mA}. We find that the longer \ac{SOA} provides considerably more power despite the shorter front \ac{DBR}. Fluctuations in power likely due to the mode competition in the cavity. Reduction in power for \SI{300}{\um} \ac{SOA} are likely attributed to heating of the diode reducing efficiency. The effective is elastic so reducing the current recovers the original efficiency.}
	\label{fig:hhi_soa_sweep}
\end{figure}

This device contains two Fabry-P\'{e}rot laser test structures to assess their characteristics. A cavity is formed from two current-injection tunable \acp{DBR} which are of different lengths. The rear \ac{DBR} is \SI{300}{\um} in length which should provide \SI{95}{\percent} reflectivity. Two different front \ac{DBR} lengths were chosen, as well as different lengths for the \ac{SOA}. The first laser had a front \ac{DBR} length of \SI{150}{\um} with an \ac{SOA} length of \SI{300}{\um}. The second laser had a front \ac{DBR} length of \SI{50}{\um} and \ac{SOA} length of \SI{500}{\um}. The reduction in \ac{DBR} length from \SI{150}{\um} to \SI{50}{\um} is estimated by the foundry to reduce the reflectivity from \SI{75}{\percent} to \SI{25}{\percent}. Both cavities also contain a \ac{cipm} which can be used to precisely tune the wavelength of the laser. This method is preferred over current injection of the \ac{SOA} as it won't not affect the output power of the laser.

Two different lasers designs were tested on chip with \num{300} and \SI{500}{\um} \ac{SOA} lengths to see the difference in performance. The rear \ac{DBR} and phase modulator were \SI{300}{\um} and \SI{50}{\um}, respectively, for both lasers. However, the front \ac{DBR} was only \SI{50}{\um} for the \SI{500}{\um} laser, while the other was \SI{150}{\um}.

A current sweep of the laser \ac{SOA} was performed to see threshold and power characteristics which are shown in figure \ref{fig:hhi_soa_sweep}, together with schematics of the two laser designs. Both lasers show a lasing threshold of around \SI{30}{\mA} and gain is initially linear. At high driving currents, both lasers exhibit fluctuations in power. This is likely attributed to mode competition within the laser cavity caused by cavity heating. The increased \ac{SOA} length allows the laser to reach around three times higher power for the same driving current. This will be partially due to the longer gain medium allowing a higher amplification and also the reduced reflectivity of the front \ac{DBR} allowing more light to escape the cavity. The power given in figure \ref{fig:hhi_soa_sweep} is not corrected for coupling or routing component loss and so the power on-chip is expected to be higher. 

At currents above \SI{100}{\mA}, the \SI{300}{\um} \ac{SOA} begins to decrease in power and by \SI{150}{\mA} the optical power has drastically reduced. While more investigation is required, this is likely due an increase in temperature of the diode reducing the efficiency of the diode. The effect is elastic so reducing the driving current recovers the efficiency of the laser. This breakdown could possibly be remedied through better thermal management.

% Fluctuations in power at higher driving current is likely attributed to mode competition. At very high currents, the \SI{300}{\um} \ac{SOA} laser starts to decrease in power. This will be due to the efficiency of the laser diode at higher temperatures.

Figure \ref{fig:hhi_dbr_sweep} shows specta of the lasers while tuning the \acp{DBR} through current injection. The lasing is single-mode with a \ac{FWHM} of \SI{30}{pm} which is limited to the resolution of the \ac{OSA} used (Anritsu MS9740A). However, the laser only achieves \SI{30}{dB} of sideband suppression when \acp{DBR} are suitably tuned. Both the laser configurations exhibit similar spectra. This suppression is much less than the \SI{50}{dB} found in the Oclaro lasers of similar design. These sidebands are likely due to the reflected wavelength of the \acp{DBR} being quite broad allowing multiple modes to co-exist. 

By tuning the front and rear \acp{DBR}, the lasing wavelength can be changed. Passing a current through the \acp{DBR} causes them to heat, expanding the grating period and reducing the peak reflected wavelength. With a current up to \SI{40}{\mA}, the wavelength of the laser can be tuned by around \SI{3}{nm}. Fine tuning of the wavelength can be achieved by adjusting the temperature of the device or current injection of the \ac{cipm} within the cavity. 

%Laser spectrum shows multi-modal operation. This is likely due to broad reflection peaks in wavelength allowing simultaneous modes in the cavity. Sidebands can be suppressed through current injection of the tunable \acp{DBR}. However, they are still only around \SIrange{30}{40}{dB} below the peak. 

\begin{figure}[tp]
	\centering
	\small	
	\def\svgwidth{0.9\textwidth} 
	\import{chapters/chapter06/fig06/Spectra/}{spectra.pdf_tex}
	\caption[Wavelength sweep through current injection of tunable DBRs]{Current injection of the tunable \acp{DBR} causing a change of peak reflected wavelength. The lasers show multi-modal operation and a sideband suppression of only \SI{30}{dB}.}
	\label{fig:hhi_dbr_sweep}
\end{figure}

%The wavelength can be tuned but around \SI{3}{nm} by current injection of the tunable \acp{DBR}. Typical values are less than \SI{2}{V}. High driving current can cause heating of the cavity causing the wavelength to increase. 

\subsection{Gain Switching}

We can characterise the on-chip laser ability to gain switch, which is important for phase randomisation in \ac{QKD}. 

The \ac{SOA} within the laser cavity is connected to a \ac{gcpw} line for high-speed operation. Using a bias tee, the DC offset can be changed whilst an RF signal is provided to demonstrate gain switching. This is shown in figure \ref{fig:gain_switch_test} with varied DC offsets.

A \ac{ppg} provided a \SI{1}{GHz} signal, Arroyo 6305 provided the stable current source. The \ac{ppg} also provided a synchronisation signal to the detectors through an optical channel. \Ac{snspd} was used for detection and events correlated on a PicoQuant Hydraharp. A histogram could then be created. A \ac{voa} was used to stop the laser light from saturating the detectors.

\begin{figure}[tp]
	\centering
	\small	
	\def\svgwidth{\textwidth} 
	\import{chapters/chapter06/fig06/laser_gain_switch/}{gain_switch_test.pdf_tex}
	\caption[Gain switching test of the HHI on-chip lasers]{Gain switching test of the on-chip lasers. A bias tee mixes a DC offset with an RF signal. A \SI{1}{GHz} \SI{2}{\Vpp} square wave was used and the DC offset was varied around the lasing threshold. the \SI{20}{mA} (left), \SI{30}{\mA} (middle), \SI{40}{\mA} (right). \SI{1}{GHz} \SI{2}{\Vpp} square wave from \ac{ppg}. Laser threshold around \SI{30}{mA}. Pulses are around \SI{120}{ps} \ac{FWHM}.}
	\label{fig:gain_switch_test}
\end{figure}

A spectrum of the gain switched laser is shown in figure \ref{fig:gain_switch_spectrum}. The spectrum is broadened with is usual for gain switched lasers. The sideband suppression is reduced to only \SI{10}{dB}.

\begin{figure}[tp]
	\centering
	\small	
	\def\svgwidth{0.9\textwidth} 
	\import{chapters/chapter06/fig06/Spectra/}{spectrum_gain_switch.pdf_tex}
	\caption[Spectrum of the gain switched laser]{A spectrum of the gain switched laser when the DC offset is just below threshold (\SI{30}{mA}) with a \SI{2}{\Vpp} \SI{1}{\GHz} square wave. The spectrum is broadened which is usual for a gain switched laser. The sideband suppression is also reduced to only around \SI{10}{dB}. Such a source would likely need filtering before being useful for quantum experiments.}
	\label{fig:gain_switch_spectrum}
\end{figure}

\subsection{Laser Seeding}

The transmitter (shown in figure \ref{fig:hhi_laser_seeding}) contains two methods possible to achieve laser seeding, which are shown in figure \ref{fig:las_seed_schem}. Due to the misalignment of the \ac{vga} and chip, only design \ref{fig:las_seed_test}a was accessible for testing. Figure \ref{fig:las_seed_test} shows the initial test of driving the two transmitters with \SI{1}{GHz} square waves. 

The setup is as in the gain switching test except the \ac{ppg} provided two RF signals and a second Arroyo box was used to provide the two lasers with independent DC offsets. Pulses were attenuated an measured with an \ac{snspd}. 

\begin{figure}[tp]
	\centering	
	\def\svgwidth{0.9\textwidth} 
	\import{chapters/chapter06/fig06/}{laser_seeding_test.pdf_tex}
	\caption[Integrated laser seeding test]{By applying \SI{1}{GHz} square waves to both lasers, we can get pulses that should be phase randomised. However,  this will need testing. Pulses show \SI{120}{ps} \ac{FWHM}.}
	\label{fig:las_seed_test}
\end{figure}

\subsection{Quantum Random Number Generation}

\begin{figure}[t]
	\centering
	\includegraphics[width=\textwidth]{schematics/qrng.pdf}
	\caption[Schematic for InP quantum random number generation]{Schematic for the \ac{qrng} using on-chip laser sources and fast \ac{pd} for homodyne measurement. Loss compensation \acp{mzi} allow the lasers to be tuned in power and an \ac{mzi} acts as a beam splitter with a tunable splitting ratio.}
\end{figure}

Random numbers are required for many parts of \ac{QKD}. One way of generating this randomness is with \acp{qrng} which has been demonstrated on \ac{InP} \cite{FrancescoThesis,Abellan2016}. However, there has not yet been a single device which fulfils both the requirements for quantum random number generation \textit{and} quantum key generation.

The chip shown in figure \ref{fig:hhi_laser_seeding} contains both a \ac{qrng} and a laser seeded \ac{QKD} transmitter. Together with electronics to directly read the randomness and generate \ac{QKD} states this could been a fully packaged system.

\section{Improved Transmitters}

There are many ways of improving the \ac{QKD} transmitters. For example, encoding in time-bin using a delay line to separate the pulses or generating different states using independent lasers.

\begin{sidewaysfigure}
	\centering
	\includegraphics[width=0.8\textwidth]{QKD_transmitter}
	\caption[Latest generation InP QKD Transmitter]{Latest generation HHI indium phosphide transmitter. The \SI[product-units=power]{6x4}{mm} chip contains a few ways to create BB84 states for QKD. Firstly, we have designs to compare \ac{dfb} and \ac{DBR} lasers. Secondly, we can use a delay line to separate the time bins. Finally, we have multiplexed lasers to pulse independently lasers for each state.}
	\label{fig:hhi_gds}
\end{sidewaysfigure}

\subsubsection*{Multiple Laser Sources}

\begin{figure}[t]
	\centering
	\def\svgwidth{\textwidth}
	\import{chapters/chapter06/fig06/schematics/}{gain_switch_transmitter.pdf_tex}
		\caption[Schematic of the multiple laser source transmitter]{Schematic of the multiple laser source transmitter. By using independent lasers to generate each of the different states, we can gain switch each to ensure phase randomisation. A relatively simple optical circuit can then be used to encode the four BB84 states in time-bins. The loss compensation \ac{mzi} can match the losses between the two paths in the \ac{amzi}. The phase compensation allows fine tuning of the relative phases in the X basis states.}
	\label{fig:multiple_lasers_tx}
\end{figure}

We introduce multiple laser sources into the circuit which can each be independently gain-switched to generate the four BB84 states. 

From chapter \ref{chap:hom}, we can see that laser wavelengths can be overlapped very precisely. Therefore, introducing multiple laser sources shouldn't introduce any side-channels for Eve or Mallory to exploit.

The top laser is pulsed to give a $\ket{0}$ state as it will go through the delay line and end up as a late pulses. The bottom laser will go through the short arm of the \ac{amzi} and will make a $\ket{1}$ state.

For the X basis states, the different paths will cause a relative phase to be introduced.

$\ket{+}$:

\begin{equation}
	a^\dagger_e \xrightarrow{\text{\acs{bs}}} \frac{1}{\sqrt{2}} (a^\dagger_e + b^\dagger_e) \xrightarrow{\text{\acs{amzi}}} \frac{1}{\sqrt{2}} (a^\dagger_l + b^\dagger_e) \xrightarrow{\text{\acs{bs}}} \frac{1}{2}(a^\dagger_e + a^\dagger_l - b^\dagger_e + b^\dagger_l)
\end{equation}

$\ket{-}$:

\begin{equation}
	b^\dagger_e \xrightarrow{\text{\acs{bs}}} \frac{1}{\sqrt{2}} (a^\dagger_e - b^\dagger_e) \xrightarrow{\text{\acs{amzi}}} \frac{1}{\sqrt{2}} (a^\dagger_l - b^\dagger_e) \xrightarrow{\text{\acs{bs}}} \frac{1}{2}(- a^\dagger_e + a^\dagger_l + b^\dagger_e + b^\dagger_l)
\end{equation}

So when looking at the output port, we get a relative phase between the two different pulsed lasers.

\subsubsection*{Delay Line Time-Bin Encoding}

\begin{sidewaysfigure}
	\centering
	\includegraphics[width=\textwidth]{./schematics/hhi_transmitter.pdf}
	\caption[Schematic of the HHI full transmitter]{Schematic of the HHI transmitter}
	\label{fig:hhi_transmitter}
\end{sidewaysfigure}

While the previous generation of devices included delay lines to encode timing information, they we too lossy to be used. We have included a delay line in the design of this chip which should have lower losses. This should reduce the complexity of the control electronics and make the system compatible with a gain-switched DFB laser.

\section{Outlook}

While integrated devices have been used to demonstrate \ac{QKD} \cite{Sibson2017InP, Sibson2017Si, semenenko2019mdi, zhang2019integrated, ma2016silicon, paraiso2019}

This chapter has extended the use of integrated devices for \acl{QKD}.

%=========================================================