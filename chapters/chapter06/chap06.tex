%
% File: chap06.tex
% Author: Henry Semenenko
% Description: Laser Seeding
%
% Set the graphics path to find figures
\graphicspath{{./chapters/chapter06/fig06/}}

\let\textcircled=\pgftextcircled
\chapter{Integrated Laser Seeded Transmitter}
\label{chap:laser-seeding}

%=======
\section{QKD Transmitter Requirements}
\label{sec:sec06}

A \ac{QKD} transmitter will need to meet certain requirements as we have discussed before.

\section{Laser Seeding}

Laser seeding provides a method of generating \ac{QKD} states by overcoming some of the issues with gain switching lasers while also reducing certain electrical requirements.. This has been performed using fibre lasers \cite{}. In this chapter, we will demonstrate the first integrated laser seeding transmitter for time-bin encoding \ac{QKD}. 

\section{Integrated Laser Seeded Transmitter}

There are no circulators or isolators currently available on-chip, so other methods of laser seeding need to be considered.

Figure \ref{fig:hhi_laser_seeding} shows the GDS for the integrated device. The complexity of integrated devices is 

\begin{sidewaysfigure}
	\includegraphics[width=\textwidth]{Laser_Seeding.png}
	\caption[InP laser seeding transmitter with QRNG]{This shows the layout of the laser seeded transmitter device fabricated by HHI. The chip measures \SI[product-units=power]{6x4}{mm} and contains two laser seeding prototype circuits, a homodyne \ac{qrng} and test structure to measure laser and waveguide performances. This demonstrates the complexity and compactness possible in the integrated platform.}
	\label{fig:hhi_laser_seeding}
\end{sidewaysfigure}

The electrical connections for integrated devices is challenging, especially in test devices where real estate is valuable. In figure \ref{fig:hhi_pcb} we show a PCB which is required to connect most (but not all) components of the chip. 

\begin{figure}[tbp]
	\includegraphics[width=\textwidth]{HHI_PCB/HHI_PCB_render_2.png}
	\caption[PCB breakout for an InP integrated circuit]{PCB designed to connect an \ac{InP} integrated device to control and test equipment. Packaging integrated devices remains challenging due to the small size of the chips. This is especially true when high speed connections are required.}
	\label{fig:hhi_pcb}
\end{figure}

\section{Results}

The transmitter (shown in figure \ref{fig:hhi_laser_seeding}) contains two methods possible to achieve laser seeding.

\section{Outlook}

This chapter has extended the use of integrated devices for \acl{QKD}.

\subsection{Transmitter and QRNG}

Random numbers are required for many parts of \ac{QKD}. One way of generating this randomness is with \acp{qrng} which has been demonstrated on \ac{InP} \cite{FrancescoThesis,Abellan2016}. However, there has not yet been a single device which fulfils both the requirements for quantum random number generation \textit{and} quantum key generation.

The chip shown in figure \ref{fig:hhi_laser_seeding} contains both a \ac{qrng} and a laser seeded \ac{QKD} transmitter. Together with electronics to directly read the randomness and generate \ac{QKD} states this could been a fully packaged system.

%=========================================================