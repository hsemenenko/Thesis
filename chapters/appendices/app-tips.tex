%
% file: app0A.tex
% author: Henry Semenenko
% description: 100 tips for doing a PhD
%

\chapter{100 Tips for Doing a PhD}
\label{app:app01}

\begin{enumerate}
	\item Always go to the bar with someone more senior than you
	\item Ask forgiveness, not permission
	\item Remember that you will forget things
	\item Always make figures on a white background
	\item Timelines are useful to give you perspective
	\item It is impossible to keep to a timeline
	\item Work out how long you think something will take, and then multiple by 10 \cite{Hofstadter}
	\item Academic clocks run \emph{at least} 5 minutes late
	\item Be nice to the admin team, they know where your supervisor is
	\item Long days $\ne$ productivity
	\item Coffee is your friend
	\item No one else has a clue what they're talking about either
	\item You will make stupid mistakes
	\item Equipment will break
	\item Where a paper is published has no relation to work's quality
	\item If someone on Stack Overflow hasn't solved your problem, you're probably in too deep
	\item Learn \LaTeX{} and never look back
	\item The complexity of coding \LaTeX{} is inversely proportional to how long you expect it to take
	\item Aesthetics are almost as important as content
	\item Remember to take a holiday
	\item Don't feel guilty about ignoring emails on holiday
	\item Take holidays \emph{after} conferences so that you're already over the jet lag
	\item Backup your data
	\item Most meetings are optional
	\item Imposter syndrome is real \cite{langford1993}
	\item Only be sassy in person, don't leave written evidence
	\item If you leave something until the last minute it only takes a minute
	\item Give your files sensible names
	\item Don't rely on collaborators
	\item If it's important it's worth a second email
	\item Your supervisor isn't always right
	\item If it is stupid but it works, it isn't stupid
	\item Don't fix what ain't broke
	\item Don't update software unless you have to
	\item Publications don't fairly represent the number of failed attempts
	\item Try not to be over ambitious i.e. 100 is a big number
	\item Don't ask questions that are ``more of a comment, really''
	\item The poster title \emph{is} the abstract
	\item ``Everything not saved will be lost'' - Nintendo "Quit Screen"
	\item Dolly Parton didn't specify \textit{which} 9 to 5 i.e. working hours are flexible
	\item The best way to find a typo is to click submit
	\item Your family won't understand what your PhD is about no matter how many times you try
	\item Finish first, perfect later
	\item It's never going to be perfect
	\item Someone's lack of provisions during their PhD doesn't excuse your lack of provisions
	\item Do something badly enough once and no one will ask you to do it again
	\item Always tell someone the deadline is earlier than it actually is
	\item Original doesn't mean good
	\item Tradition is a terrible reason to do something
	\item Always save the data, not just a plot
	\item Start saving early for when you run out of funding 
	\item It's important to admit when you don't know or understand something
	\item Remember it's not your money when buying something
	\item Make sure to spend all of your travel budget
	\item Free food tastes better
	\item It's never too early to start writing your thesis
	\item Don't compare yourself to others
	\item Save plots as vector images
	\item Don't expect anything to work the first time
	\item Find something positive to say when giving feedback
	\item Simple doesn't mean easy
	\item Bodge jobs don't save time in the long run
	\item Your postdoc is your real supervisor
	\item Keep up with a hobby
	\item Experiments on a Friday never work, best not to bother
	\item Don't volunteer information to the safety officer about that stupid thing you did 
	\item Be nice to fellow PhD students, they'll be reviewing your future papers
	\item Never half-ass two things, whole-ass one thing
	\item Get over your fear of asking stupid questions
	\item Learn to code sooner rather than later
	\item For the love of God, put useful comments in your code
	\item Sometimes working on weekends is necessary
	\item Writing a thesis takes longer than you think
	\item ``A picture is worth a thousand words'' doesn't apply to your thesis word count
	\item Prioritise work that you can actually write about in your thesis
	\item Your thesis is as much for you as it is for your examiners 
	\item No one is going to do it for you
	\item Learn when to take a break
	\item "How's writing going?" never gets less annoying
	\item The stages of thesis writing are surprisingly similar to those of grief
	\item Don't work in your office if you want to get work done 
	\item 80\% of a talk should be motivation
	\item Tell 'em what you're going to say, then tell 'em, then tell 'em what you told 'em
	\item Don't expect your supervisor to make your life easier
	\item It isn't unusual for you to know more than your supervisor
	\item It only has to be `good enough'
	\item Always write the abstract last
	\item Have your viva in the afternoon so that your external has to leave to travel back
	\item There is a fine line between writing a background chapter and plagiarism
	\item Check your figures in black and white
	\item It is quite alright to admit that you were wrong
	\item Nothing will ever feel finished
	\item ``The best thesis defence is a good thesis offence'' \cite{xkcd1403}
	\item Remember to label the label maker first
	\item A good way to be more productive is to lower your standards
	\item Lent equipment has a tendency to disappear
	\item You don't finish a thesis, you abandon it 
	\item Don't waste time writing a list of 100 tips
	\item You should be writing your thesis now instead of reading this
	\item The only way to find out how to do a PhD is to do one. Therefore, all advice is useless\cite{richardbutterworth}
\end{enumerate}