%
% file: app0A.tex
% author: Henry Semenenko
% description: 100 tips for doing a PhD
%

\chapter{100 Tips for Doing a PhD}
\label{app:app01}

\begin{enumerate}
	\item Always go to the bar with someone more senior than you
	\item Ask forgiveness, not permission
	\item Remember that you will forget things
	\item Always make figures on a white background
	\item Timelines are useful to give you perspective
	\item It is impossible to keep to a timeline
	\item Work out how long you think something will take, and then multiple by 10 \cite{Hofstadter}
	\item Be nice to the admin team, they know where your supervisor is
	\item Long days $\ne$ productivity
	\item Coffee is your friend
	\item No one else has a clue what they're talking about either
	\item You will make stupid mistakes
	\item Equipment will break
	\item Where a paper is published has no relation to its quality of work 
	\item If someone on Stack Overflow hasn't solved your problem, you're probably in too deep
	\item Learn \LaTeX{} and never look back
	\item Aesthetics are almost as important as content
	\item Remember to take a holiday
	\item Don't feel guilty about ignoring emails on holiday
	\item Take holidays \textit{after} conferences so that you're already over the jet lag
	\item Backup your data
	\item Most meetings are optional
	\item Imposter syndrome is real \cite{langford1993}
	\item Only be sassy in person, don't leave written evidence
	\item If you leave something until the last minute it only takes a minute
	\item Give your files sensible names
	\item Don't rely on collaborators
	\item
	\item
	\item
	\item
	\item
	\item
	\item
	\item
	\item
	\item
	\item
	\item 
	\item 
	\item 
	\item 
	\item 
	\item 
	\item 
	\item
	\item
	\item
	\item
	\item
	\item
	\item
	\item
	\item
	\item
	\item
	\item
	\item
	\item
	\item
	\item
	\item
	\item
	\item
	\item
	\item
	\item
	\item
	\item
	\item
	\item
	\item 
	\item 
	\item 
	\item 
	\item 
	\item 
	\item 
	\item
	\item
	\item
	\item
	\item
	\item
	\item
	\item
	\item
	\item
	\item
	\item
	\item
	\item
	\item
	\item
	\item
	\item
	\item
	\item
	\item
	\item You should be writing your thesis now instead of reading this
\end{enumerate}








\