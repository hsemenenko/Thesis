%
% File: chap03.tex
% Author: Henry Semenenko
% Description: HOM Interference
%
% Set the graphics path to find figures
\graphicspath{{./chapters/chapter03/fig03/}}

\let\textcircled=\pgftextcircled
\chapter[HOM Interference Between Integrated Devices]{Hong-Ou-Mandel Interference Between Integrated Devices}
\label{chap:hom}
%=======

\ac{HOM} interference is a fundamental tool in any quantum engineer's toolbox. It underpins a range of quantum technologies ranging from computing to communication. While it is well understood, generating states for quantum interference at the high speeds required for modern telecommunication remains practically challenging. In this chapter, we experimentally demonstrate that \ac{InP} devices fulfil all the requirements of state-of-the-art Hong-Ou-Mandel interference while also being practically scalable. 

\section{Hong-Ou-Mandel Interference}

\Ac{HOM} interference is a quantum phenomena where two single photons incident on a beam splitter will interfere and bunch. This is an important process that is fundamental is many quantum information technologies.

\subsection{Coherent States on a Beam Splitter}

Consider two coherent states, $\ket{\alpha}$ and $\ket{\beta}$, incident on a beam splitter. Before the beam splitter, the states can be written as

\begin{equation}
\begin{split}
	\ket{\alpha} = \text{D}(\alpha)\ket{0} = e^{\alpha a^\dagger - \alpha^\ast a}\ket{0} = e^{-\frac{|\alpha|^2}{2}} e^{-\alpha^\ast a} e^{\alpha a^\dagger}\ket{0}\\
	\ket{\beta} = \text{D}(\beta)\ket{0} = e^{\beta b^\dagger - \beta^\ast b}\ket{0} = e^{-\frac{|\beta|^2}{2}} e^{-\beta^\ast b} e^{\beta b^\dagger}\ket{0}
\end{split}
\end{equation}
which, before the beam splitter, can be written jointly as
\begin{equation}
	\text{D}(\alpha)\otimes\text{D}(\beta)\ket{0}
\end{equation}

For generality, we can consider a beam splitter in terms of reflections, $r$, and transmissions, $t$, which yields the transformation matrix
\begin{equation}
	\left(
	\begin{matrix}
		t & r \\
		-r & t
	\end{matrix}
	\right)
	\quad\text{such that}\quad
	|r|^2 + |t|^2 = 1 
	\quad\text{and}\quad
	r^\ast t + r t^\ast = 0
\end{equation}
Using this beam splitter, where modes $a$ and $b$ are the incoming channels and $c$ and $d$ are the outgoing, the transformations are
\begin{equation}
	a^\dagger \rightarrow rc^\dagger + td^\dagger, \quad a \rightarrow r^\ast c + t^\ast d, \quad b^\dagger \rightarrow tc^\dagger - rd^\dagger, \quad b \rightarrow t^\ast c - r^\ast d
\end{equation}
so after the beam splitter, the state becomes
\begin{align}
	\text{D}_a(\alpha)\otimes\text{D}_b(\beta)\ket{0} &=  e^{\alpha a^\dagger - \alpha^\ast a} e^{\beta b^\dagger - \beta^\ast b}\ket{0}\\
	&\longrightarrow  e^{\alpha (r c^\dagger + t d^\dagger) - \alpha^\ast (r^\ast c + t^\ast d)}e^{\beta (t c^\dagger - r d^\dagger) - \beta^\ast (t^\ast c - r^\ast d)}\ket{0}\\
	&=  e^{(\alpha r + \beta t) c^\dagger - (\alpha^\ast r^\ast + \beta^\ast t^\ast) c} e^{(\alpha t - \beta r) d^\dagger - (\alpha^\ast t^\ast - \beta^\ast r^\ast) d}\ket{0}\\
	&=  \text{D}_c(\alpha r + \beta t)\text{D}_d(\alpha t - \beta r)\ket{0}
\end{align}
given a 50:50 beam splitter, this gives the output $\text{D}_c(\frac{1}{\sqrt{2}}\big(i\alpha + \beta\big))\text{D}_d(\frac{1}{\sqrt{2}}\big(\alpha - i\beta\big))$. The single photon detectors are threshold, meaning that single photon events are not distinguishable from multi-photon events. The two detectors $\text{Det}_1$ and $\text{Det}_2$ have efficiencies $\eta_1$ and $\eta_2$, respectively. Therefore, the probability of a click in $\text{Det}_\text{c}$ is given by
\begin{align}
	\text{P}(\text{Det}_1 \text{ click}) &= \sum_{n=0}^\infty \frac{|\alpha r + \beta t|^{2n} e^{-|\alpha r + \beta t|^2}}{n!}(1-(1-\eta_1)^n)\\
	&= 1 - e^{-|\alpha r + \beta t|^2 \eta_1}
\end{align}
and equivalently,
\begin{equation}
	\text{P}(\text{Det}_2 \text{ click}) = 1 - e^{-|\alpha t - \beta r|^2 \eta_2}
\end{equation}
Therefore, the probability of a coincidence click, with both coherent states overlapped on a beam splitter, is
\begin{align}
	\text{P}_\text{HOM}\text{(coincidence)} &= \text{P}(\text{Det}_1 \text{ click}) \times \text{P}(\text{Det}_2 \text{ click})\\
	&= \left(1 - e^{-|\alpha r + \beta t|^2 \eta_1}\right)\left(1 - e^{-|\alpha t - \beta r|^2 \eta_2}\right)
\end{align}

\begin{figure}[tbp]
\begin{tikzpicture}
\begin{axis}[
    axis line on top,
	xlabel = {Reflectivity/Transmission},
	ylabel = {Visibility},
	width = 0.9\linewidth,
	height = 0.5\linewidth,
	cycle list name = RdYlGn-reversed,
	xmin = 0,
	xmax = 1,
	ymin = 0,
	ymax = 0.55,
	xtick pos=left,
	ytick pos=left,
	xtick = {0,0.25,0.5,0.75,1},
	ytick = {0,0.25,0.5},
    tick align=outside,
    grid=both,
    grid style={line width=.1pt, draw=gray!10},
    legend style={fill=blue!5, draw=gray!50}
	]
\addplot+[very thick] table[x index=0, y index = 1, col sep=comma] {./chapters/chapter03/fig03/HOM_Coherent/HOM_Bs_vis.dat};
\addlegendentry{0.01};
\foreach \y in {2,...,10}{
  \edef\temp{\noexpand\addlegendentry{\y}}
  \addplot+[very thick] table[x index=0,y index=\y, col sep=comma] {./chapters/chapter03/fig03/HOM_Coherent/HOM_Bs_vis.dat};
  \temp
}
\end{axis}
\end{tikzpicture}
	\caption[Coherent state photon number visibility]{Graph plotting the effect of an unbalanced beam splitter on the visibility of a HOM dip. Maximal visibility is found when the beam splitter is 50:50. Photon number also has an effect on visibility when using coherent states. Here we see average photon numbers ranging from 0.01 to 10. Both states are assumed to have equal average photon number.}
	\label{fig:HOM_BS_Vis}
\end{figure}

To get a visibility for the HOM dip, we need to consider two coherent states not interfering on a beam splitter. The distributions after a beam splitter is a Poisson distribution of $P(|r\alpha|^2 + |t\beta|^2)$ and $P(|t\alpha|^2 + |r\beta|^2)$. Therefore, the probability of coincidence is given by
\begin{align}
	\text{P}_\text{ind}(\text{coincidence}) &= \left(1 - e^{-\eta_1(|r\alpha|^2 + |t\beta|^2)}\right)\left(1 - e^{-\eta_2(|t\alpha|^2 + |r\beta|^2)}\right)
\end{align}
We want to consider pulses that are phase randomised relative to each other. For this, we can introduce a phase $e^{i\theta}$ into the pulse incident from channel $a$. The coincidence probabilities then become
\begin{align}
	&\text{P}_\text{HOM}\text{(coincidence)} = \frac{1}{2\pi}\int_0^{2\pi}\left(1 - e^{-|e^{i\theta}\alpha r + \beta t|^2 \eta_1}\right)\left(1 - e^{-|e^{i\theta}\alpha t - \beta r|^2 \eta_2}\right)\text{d}\theta\\
	&\text{P}_\text{ind}(\text{coincidence}) = \left(1 - e^{-\eta_1(|r\alpha|^2 + |t\beta|^2)}\right)\left(1 - e^{-\eta_2(|t\alpha|^2 + |r\beta|^2)}\right)
\end{align}
The visibility of the HOM dip can be calculated as
\begin{equation}
	\text{Visibility} = 1 - \frac{\text{P}_\text{HOM}}{\text{P}_\text{ind}}
\end{equation}

Using these equations, it is possible to estimate the visibility of a HOM dip given a beam splitter that is not 50:50. The effects of this will be seen in a reduced visibility. The average photon number of the pulses used will also affect the visibility as the detectors are threshold. In figure \ref{fig:HOM_BS_Vis} the effect of varying the photon number and using an unbalanced beam splitter on the visibility of interference.

\section{Sources and Requirements}
\label{sec:sources}

In this section, we discuss the main difficulties in performing	\ac{HOM} interference and some of the sources that can be used to fulfil the requirements. 

It is well understood that for two light pulses to interfere, they need to be indistinguishable. That is to say, for maximum interference, the two pulses need to have the same wavelength; arrive at the same time; have equal intensities; be in the same polarisation; and have the same pulse shape.

\subsection{Single Photon}

\subsection{Weak Coherent States}

A source for \ac{QKD} will need to fulfil a few requirements listed here:

\begin{itemize}
	\item Good extinction ratio
	\item Good timing jitter
	\item Narrow linewidth/Long coherence
	\item Linearly polarised
	\item Phase randomisation
\end{itemize}

Several methods can be conceived to generate \acp{wcp}.

\begin{itemize}
	\item Gain switched lasers (filter required and timing jitter problems)
	\item Laser seeding	(Multi-level and expensive control required, filtering required)
	\item Intensity modulation
\end{itemize}

\section{Experiment}

In this section, we detail the experimental setup used to demonstrate \ac{HOM} interference at errors and speeds required to perform state-of-the-art \ac{MDI}.

\begin{figure}[tbp]
	\centering
	\includegraphics[width=\textwidth]{Experiment}
	\caption[HOM experimental setup]{Experimental setup of of the \ac{HOM} interference experiment with two identical \ac{InP} chips.}
	\label{fig:hom_experiment}
\end{figure}

The experiment uses two \ac{InP} devices which will we give the usual labels: Alice and Bob. Each device measures only \SI[product-units=power]{6x2}{mm} and contains all the required photonic components to generate the required states to perform \ac{HOM} interference at GHz rates. On only needs to compare the sizes of these optical components to fibre based optics to justify the benefits of integrated devices.

Each devices contains a \ac{CW} laser which consists of two \acp{DBR} and a \ac{SOA} which forms a cavity to create a laser only \SI{1}{mm} in length. Typical spectra of the two transmitter lasers is shown in figure \ref{fig:spectra} which demonstrates a \SI{<30}{pm} \ac{FWHM} and a \SI{>50}{dB} sideband suppression. Notice that the linewidth here is stated as \SI{<30}{pm} \ac{FWHM} as this is limited to the precision of the \ac{OSA} used. 

\begin{figure}[tbp]
	\centering
	\includegraphics[width=0.8\textwidth]{spectrum}
	\caption[Laser spectrum]{Typical spectra of the two on-chip laser demonstrating a \ac{FWHM} of \SI{<30}{pm}.}
	\label{fig:spectra}
\end{figure}

We note here that the narrow linewidth of the laser means that during no experiment with the \ac{InP} chips do we need to use a wavelength filter to clean up the light or optical pulses. This is important as it removed a component that would add cost to any system but also restrict the wavelength operation of the devices. Without the need for a filter, the lasers are free to operate in a wavelength that spans more than \SI{10}{nm}. This will be crucial as it is likely that \ac{QKD} systems will needed to be wavelength-division multiplexed (WDM) to meet the demands of high-speed networks.

Alice (and Bob) uses a stable current sources to drive an on-chip \ac{DBR} laser which demonstrates a \ac{FWHM} of < \SI{30}{pm}, although this is limited by the precision of the \ac{OSA} used.

\begin{figure}[tbp]
	\centering
	\includegraphics[width=0.8\textwidth]{Pulse}
	\caption[Pulse carving]{Pulse carving showing >20dB extinction ratio and a \ac{FWHM} of \SI{175}{\pico\second}.}
	\label{fig:pulses}
\end{figure}

\section{Hong-Ou-Mandel Interference Between GHZ Coherent States}

\begin{figure}[tbp]
	\centering
	\includegraphics[width=0.8\textwidth]{HOM}
	\caption[Hong-Ou-Mandel interference between integrated devices]{\ac{HOM} interference between two integrated devices demonstrating $46.5\pm0.8\%$ visibility.}
	\label{fig:HOM}
\end{figure}

\section{Phase Randomisation}

\section{Outlook}

%=========================================================