%
% File: chap03.tex
% Author: Henry Semenenko
% Description: HOM Interference
%
% Reset all acronyms 
\glsresetall
% Set the graphics path to find figures
\graphicspath{{./chapters/chapter03/fig03/}}

\let\textcircled=\pgftextcircled
\chapter[HOM Interference Between Integrated Devices]{Hong-Ou-Mandel Interference Between Integrated Devices}
\label{chap:hom}
%=======

\ac{HOM} interference is a fundamental tool in any quantum engineer's toolbox. It underpins a range of quantum technologies ranging from computing to communication. While it is well understood, generating states for quantum interference at the high speeds required for modern telecommunication remains practically challenging. In this chapter, we experimentally demonstrate that \ac{InP} devices fulfil all the requirements of state-of-the-art Hong-Ou-Mandel interference while also being practically scalable. 

\section{Hong-Ou-Mandel Interference with Coherent States}

\section{Sources and Requirements}
\label{sec:sources}

In this chapter, we discuss the main difficulties in performing	\ac{HOM} interference and some of the sources that can be used to fulfil the requirements. 

It is well understood that for two light pulses to interfere, they need to be indistinguishable. That is to say, for maximum interference, the two pulses need to have the same wavelength; arrive at the same time; have equal intensities; be in the same polarisation; and have the same pulse shape.

A source for \ac{QKD} will need to fulfil a few requirements listed here:

\begin{itemize}
	\item Good extinction ratio
	\item Good timing jitter
	\item Narrow linewidth/Long coherence
	\item Linearly polarised
	\item Phase randomisation
\end{itemize}

Several methods can be conceived to generate \acp{wcp}.

\begin{itemize}
	\item Gain switched lasers
	\item Laser seeding
	\item Intensity modulation
\end{itemize}

\section{Experiment}

In this section, we detail the experimental setup used to demonstrate \ac{HOM} interference at errors and speeds required to perform state-of-the-art \ac{MDI}.

The experiment uses two \ac{InP} devices which will we label Alice and Bob. Each devices uses a \ac{CW} laser which is carved into  

Alice (and Bob) uses a stable current sources to drive an on-chip \ac{DBR} laser which demonstrates a \ac{FWHM} of < \SI{30}{\pico\metre}, although this is limited by the precision of the \ac{OSA} used.

\begin{figure}[tbp]
	\centering
	\includegraphics[width=\textwidth]{Experiment.png}
	\caption[HOM experimental setup]{Experimental setup of of the \ac{HOM} interference experiment with two identical \ac{InP} chips.}
	\label{fig:hom_experiment}
\end{figure}

\subsection{Laser Linewidth}

\begin{figure}[tbp]
	\centering
	\includegraphics[width=0.8\textwidth]{spectrum.pdf}
	\caption[Laser spectrum]{Spectrum of the on-chip laser demonstrating a \ac{FWHM} of < \SI{30}{\pico\metre}.}
	\label{fig:spectrum}
\end{figure}

\subsection{Pulse Carving}

\begin{figure}[tbp]
	\centering
	\includegraphics[width=0.8\textwidth]{Pulse.pdf}
	\caption[Pulse carving]{Pulse carving showing >20dB extinction ratio and a \ac{FWHM} of \SI{175}{\pico\second}.}
	\label{fig:pulses}
\end{figure}

\subsection{Phase Randomisation}

\subsection{Coincidence Logic}

\section{Results}

\subsection{Hong-Ou-Mandel Interference Between GHZ Coherent States}

\begin{figure}[tbp]
	\centering
	\includegraphics[width=0.8\textwidth]{HOM.pdf}
	\caption[Hong-Ou-Mandel interference between integrated devices]{\ac{HOM} interference between two integrated devices demonstrating $46.5\pm0.8\%$ visibility.}
	\label{fig:HOM}
\end{figure}

\subsection{Phase Coherence}

\section{Outlook}

%=========================================================