%
% File: chap07.tex
% Author: Henry Semenenko
% Description: Conclusion
%
% Set the graphics path to find figures
\graphicspath{{./chapters/chapter07/fig07/}}

\let\textcircled=\pgftextcircled
\chapter{Conclusion}
\label{chap:conclusion}


%=======

It is becoming increasingly evident that the security of our modern networks will be forfeit as computing power inevitably increases. Quantum computing is one known threat against modern communication protocols. However, the situation is more dire than arguments over the `if' or `when' of quantum computing. The underlying security of all widely used cryptographic systems is based from assumed computationally-hard mathematical problems. There is no guarantee of the validity of these assumption, or even that they currently remain valid.

In developing a new precedent for secure cryptography, dubbed quantum-safe, there are two available routes. The first, which is favourable to the current network architecture, creates public-cryptography based off of new problems that are thought to be even more difficult to solve than those in current protocols. While these protocols will utilise the same classical computers, there is still no guarantee of their security. Even under full scrutiny of the scientific community, a new quantum algorithm may be imminent. 

\Ac{QKD} offers an entirely different solution with security founded in well established laws of physics. However, it has been demonstrated the even the security of \ac{QKD} is still only as good as the assumptions of the physical system. Unless a system if completely characterised, it has been demonstrated that assumed secret information could be leak or accessed through classical means. As such, the security could be compromised without violating the underlying physical principle of quantum mechanics.

Due to the nature of the key exchange, \ac{QKD} even offers the unique capability to detect an eavesdropper. 

However, we saw that the theoretical description of devices doesn't necessarily meet the physical implementation.

Integrated photonics is the only way we will be able to make many devices for ubiquitous quantum technology. This thesis describes how we can use them for quantum communication protocols. 

\Ac{HOM} interference is a vital phenomenon which underpins much of quantum photonic information. We demonstrated that interference between independently generate states from \acp{pic} was possible at \SI{431}{MHz}

\Ac{MDI} relaxing the assumptions require for a secure key exchange while facilitating a more accessible topology. A centralised resource that houses expensive single-photon detectors and processing equipment, many users can exchange keys and split the costs. Waveguide detectors on silicon devices would allow a further increase in the 

Integrated quantum photonics is still advancing and there are many improvements to devices that can be done. 

Key exchange protocols will be the first use-case of quantum-ready networks but the advantages certainly don't stop there. \Ac{QKD} networks have a lower tolerances that networks for other quantum protocols. However, they required the same architecture. Single-photon detection and low losses are vital. Multiplexing with classical data. Networks can evolve to provided limited local operations before distributed quantum computing will be accessible. 

The technological advances in the information age far exceeded all expectations. Should we be so naive to doubt that humanity has the capability of developing a global quantum network? This these is to a quantum internet what the first transistor is to a modern supercomputer. The building blocks are here, they just require some assembly.

%=========================================================