%
% File: chap05.tex
% Author: Henry Semenenko
% Description: Node chapter
% 
% Set the graphics path to find figures
\graphicspath{{./chapters/chapter05/fig05/}}

\let\textcircled=\pgftextcircled
\chapter{Fully Integrated Quantum Key Distribution}
\label{chap:node}

%=======
\section{Introduction}

Ideally, to create a system that can be widely adopted, as much of the system as possible would be integrated to ease scalability and reduce power, size and weight. In this section, we will introduce the first demonstration of a fully optically integrated \ac{QKD} system. The system is a hybrid between \ac{InP} transmitters and silicon receivers.

\section{QKD Metropolitan Network Requirements}
\label{sec:sec05}

\subsection{Practical Challenges}

Unlike laboratory conditions, the real world can't be controlled as thoroughly. This means that active feedback is required to keep timing, wavelength and polarisation overlapped during a key exchange. 

\section{Receiver Device}

By using a \ac{soi} device, we can replace some of the fibre optic components used in chapter \ref{chap:mdiqkd}.

Integrating the receiver for \ac{MDI} is appealing due to its simplicity. There have been integrated receiver devices in laser-written glass waveguides \cite{wang2019}. However, the detectors are still off-chip.

\subsection{Silicon Receiver}

The receiver device is an \ac{soi} device meaning we can take advantage of many well established photonic techniques.

As the measurement for \ac{MDI} in a time-bin encoding scheme is simply a beam splitter and two detectors, is it much simpler to integrated in a cryostat than other receivers (e.g. BB84, COW or DPS).

\subsection{Waveguide Integrated Detectors}

We can deposit some superconducting material onto the silicon and create \acp{snspd}. 

The high confinement in the waveguide means that coupling to the nanowire is strong giving an increased efficiency. 

\section{Experimental Setup}

\begin{sidewaysfigure}
	\centering
	\includegraphics[width=\textwidth]{NODE.png}
	\caption[Fully integrated QKD setup]{Experimental setup for fully integrated quantum key distribution. Alice and Bob use independent InP devices which generate BB84 states on-chip. The receiver (Charlie) is an SOI device with grating couplers and waveguide-integrated detectors.}
	\label{fig:node}
\end{sidewaysfigure}

The experimental setup is shown in figure \ref{fig:node}. 

\section{Results}

\section{Outlook}

\subsection{Metropolitan Network}

We should put this onto a real world network.

More electronics required.

Active feedback required (wavelength, polarisation, switching)

Potentially multiplexing (Time or wavelength)

\subsection{Resource Scaling}

Once we have a process of making one receiver or transmitter device, it becomes much easier to create many more. Given a big enough demand, the cost of these devices quickly decreases. Access to the networks becomes cheap, as \ac{InP} devices become more ubiquitous, and capacity of the networks can be scaled by taking advantage of the mature silicon photonics industry.

\subsection{Integrated Multi-User Switching}

A scalable multi-user network will always have to overcome routing quantum signals without introducing drastic loss into the system. Commercial optical routing systems exist with \SI{\sim 1}{dB} loss per switch. However, these systems will be large to meet the numbers required for city-wide networks. To overcome this, an integrated switch should fulfil the requirements. 

Optical on-chip switching can be achieved through a few methods. The first, and most widely used, phase modulation is done using \acp{topm}. Obviously, if we want to integrated both the switching and detection on the same device, the cryogenics will need to be able to provide enough cooling whilst heaters are on.

\acp{eopm}.

Opto-mechanical switches.

\subsection{CMOS Compatible Electronics}

The electronics for the system could eventually be integrated into the receiver device inside the cryostat. Electronics at low-temperatures have been demonstrated \cite{bardin2019}

%=========================================================