%
% File: abstract.tex
% Author: Henry Semenenko
% Description: Thesis abstract
%
% UoB guidelines:
%
% Each copy must include an abstract or summary of the dissertation in not
% more than 300 words, on one side of A4, which should be single-spaced in a
% font size in the range 10 to 12. If the dissertation is in a language other
% than English, an abstract in that language and an abstract in English must
% be included.


%Add bookmark for title page.
\pdfbookmark[0]{Abstract}{abstract}

\chapter*{Abstract}
%\begin{SingleSpace}
%To consider information as a quantum system offers many advantages in computation, sensing and communication. 

%Upholding the security of communication is of vital importance if we want to maintain the integrity of the national infrastructure that supports modern society.

Technology founded in quantum phenomena is set to revolutionise computation, sensing and communication. With an entirely different method of manipulating information, quantum computers in particular are able to offer significant advances when compared to their classical counterparts. Unrelenting research throughout the world implies that such machines are set to be the demise of public-key cryptography that is critical to modern society.

\Ac{QKD} offers a method to securely distribute randomness between distant parties using the fundamental nature of the universe. Protocols do not depend on assumed hard problems and so the security of the key does not decrease as computing power increases. However, \ac{QKD} systems have been susceptible to information leakage or hacks which compromises their security at the time of key exchange. As research demonstrations evolve into commercial systems, the security of a physical implementations must be addressed. 

% The unique nature of the protocol allows eavesdropper detection.

Device-independent protocols have since been introduced to relieve the possibility of secret information falling into the hands of an adversary. Using correlations between random events, the physical system does not contain any information about the key. Therefore, it is not susceptible to attacks or able to reveal the secret key. This simplifies the task of characterising a system to guarantee a secure key exchange.

Before \ac{QKD} can be widely adopted, a cost-effective and scalable platform must be developed. In the last decade, photonic integration has been refined to facilitate circuit complexity simply not possible with bulk alternative. The inherent robustness and phase-stability make it a excellent candidate for future quantum-secured networks.

In this thesis, will explore how integrated photonics can be deployed in device-independent \ac{QKD} protocols to both ensure practical security and enhance network accessibility. We will show how integrated components can be used to generate quantum states with high-fidelity and demonstrate quantum interference. The complexity of photonic integration will be explored to demonstrate new circuits for \ac{QKD} that will eventually form the backbone of quantum-secured networks.

%we will extend the functionality of integrated devices by considering how practical security can be increased. 


%Investigations have revealed that the theoretical models are not always satisfied by devices that exist in the real world. In order to reduce potential leaks of secret information, device-independent protocols were proposed. These allow the characterisation to be relaxed whilst fulfilling theoretical models which alleviates security concerns.

%Photonic integration has facilitated a complexity in quantum photonics not previously possible. 

%New methods are required to ensure that the \ac{QKD} systems match their 

%Leverage the knowledge from microelectronics, integrated optics provides a platform for scalable quantum information processing. 

%The complexity necessary for future quantum networks is something that can only be satisfied through the integrated photonics platform.

%Mass-manufacturable integrated devices will become the building blocks for future quantum networks.

%ensuring that security based on quantum physics is practical and accessible.

%\end{SingleSpace}
\clearpage